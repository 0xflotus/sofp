%% LyX 2.2.0 created this file.  For more info, see http://www.lyx.org/.
%% Do not edit unless you really know what you are doing.
\documentclass[english]{beamer}
\usepackage[T1]{fontenc}
\usepackage[latin9]{inputenc}
\setcounter{secnumdepth}{3}
\setcounter{tocdepth}{3}
\usepackage{babel}
\ifx\hypersetup\undefined
  \AtBeginDocument{%
    \hypersetup{unicode=true,pdfusetitle,
 bookmarks=true,bookmarksnumbered=false,bookmarksopen=false,
 breaklinks=false,pdfborder={0 0 1},backref=false,colorlinks=true}
  }
\else
  \hypersetup{unicode=true,pdfusetitle,
 bookmarks=true,bookmarksnumbered=false,bookmarksopen=false,
 breaklinks=false,pdfborder={0 0 1},backref=false,colorlinks=true}
\fi

\makeatletter
%%%%%%%%%%%%%%%%%%%%%%%%%%%%%% Textclass specific LaTeX commands.
 % this default might be overridden by plain title style
 \newcommand\makebeamertitle{\frame{\maketitle}}%
 % (ERT) argument for the TOC
 \AtBeginDocument{%
   \let\origtableofcontents=\tableofcontents
   \def\tableofcontents{\@ifnextchar[{\origtableofcontents}{\gobbletableofcontents}}
   \def\gobbletableofcontents#1{\origtableofcontents}
 }
 \newenvironment{lyxcode}
   {\par\begin{list}{}{
     \setlength{\rightmargin}{\leftmargin}
     \setlength{\listparindent}{0pt}% needed for AMS classes
     \raggedright
     \setlength{\itemsep}{0pt}
     \setlength{\parsep}{0pt}
     \normalfont\ttfamily}%
    \def\{{\char`\{}
    \def\}{\char`\}}
    \def\textasciitilde{\char`\~}
    \item[]}
   {\end{list}}

%%%%%%%%%%%%%%%%%%%%%%%%%%%%%% User specified LaTeX commands.
\usetheme[secheader]{Boadilla}
\usecolortheme{seahorse}
\title[Chapter 3: Logic of Types I]{Chapter 3: The Logic of Types, Part I}
\subtitle{Higher-order functions}
\author{Sergei Winitzki}
\date{December 2, 2017}
\institute[ABTB]{Academy by the Bay}
\setbeamertemplate{navigation symbols}{}

\makeatother

\begin{document}
\frame{\titlepage}
\begin{frame}{Types and syntax of functions that return functions}


\framesubtitle{``Curried functions'' in Scala}
\begin{itemize}
\item A function that returns a function:
\end{itemize}
\begin{lyxcode}
\textcolor{blue}{\footnotesize{}def~logWith(topic:~String):~(String~$\Rightarrow$~Unit)~=~\{}{\footnotesize \par}

\textcolor{blue}{\footnotesize{}~~~x~$\Rightarrow$~println(s\textquotedbl{}\$topic:~\$x\textquotedbl{})}{\footnotesize \par}

\textcolor{blue}{\footnotesize{}\}}{\footnotesize \par}
\end{lyxcode}
\begin{itemize}
\item Calling this function:
\end{itemize}
\begin{lyxcode}
\textcolor{blue}{\footnotesize{}val~statusLogger:~(String~$\Rightarrow$~Unit)~=~logWith(\textquotedbl{}Result~status\textquotedbl{})}{\footnotesize \par}

\textcolor{blue}{\footnotesize{}statusLogger(\textquotedbl{}success\textquotedbl{})}{\footnotesize \par}
\end{lyxcode}
\begin{itemize}
\item One-line syntax: \texttt{\textcolor{blue}{\footnotesize{}logWith(\textquotedbl{}Result
status\textquotedbl{})(\textquotedbl{}success\textquotedbl{})}} 
\item Alternative syntax \textendash{} ``\href{https://en.wikipedia.org/wiki/Currying}{Curried}''
function:
\end{itemize}
\begin{lyxcode}
\textcolor{blue}{\footnotesize{}val~logWith:~String~$\Rightarrow$~String~$\Rightarrow$~Unit~=~}{\footnotesize \par}

\textcolor{blue}{\footnotesize{}~~topic~$\Rightarrow$~x~$\Rightarrow$~println(s\textquotedbl{}\$topic:~\$x\textquotedbl{})}{\footnotesize \par}
\end{lyxcode}
\begin{itemize}
\item Syntax conventions: \texttt{\textcolor{blue}{\footnotesize{}x}} \texttt{\textcolor{blue}{\footnotesize{}$\Rightarrow$
y}} \texttt{\textcolor{blue}{\footnotesize{}$\Rightarrow$}} \texttt{\textcolor{blue}{\footnotesize{}z}}
means \texttt{\textcolor{blue}{\footnotesize{}x}} \texttt{\textcolor{blue}{\footnotesize{}$\Rightarrow$
(y}} \texttt{\textcolor{blue}{\footnotesize{}$\Rightarrow$}} \texttt{\textcolor{blue}{\footnotesize{}z)}}{\footnotesize \par}
\begin{itemize}
\item This is so because \texttt{\textcolor{blue}{\footnotesize{}f(g)(h)}}
means \texttt{\textcolor{blue}{\footnotesize{}(f(g))(h)}}{\footnotesize \par}
\end{itemize}
\end{itemize}
\end{frame}

\begin{frame}{Functions with fully parametric types}


\framesubtitle{``No argument type left non-parametric''}

Compare these two functions (note tuple type syntax):
\begin{lyxcode}
\textcolor{blue}{\footnotesize{}def~hypothenuse~=~(x:~Double,~y:~Double)~$\Rightarrow$~math.sqrt(x{*}x~+~y{*}y)}{\footnotesize \par}

\textcolor{blue}{\footnotesize{}def~swap:~((Double,~Double))~$\Rightarrow$~(Double,~Double)~=}{\footnotesize \par}

\textcolor{blue}{\footnotesize{}~~\{~case~(x,~y)~$\Rightarrow$~(y,~x)~\}~}{\footnotesize \par}
\end{lyxcode}
\begin{itemize}
\item We can rewrite \texttt{\textcolor{blue}{\footnotesize{}swap}} to make
the argument types \textbf{fully parametric}:
\end{itemize}
\begin{lyxcode}
\textcolor{blue}{\footnotesize{}def~swap{[}X,~Y{]}:~((X,~Y))~$\Rightarrow$~(Y,~X)~=~\{~case~(x,~y)~$\Rightarrow$~(y,~x)~\}~}{\footnotesize \par}
\end{lyxcode}
\begin{itemize}
\item (The first function is too specific to generalize the argument types.)
\item Note: Scala does not support a \texttt{\textcolor{blue}{\footnotesize{}val}}
with type parameters
\begin{itemize}
\item Instead we can use \texttt{\textcolor{blue}{\footnotesize{}def}} or
parametric classes/traits
\end{itemize}
\item More examples:
\end{itemize}
\begin{lyxcode}
\textcolor{blue}{\footnotesize{}def~id{[}T{]}:~(T~$\Rightarrow$~T)~=~x~$\Rightarrow$~x}{\footnotesize \par}

\textcolor{blue}{\footnotesize{}def~const{[}C,~X{]}:~(C~$\Rightarrow$~X~$\Rightarrow$~C)~=~c~$\Rightarrow$~x~$\Rightarrow$~c}{\footnotesize \par}

\textcolor{blue}{\footnotesize{}def~compose{[}X,~Y,~Z{]}(f:~X~$\Rightarrow$~Y,~g:~Y~$\Rightarrow$~Z):~X~$\Rightarrow$~Z~=~x~$\Rightarrow$~g(f(x))}{\footnotesize \par}
\end{lyxcode}
\begin{itemize}
\item Functions with fully parametric types \emph{are} useful despite appearances!
\end{itemize}
\end{frame}

\begin{frame}{Worked examples}

\begin{itemize}
\item For the functions \texttt{\textcolor{blue}{\footnotesize{}const}}
and \texttt{\textcolor{blue}{\footnotesize{}id}} defined above, what
is the value \texttt{\textcolor{blue}{\footnotesize{}const(id)}} and
what is its type? Write out the type parameters.
\item Define a function \texttt{\textcolor{blue}{\footnotesize{}twice}}
that takes a function $f$ as its argument and returns a \emph{function}
that applies $f$ twice. E.g., \texttt{\textcolor{blue}{\footnotesize{}twice((x:Int)}}
\texttt{\textcolor{blue}{\footnotesize{}$\Rightarrow$}} \texttt{\textcolor{blue}{\footnotesize{}x+3)}}
must return a function equivalent to \texttt{\textcolor{blue}{\footnotesize{}x}}
\texttt{\textcolor{blue}{\footnotesize{}$\Rightarrow$}} \texttt{\textcolor{blue}{\footnotesize{}x+6}}.
Find the type of \texttt{\textcolor{blue}{\footnotesize{}twice.}}{\footnotesize \par}
\item What does \texttt{\textcolor{blue}{\footnotesize{}twice(twice)}} do?
Test your answer on this expression: \texttt{\textcolor{blue}{\footnotesize{}twice(twice{[}Int{]})(x}}
\texttt{\textcolor{blue}{\footnotesize{}$\Rightarrow$}} \texttt{\textcolor{blue}{\footnotesize{}x+3)(10)}}.
What are the type parameters here?
\item Take a function with two arguments, fix the value of the first argument,
and return the function of the remaining one argument. Define this
operation as a function with fully parametric types:\texttt{\textcolor{blue}{\footnotesize{}}}~\\
\texttt{\textcolor{blue}{\footnotesize{}def firstArg{[}X, Y, Z{]}(f:~(X,
Y) $\Rightarrow$ Z, x0:~X):~Y $\Rightarrow$ Z = ???}}{\footnotesize \par}
\item Implement a function that applies a given function $f$ repeatedly
to an initial value $x_{0}$, until a given condition function \texttt{\textcolor{blue}{\footnotesize{}cond}}
returns true:
\end{itemize}
\begin{lyxcode}
\textcolor{blue}{\footnotesize{}def~converge{[}X{]}(f:~X~$\Rightarrow$~X,~x0:~X,~cond:~X~$\Rightarrow$~Boolean):~X~=~???}{\footnotesize \par}
\end{lyxcode}
\begin{itemize}
\item Infer types in \texttt{\textcolor{blue}{\footnotesize{}def p{[}}}...\texttt{\textcolor{blue}{\footnotesize{}{]}:}}...\texttt{\textcolor{blue}{\footnotesize{} =
f $\Rightarrow$ f(2)}}. Does \texttt{\textcolor{blue}{\footnotesize{}p(p)}}
work?
\item Infer types in \texttt{\textcolor{blue}{\footnotesize{}def q{[}}}...\texttt{\textcolor{blue}{\footnotesize{}{]}:}}...\texttt{\textcolor{blue}{\footnotesize{} =
f $\Rightarrow$ g $\Rightarrow$ g(f)}}. What are \texttt{\textcolor{blue}{\footnotesize{}q(q)}},
\texttt{\textcolor{blue}{\footnotesize{}q(q(q))}}?
\end{itemize}
\end{frame}

\begin{frame}{Exercises I}

\begin{enumerate}
\item For the function \texttt{\textcolor{blue}{\footnotesize{}id}} defined
above, what is \texttt{\textcolor{blue}{\footnotesize{}id(id)}} and
what is its type? Same question for \texttt{\textcolor{blue}{\footnotesize{}id(const)}}.
Does \texttt{\textcolor{blue}{\footnotesize{}id(id)(id)}} or \texttt{\textcolor{blue}{\footnotesize{}id(id(id))}}
work? 
\item For the function \texttt{\textcolor{blue}{\footnotesize{}const}} above,
what is \texttt{\textcolor{blue}{\footnotesize{}const(const)}}, what
is its type?
\item For the function \texttt{\textcolor{blue}{\footnotesize{}twice}} above,
what does \texttt{\textcolor{blue}{\footnotesize{}twice(twice(twice)))}}
do? Write out the type parameters. Test your answer on an example.
\item Define a function \texttt{\textcolor{blue}{\footnotesize{}thrice}}
that applies its argument function 3 times, similarly to \texttt{\textcolor{blue}{\footnotesize{}twice}}.
What does \texttt{\textcolor{blue}{\footnotesize{}thrice(thrice(thrice)))}}
do?
\item Define a function \texttt{\textcolor{blue}{\footnotesize{}ence}} that
applies a given function $n$ times.
\item Define a function \texttt{\textcolor{blue}{\footnotesize{}swapFunc(f)}}
with fully parametric types, which swaps arguments for any given function
\texttt{\textcolor{blue}{\footnotesize{}f}} of two arguments. To test: 
\begin{lyxcode}
\textcolor{blue}{\footnotesize{}def~f(x:~Int,~y:~Int)~=~x~-~y}\textsf{\textcolor{gray}{\footnotesize{}~//~check~that~f(10,~2)~gives~8}}{\footnotesize \par}

\textcolor{blue}{\footnotesize{}val~g~=~swapFunc(f)}\textsf{\textcolor{gray}{\footnotesize{}~~//~now~check~that~g(10,~2)~gives~(\textendash{}~8)}}{\footnotesize \par}
\end{lyxcode}
\item Infer types in \texttt{\textcolor{blue}{\footnotesize{}def r{[}}}...\texttt{\textcolor{blue}{\footnotesize{}{]}:}}...\texttt{\textcolor{blue}{\footnotesize{} =
p $\Rightarrow$ q $\Rightarrow$ p(t $\Rightarrow$ t(q))}}{\footnotesize \par}
\item Show that \texttt{\textcolor{blue}{\footnotesize{}def s{[}}}...\texttt{\textcolor{blue}{\footnotesize{}{]}:}}...\texttt{\textcolor{blue}{\footnotesize{} =
p $\Rightarrow$ p(q $\Rightarrow$ q(p))}} is not well-typed
\item Infer types in \texttt{\textcolor{blue}{\footnotesize{}def u{[}}}...\texttt{\textcolor{blue}{\footnotesize{}{]}:}}...\texttt{\textcolor{blue}{\footnotesize{} =
p $\Rightarrow$ q $\Rightarrow$ q(x $\Rightarrow$ x(p(q)))}}{\footnotesize \par}
\item Show that \texttt{\textcolor{blue}{\footnotesize{}def v{[}}}...\texttt{\textcolor{blue}{\footnotesize{}{]}:}}...\texttt{\textcolor{blue}{\footnotesize{} =
p $\Rightarrow$ q $\Rightarrow$ q(x $\Rightarrow$ p(q(x)))}} is
not well-typed
\end{enumerate}
\end{frame}

\end{document}
