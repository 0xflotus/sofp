
\chapter{The formal logic of types II. Higher-order functions\label{chap:Higher-order-functions} }

\subsection{Types and syntax of functions that return functions}

``Curried functions'' in Scala

A function that returns a function:

def logWith(topic: String): (String $\Rightarrow$ Unit) = \{

   x $\Rightarrow$ println(s\textquotedbl\$topic: \$x\textquotedbl )

\}

Calling this function:

val statusLogger: (String $\Rightarrow$ Unit) = logWith(\textquotedbl Result
status\textquotedbl )

statusLogger(\textquotedbl success\textquotedbl )

One-line syntax: logWith(\textquotedbl Result status\textquotedbl )(\textquotedbl success\textquotedbl ) 

Alternative syntax \textendash{} ``\href{https://en.wikipedia.org/wiki/Currying}{Curried}''
function:

val logWith: String $\Rightarrow$ String $\Rightarrow$ Unit = 

  topic $\Rightarrow$ x $\Rightarrow$ println(s\textquotedbl\$topic:
\$x\textquotedbl )

Syntax conventions: x $\Rightarrow$ y $\Rightarrow$ z means x $\Rightarrow$
(y $\Rightarrow$ z)

This is so because f(g)(h) means (f(g))(h) 

\subsection{Functions with fully parametric types}

``No argument type left non-parametric''

Compare these two functions (note tuple type syntax):

def hypothenuse = (x: Double, y: Double) $\Rightarrow$ math.sqrt(x{*}x
+ y{*}y)

def swap: ((Double, Double)) $\Rightarrow$ (Double, Double) =

  \{ case (x, y) $\Rightarrow$ (y, x) \} 

We can rewrite swap to make the argument types fully parametric:

def swap{[}X, Y{]}: ((X, Y)) $\Rightarrow$ (Y, X) = \{ case (x, y)
$\Rightarrow$ (y, x) \} 

(The first function is too specific to generalize the argument types.)

Note: Scala does not support a val with type parameters

Instead we can use def or parametric classes/traits

More examples:

def id{[}T{]}: (T $\Rightarrow$ T) = x $\Rightarrow$ x

def const{[}C, X{]}: (C $\Rightarrow$ X $\Rightarrow$ C) = c $\Rightarrow$
x $\Rightarrow$ c

def before{[}X, Y, Z{]}(f: X $\Rightarrow$ Y, g: Y $\Rightarrow$
Z): X $\Rightarrow$ Z = x $\Rightarrow$ g(f(x))

Functions with fully parametric types are useful despite appearances!

\subsection{Solved examples}

For the functions const and id defined above, what is the value const(id)
and what is its type? Write out the type parameters.

Define a function twice that takes a function $f$ as its argument
and returns a function that applies $f$ twice. E.g., twice((x:Int)
$\Rightarrow$ x+3) must return a function equivalent to x $\Rightarrow$
x+6. Find the type of twice.

What does twice(twice) do? Test your answer on this expression: twice(twice{[}Int{]})(x
$\Rightarrow$ x+3)(10). What are the type parameters here?

Take a function with two arguments, fix the value of the first argument,
and return the function of the remaining one argument. Define this
operation as a function with fully parametric types:\\
def firstArg{[}X, Y, Z{]}(f:~(X, Y) $\Rightarrow$ Z, x0:~X):~Y
$\Rightarrow$ Z = ???

Implement a function that applies a given function $f$ repeatedly
to an initial value $x_{0}$, until a given condition function cond
returns true:

def converge{[}X{]}(f: X $\Rightarrow$ X, x0: X, cond: X $\Rightarrow$
Boolean): X = ???

Infer types in def p{[}...{]}:... = f $\Rightarrow$ f(2). Does p(p)
work?

Infer types in def q{[}...{]}:... = f $\Rightarrow$ g $\Rightarrow$
g(f). What are q(q), q(q(q))?

\subsection{Exercises}

For the function id defined above, what is id(id) and what is its
type? Same question for id(const). Does id(id)(id) or id(id(id)) work? 

For the function const above, what is const(const), what is its type?

For the function twice above, what does twice(twice(twice))) do? Write
out the type parameters. Test your answer on an example.

Define a function thrice that applies its argument function 3 times,
similarly to twice. What does thrice(thrice(thrice))) do?

Define a function ence that applies a given function $n$ times.

Define a function swapFunc(f) with fully parametric types, which swaps
arguments for any given function f of two arguments. To test: 

def f(x: Int, y: Int) = x - y // check that f(10, 2) gives 8

val g = swapFunc(f)  // now check that g(10, 2) gives (\textendash{}
8)

Infer types in def r{[}...{]}:... = p $\Rightarrow$ q $\Rightarrow$
p(t $\Rightarrow$ t(q))

Show that def s{[}...{]}:... = p $\Rightarrow$ p(q $\Rightarrow$
q(p)) is not well-typed

Infer types in def u{[}...{]}:... = p $\Rightarrow$ q $\Rightarrow$
q(x $\Rightarrow$ x(p(q)))

Show that def v{[}...{]}:... = p $\Rightarrow$ q $\Rightarrow$ q(x
$\Rightarrow$ p(q(x))) is not well-typed

\section{Types of higher-order functions}

\begin{comment}
we will talk about the types of higher-order functions higher-order
functions are functions that return functions or that take functions
as arguments it very interesting thing happens when you consider functions
that return a function here is an example the function log width it
takes a string argument which is not itself a function so that's just
a normal argument non function but it returns the type which is a
function so it returns the value of type function from string to unit
why would we want to use such a function well for example we want
a logger function we want something that prints logging statements
and we want to prepare this logger so that whatever statemented prints
before that statement it also writes the topic of the statement let's
say some topics such as result or intermediate or something like this
to prepare this function we give a topic and then we return an expression
which is itself a function the function takes some argument X and
prints a string which is topic : X so then in the code we will say
something like this a status logger is a function from string to unit
well it's just going from strength to unit because println returns
unit it does not return any you school value this is just an example
so status logger is a function from string to unit which we obtain
by calling log width on the string argument result status and the
result will be that status logger will be a function that prints let's
say success but before success it prints result status so this is
a mistake actually this should be status log you're not a primary
logger I will fix that in the slides so this must be primary not primary
logging about status vulgar for this code to work so the result is
that status logger called on success will print result status which
is the topic : success and we can call the same function in the syntax
that is in one line so you see here we first made this call which
returns this value and then we call that same value sorry about this
area we call that same value on a success and we can do the same since
these are all values by first doing this first calling this function
getting the result which is a logging function and then calling this
login function on the success argument let's see how this works in
interactive scholar mode so you define a function takes topic of type
string and returns let's say string to unity as we discount decided
and that is going to be X going to print println topic X so we evaluate
this and then we say well status blogger equals walk with result status
then we say status lager or success so let's see now okay so we got
a printed value result status equals success so the same thing can
be done if we call logged with without status success so yeah same
same thing so you see what happens here is that the function log with
when called on this argument return the function which is this expression
so we're using we're using expressions that are anonymous functions
we could have a syntax that is like this as well just be I have done
something wrong this right so the syntax could be like this it's exactly
the same result so log West returns this expression which is itself
a function that function takes some X and prints this message which
is topic X the topic was already given as the argument here so here
we put in the topic right and in this function that we returned so
when we say status logger.log with result status the result Stiles
is already in that status logger inside this expression so when we
say status logger success it prints results that of success and exactly
the same is happening when we do this so first we do list log with
of result status that returns a function and that function can be
called on its argument which is success so that results in the syntax
that looks a little odd maybe which is a function with two pairs of
parenthesis that syntax is what I'm going to talk about next this
syntax is actually very important in functional programming because
we can write it in a much shorter way namely like this and why is
that well you see there are two ways of defining functions in Scala
in terms of syntax one way is to define deft right def and then function
name and then in parenthesis the right arguments the other way is
to say Val or let's say Val and then the type say that the type of
this Val is a function and then equals so still have echoes here but
it's going to be a function expression so here it's not so obvious
what the type is of log width because log with actually has two sets
of arguments one which you can see here log with is a function that
looks like it has two sets of arguments the first set of argument
here in the first pair of parentheses and the second set of arguments
here in the second set it's like a second pair of parentheses so an
alternative syntax for that is to write let me show you how to write
that so I would say Val log with let's call this log with one just
to make it a different name and the type of that I'm going to show
you different ways of putting types in it but let's say I want to
have a function that takes a topic and returns this so we can do this
right so we take a function which was returned here and we say that
function is returned when we give topic as argument so now if we use
this syntax IntelliJ is unhappy because it wants parameter types so
you need to specify types there are two ways of doing this I mean
this would not compile it's not just IntelliJ that's not happy you
need to specify types so in scala you need to specify the types in
quite a few situations although not everywhere so we will go through
this more slowly so what is the parameter type well the syntax for
parameters is that you put parentheses and you say type let's say
this is string so now it's happy about this but it's not happy about
this X let's put a parameter type in here as well now it's completely
happy and now look with one the type is string to string to unit now
we see it put parentheses here which is kind of more clear maybe so
we take us it's a look with one as a function that takes a string
and returns this which is itself a function that takes a string and
returns unit now there is a syntax convention where the second pair
of parenthesis this last one can be omitted so let me show you how
to do that I call this log with two and I specify the type string
string unit and then I say topic and then I do the same as I did here
topic returns this value okay it's happy you know red actually tells
me there are unnecessary parentheses so this is the syntax convention
I'm talking about the second to the well as the second pair of types
don't need parentheses around them so neither here I need parentheses
around these two although IntelliJ will not mind if I put them here
but they are not necessary it's exactly the same thing this is the
syntax convention basically the syntax convention is that this syntax
XYZ means this the parentheses are implied on the last pair not on
the first one now what would be the first type let me write this as
a comment so what is string to string to unit what would that mean
that would be a function that takes a string to string function as
the first argument or the only argument and returns unit so that would
be also a higher-order function but that function does not return
a function that returns unit it's argument is a function that's a
difference right the function that is an argument or a function that
is a return value so here we have a function whose return value is
a function here we I'm writing a type of a function whose argument
is a function in this case you need parentheses around the first pair
in the second case you don't so this don't need the parentheses around
these two this is not necessary so this is the syntax convention I'm
talking about the reason for the syntax convention is what we just
saw the natural way in which we call a function that returns a function
is that we want to write things like this we want to call the function
with an argument that returns a function we want to call that function
again with some other argument just like we did in this example we
we say status logger equals this and we call status logger of success
so we should be able to replace when we write status logger of success
we should be able to replace status logger with this expression right
so status logger is equal to this so what we can replace status longer
here with what it's equal to it will be very convenient if we didn't
need to write parenthesis like this again so these parentheses are
unnecessary as intellij tells me that's the convention so because
we don't write these parenthesis but that is natural because of this
replacement of the expression right so I can write this or I can write
this these expressions are equal inside you find status logger to
be that other thing because of the natural way in which we write this
naturally the type of this must be a function which is which is this
and so we don't want to write extra parentheses here either since
we don't write them here so if we did the opposite convention here
then we would have to write parentheses like this all the time and
that's not going to be well that's not going to bring us any advantage
so these two syntactic conventions go hand-in-hand for the reason
I just described very natural not to write extra parentheses here
and once we do that the type of F is actually this and we don't want
to write extra parentheses in that type either so this syntax is called
curried I put here a link to the Wikipedia page if you want to write
about why this is called like this is in faith in honor of a mathematician
whose name was curry now having understood how we write the syntax
for higher-order functions let's do something with these types with
the types of the functions to understand what we do in functional
programming let's contrast two different kinds of functions and I
selected this example of two functions the first function in the hypotenuse
is a calculation of the Pythagoras theorem so you take x and y as
two arguments and you return the lengths of the diagonal of a rectangle
with sides x and y now note again the syntax I'd like to call your
attention to the syntax in Scala this is a little quirky this function
actually has two arguments this is like a tuple but it is actually
not a tuple so in Scala there is a difference between a function that
takes two arguments like this one and the function that takes a tuple
as it's one argument in other words a function with one argument whose
type the function the type of that argument being a tuple of two elements
the difference in the syntax is extra parentheses so the next example
called swap is a function that takes a tuple as its own only sorry
as its only argument and returns a tuple also so this is a type expression
which means we take a tuple and we return a tuple extra pair an extra
pair of parentheses is necessary on the left but it is not necessary
on the right this is because of Scala's special feature that distinguishes
between two different function types so that we could rewrite the
first function as a function that takes a tuple of two doubles and
then I would have to put extra parentheses here and then I would have
to do pattern matching on the tuple on the right-hand side which would
make a code a little longer but it would be possible just a little
extra writing so in Scala usually arguments of functions are not tuples
it's not usually necessary but sometimes it is and then we do so this
walk is an example it takes a tuple of two doubles and returns again
a tuple of two doubles and in that tuple the order of two numbers
is reversed that's a very simple function now the interesting thing
about this function is that it doesn't actually use the fact that
its arguments have type double all that function does is exchange
the order the order of these two arguments so in principle this function
could with the same code this code remains the same it could work
on a tuple of any type let's say a tuple of types x and y and in order
to write this in scala we parameterize the argument types so the syntax
looks like this we introduce type parameters in square brackets and
once we introduce them we use them in the type and notation for this
function type annotation is whatever you write after the colon so
this is a type annotation in the type annotation and also on the right
hand side if you want you can use x and y as if they are defined but
actually these are type parameters so this function can work with
any two different types x and y it could be even they could be the
same or they could be different right now there's no convention sorry
no specification about them there's just two different type variables
x and y so you see the code of the function is exactly the same as
we had here so we just we we kept the same code we just changed the
type of this function and we made it very general so this is what
I would call a function with full appellate parametric type in other
words this function doesn't have any type in its signature that is
not a type of parameter it doesn't use any specific types like doubles
integers boolean so it only uses type parameters and no information
about what these types are so this is a fully parametric function
by my definition now notice that the first function the hypotenuse
the hypotenuse cannot be generalized in this way the first function
is very specific in what it does and so this kind of calculation makes
no sense for instance for string arguments or boolean arguments or
even integer arguments because you cannot very meaningfully extract
square roots from integers I mean you can do it in a very rough approximation
so we cannot easily make this function work let's say for string and
boolean instead of double and double but this function we can so so
this the first function cannot be very usefully generalized it's too
specific in what it does but the second function can can be generalized
so there's one little quirk a detail or about Scala notice we could
previously write this as a vowel so this we could say Deaf's walk
or look it's a Val swap because this is just a function it is defined
by a function expression on the right hand side so we could date we
could say Val hypotenuse or Val swap but here we cannot say well we
must say def this is so because Scala does not actually support Val's
with type parameters this is a limitation of the language in practice
this is not very serious you just used if there are other ways of
going over this limitation which is to use type parametric classes
or traits I'm not going to talk about this right now but later we
will see classes and traits with type parameters inside these classes
and traits you can define the Val's and then these vowels could have
types involving two type parameters but you cannot say for example
Val swap of X Y that just doesn't work like this so in this tutorial
we're just going to use diff but think about this as a vowel or as
a value it's an important it's important that we think about functions
as values and so it's emphasized that this is a value because it's
although it's a def there are no arguments so it's basically like
if L it has a type so it's like Val blah colon blah equals expression
so we are going to think about these as values with this type equal
to this expression just like Val X colon int equals 5 def eid : this
equals this so we think about functions as values here are more examples
it is a function that takes X of type T and returns the same X not
a very interesting function perhaps but it's quite useful in certain
in certain cases Const is another interesting function has two type
parameters C and X and the type of this function I put this in parenthesis
here just for clarity like this and like this but in actual code you
don't need these parenthesis if you don't like them just to make clear
visually that this is the type of this and these are the type parameters
so what is this is a function that takes a value of type C and returns
a function the tape's takes a value of type X and returns a value
of type C how does it work well it takes a C and then it takes an
x and then it returns that C so the X is ignored that's how it works
okay well this function may seem to be very strange it takes an argument
it ignores that and returns the first argument but this is a actually
also called an interesting and useful function a final example on
this slide compose is a is a function that takes two functions as
arguments and returns a function which is their composition we have
seen this in a previous tutorial perhaps but let me write a full general
form of it so it has three type parameters X Y Z the first parameter
sorry the first argument of composed is a function that goes from
X to Y and the second argument is a function that goes from Y to Z
and the result is a functional goes from X to Z and then the code
of this function is very simple it's this expression very short takes
argument X of type X and returns G of f of X so f of X will have type
Y G of f of X will have typed Z I would like to emphasize that the
Scala compiler will not allow you to use these types incorrectly these
type parameters are going to be checked by the Scala compiler let
me show how that works so I define this compose function and I suppose
I wanted to well suppose I made a mistake here I did F of G instead
of G of F there is a mistake and IntelliJ shows me that in red actually
this also will not compile it's not just IntelliJ this will not compile
expression of type Y does not conform to expected type Z so G of X
is a function that takes Y and returns Z I did not say what type X
was but actually it doesn't matter so G always returns Z f always
returns Y and I said so whatever the argument of F is it will return
Y and but I said that the function returns Z so it says expression
of type Y doesn't conform to expected type Z so Z was specified as
the return type side expected Z but I did not give you an expression
of type Z as a result let's correct this error and test that this
works to test we will define two functions let's say a function going
from integer to boolean and the function going from boolean to string
will compose them and get a function going from integer to string
so let's say first function is going to be from int bool it's just
going to be X going to whether X is even so X remainder in division
by 2 is 0 and B will be a function from boolean to string it's just
going to be printing the boolean let's say I forgot the syntax B is
going to be to string alright so now let's say C equals compose a
B now what type is C let's press control shift P which is IntelliJ
and it tells me it's from integer string so that's fine let me compute
C of 12 so the string true and that's incomplete C of 11 that's so
these are my working examples now suppose I made a mistake here I
said compose be a what would happens right away I've got red here
but actually we see here an interesting thing intelligent it does
not show you the error correctly if I press this button I get a different
error it does not show me here red but actually it's a type mismatch
so this code actually won't compile on line six line eight and nine
also have a type mismatch but actually line six doesn't compile because
the functions have the wrong types B and a must be composed in this
order similarly if I wanted if I made a mistake here and I said G
of X not just G of f of X then if I compile this there's a type mismatch
so just to warn you that IntelliJ doesn't always show red in your
in your editing window but actually the Scala compiler will not allow
you to compile code that is having a type mismatch of any kind so
the correct code would be G of f of X denied then then you get the
result here so actually this button will then run and this was just
a digression to show you that the Scala compiler would strictly check
all the types that you specify and will prevent many errors happening
by oversight if you input functions and in correct order when you
compose them or something like this so at first it could appear that
these functions like this taking extra and returning X again or this
are pretty useless this is not so we will see later so that these
functions are building blocks in several in several ways of interesting
programs but for now these are just examples of function types let
us go through some more worked examples all these examples are implemented
as tests in the test code here so here are the hypotenuse examples
the swap example there are some more examples here with a Const using
the Const and using compose you can use compose in line like this
so you do in line two functions like that function expressions and
the result is a function of this type and one interesting thing is
that you don't necessarily need to specify the type if I delete this
type here and I press control shift P if the intellij then shows me
the type so what's happening here IntelliJ and the Scala compiler
both they will infer the type from the code that you write so you
write some code do specify some types like this and then clearly this
is a string since you're doing two strings so if you if you do control
shift P it knows is that to strain in the function it turns the string
so compose has type parameters the Scala compiler will find the values
of these type parameters that fit in this case so for example this
function obviously you're returning a boolean here because this expression
is a boolean expression so this first argument has type integer -
boolean this is inferred from the code that you wrote you don't have
to say explicitly that this is integer - boolean you just say this
is integer then it knows this must be boolean so it infers that this
expression is integer to boolean and the compose function has the
first argument text Y therefore X the type parameter the capital X
must be equal to the type integer as a type variable capital X and
the capital y must be equal to the type boolean similarly the second
argument of compose is this expression and it has type of a function
from boolean to string which follows from the code so the skeleton
Pilar then looks at your definition of the function compose and finds
that G has typed Y to Z so obviously the type variable y must be equal
to the type boolean and the type variable Z must be equal to the type
string in the previous argument Y was also equal to boolean so that
fits Y is the same one as this it has value blue in Z has value string
X has value int if any of these types did not match like for example
the first argument gave you from integer to integer and a second from
boolean to strengthen this y must be integer but this y must be boolean
that does not match this woman this must be the value of the type
variable the compiler will then give you an error in this way the
compiler in first the type that this is integer to string and so f
if you do control shift P it tells you it's integer to string another
function in IntelliJ is option enter which says add type annotation
it automatically inserts the type annotation that it inferred from
your code this is also a very useful function to check that the compiler
understands your code in the same way as you think your code is supposed
to be working in terms of types so this will give you a check that
the function f has the type you think it should help but you don't
once you check that maybe you don't need to write this type a Scala
compiler does not infer every type it in first most but not all types
in many cases you have to specify some types for example type of argument
in a function often needs to be specified alternatively you could
specify type arguments of the compose function let me show you how
that works let's call as f f1 now I'm going to delete the type annotation
here and now once I delete them things become red because it doesn't
know that X is integer you see compose has a fully parametric type
so the first function is X 2y it has no clue that X you think must
be integer but if I write here I can with this syntax I can say it's
integer so I I have to specify these three arguments integer boolean
string once I do that the Scala compiler will check that everything
I wrote here fits what I just said so what I said is that compose
in this line is called with type arguments int boolean string so F
must be in two boolean and then this fits X is in let me do ctrl shift
key so now Scala knows that this X is integer this Y is boolean so
it knows that once it knows that I can write it in the shorter format
which I showed you in a previous tutorial this format of the functions
typically if you have a function that looks like X going to X something
you just replace this by underscore this is a special syntax that
Scala provides to write functions in a shorter way and now since all
types are specified you don't need to specify in the don't need to
write type annotations in this long format so this is a shorter format
and then F 1 of 23 should equal false F 1 of 22 and equal true let's
run this test and see that it passes and let it compiles of course
even though we don't see in your head there might be errors but the
Scala compiler will catch everything is great good so let's look at
our first worked example I define the functions constant Eid as in
the previous slide what is coin stop it and what is the type of that
value cost of it so how do we find out well this looks like an idle
question but it's a useful exercise so let's reason about constant
it constant it are defined here so what would be cost of it const
has the first argument of type C so when we say Const of it it means
that C is the type of it now the type of it is T two T is a function
it means that C must be equal to the function T two T let's write
this as a comment constant get so C Const has two type parameters
and it has one type to another so let's try to put these parameters
in so it can have any parameter T constant must have the trend or
C of X so the first argument C must be equal to the type of Eid of
T which is C 2 T so C must be T 2 T 2 good then when we apply constitu
its first argument of type C it returns a function of type X to see
where X is anything so it means that the result of applying constant
to its argument is something of type X 2 C now C we know is T 2 T
so the result is this I don't write parentheses because by convention
these parentheses are unnecessary on the right side of the error so
this is the type of Const of it it has still two unknown type parameters
X and T and the type is X 2 T 2 T so that is our reasoning let me
go to the actual code that I prepared so Const of it so that let me
put the comment here and let's see what happens if we just start writing
I will delete this let's see what happens if we write code like this
Const of it what would be the type of this so let me press option
enter and add the type and notation it's a very curious type it inferred
any - nothing - nothing so it looks like X 2 T 2 T except it didn't
introduce type parameters so Scala does not infer the most general
type which we inferred well scholars compiler is limited in this way
its type inference is limited so this is certainly not going to work
nothing isn't very useless type in this case we actually don't need
one test type but that's what Scala in first because it has no information
now the second attempt we made I made was this I put a type parameter
here and then I say it has that parameter here so what the result
of this is any Tootsie Tootsie so it instead of it did not infer type
tramp your ex like I said Scala cannot really do this it will infer
nothing or any it cannot infer that there is a type parameter okay
so that is a little better any is really any type it's a type that
fits anything is also not very useful it doesn't check any any correct
types so that's not great let's try this so you see we had this idea
that this works when C is T - t let's put explicitly name this - T
so this is what we wrote okay so let's see if this makes intelligent
infer the type es so let me rename this to X and then we have exactly
what I have here in the comment so another way of achieving the same
result is that I write the type here and then I don't write any type
parameters on the right hand side that works too so here what happens
is that I specify the type and IntelliJ or Scala compiler rather will
fit the types so it knows this is T 2 T this is C 2 X 2 C and C will
be equal to T 2 T so until now I've only looked at types I don't actually
talk about didn't talk talk about what this value does this constant
it well it's a function it takes an argument of any type and returns
this which is the identity function let's check that it works actually
so let me say C equals example 0 1 B of let's say some number okay
so something is wrong here yes so I define this method twice let's
go see okay now the result of calling this function which is constant
of it on the number is a function which has no type nothing to nothing
so I need to specify type parameters here so let me say this is int
and this is moving then C will have type boolean tubulin so in this
way I get an identity function of type boolean to boolean now this
is more flexible I can define a function which is of type C you know
where I I say this is of type T and then I can have a function of
type t - t right so this is an identity so I can say for example C
of ABC that should be call ABC let me run this test so now C is identity
function now this identity function is fully parametric because I
put all the types in here so green to run the test I press control-shift
are by the way very useful thing you are in in a test code then you
just press control shift or it only runs one test all right so these
are the types and this is the value it's an identity function of some
arbitrary type second example define a function twice that takes the
function f as its argument returns a new function it applies F twice
for example twice of this must return a new function which is equivalent
to that is equivalent to adding six so adding three twice so let's
implement this function here is the implementation basically what
we remove the type now it's red so the function twice takes a function
f as an argument and function f has some type which type should it
have what's reason about this so this is a solution let's find out
how we reason about the problem to derive the solution so let me write
this as a comment I want to do a a function twice must take some F
as an argument so let's say this is a 2b and it must return some C
so what does it do what is this C well we return required to return
the function okay so that's this function should be applying F twice
to some argument so f of f of X now what is X well obviously X must
be the argument so we need to return this code so this code seems
to be clear we need to return a function that takes X and applies
F twice 2x so what remains is to determine the types here right now
I just said ABC because I don't know yet what they are the code is
already fixed this is what we want now what is the type of X well
it can be any type so let's say that the type of X is deep for example
well then we need to apply F to X and F has type A to B the only way
that can work is when X has type a then f of X will have type B okay
so D actually needs to be a now the result is that f of X has type
B now we apply F to that but F means type a has argument this can
only work when a is the same as B all right now F of a is a so the
result of this whole thing must be a so C is not actually a different
type it's also a alright so now we need an a as a parameter just one
parameter in this way we derived the code and actually oh actually
this is a to AML today sorry about that because we are returning a
function right now you see there are two ways of defining this they
are different only by syntax one is that we say define twice and then
we specify argument in parenthesis then we specify the return type
and when we write the code second way is just different syntax is
to say this is a kind of a value of this type so the first argument
this type which is the same as this it takes this argument and returns
this and here we could write parentheses again like that but this
is not necessary by the syntax convention the first pair of parentheses
is required it is necessary okay and what is the value of twice V
it is a function that takes this so it takes F of this type and return
the function that takes X of type T and returns this so these are
two I I not identical but equivalent ways of defining this function
which one is more convenient well the first one is less code to read
plus code to write and however the second one emphasizes the value
nature of functions so sometimes for clarity you would do this but
most of the time you undo this because it's shorter and also easier
to read as a function its argument its value and its code that takes
argument and so on and you could even declare this function with the
syntax with two pairs of parenthesis so let me show you how that works
so I declare the first argument I declare the second argument which
is of type T in the second pair of parenthesis so this is this X which
was here which must be of type T so I declare it here and then the
result is of type T and then I type this as the code now let's call
it twice zero so the tests don't stop working now all of these three
definitions declare exactly the same function which has effectively
two pairs of arguments at first sorry two sets of arguments the first
set is this the second set is that so in Scala you can have functions
that have any number of sets of arguments in this case each set has
just one argument so now notice this is not the same as twice f T
goes to T X T that's not the same that function is not returning a
function it just returns a value and it has two arguments you cannot
call this function with one argument at a time with this syntax with
two sets of arguments you can call as function with the first set
of argument and get a function back this is the curried syntax this
is not curried so this is not what we want if we wanted a function
that returns a function the only way to do that is to use this syntax
or this syntax or list syntax Scala gives you all these possibilities
most people would prefer this because this is the smallest code to
write but I just wanted to make it clear this is the same these all
three are the same thing and this is not the same all right so let
me delete that and now let's test so let's call twice on this function
right so we apply this function twice to ten we get 16 let's see how
we can specify types while Scala Allah requires you to specify types
somewhere so you can specify it here it will then infer that the type
arguments T on twice must be int or you can specify the type argument
T on twice it will then infer that this X must be int if you don't
do that things won't compile because it won't you know that T must
be int both are the same so another way of writing it is to write
both arguments at once so twice of this applied to 10 same answer
is 16 now I'm just testing twice V which I put over there let me just
doing this it's unnecessary to define it twice just to see that it
gives you the same results 16 another syntax for the same as this
remember in Scala when you have a function that looks like X goes
to X something then you can replace this entire thing next to X by
underscore so that is just making it shorter like this is a function
that adds 3 to its arguments so this underscore means argument this
is just fancy syntax a lot of people like it um so here the tests
basically show you all the different ways in which you can write the
same thing here's another interesting syntax that scholar gives you
again it's exactly the same thing it's twice V or applying which is
syntactic they are a variant of twice applied to the same argument
the function that takes X and goes to X plus 3 however now I put early
braces around this function which is possible acceptable and note
I don't need parentheses around the type anymore now this is a syntactic
convenience in Scala in most cases you want parentheses round round
parentheses for very simple functions and you want curly braces for
complicated functions functions in a curly brace can be multi-line
they can introduce new web new names like vowels you can do new depths
you can do whatever you want you cannot do this if you write that
in ordinary round parentheses that just won't work you are not allowed
to have multi-line expressions with vowels and deaths inside this
kind of parentheses Scala sandbox requires you in this case the right
curly brace once you write the curly brace we can't have multi-line
code with all kinds of stuff in it so we can here you can define new
things whatever you want it can be very complicated so for this reason
the Scala convention is that if your function is complicated use curly
braces if your function is extremely simple use round parentheses
let me undo all this and show you so this is a very simple function
that adds 3 to the argument for such things Scala provides a short
syntax and this syntax is equivalent but normally used for complicated
functions alright so much about cindex so Scala gives you all these
syntactic possibilities no matter how you define if and the syntax
works I just have all these tests that check that this all combinations
of syntax always worked in other programming languages there are fewer
versions of syntax in Scala der quite a lot don't think that I don't
think this is a drawback you can choose the one you like most the
most important thing however is the type that always needs to be given
when you have functions with fully parametric types Scala don't doesn't
usually know what X is in an anonymous function expression such as
this the next question is to derive what twice of twice is what does
it do well it's a function right twice returns a function so when
you call twice on something that returns a function so what does that
function do so if we reason about this twice applies its argument
two times to something so twice of twice will apply twice two times
to something so it will be like twice of twice of something now twice
of something is that something applied two times so twice of twice
of something means that itself applied to x so now I'm getting already
confused so let's write it down actually so these are the various
definitions of twice twice but before we start writing code let's
reason about what it should be so mathematically twice of twice is
a action that should be applied to some F so what is that first we
compute this so twice of something is going to be applying this two
times to that so this is the same as twice of twice all right now
twice of F is a function so this entire thing needs to be applied
to some X so this is equal to twice of X going to F of f of X now
this twice is going to apply this function two times to some argument
in other words this is some argument let's say X and it's going to
be applying this function two times to this argument so this is F
of F of F of f of X so that applies the function f for x to the argument
well maybe sense so these are the tests if you apply this twice twice
to a function like that it it applies this four times so it adds 12
and so the result of adding to a 12 to 10 is 22 now the interesting
thing is to apply it three times I leave that to you as an exercise
apply twice of twice of twice so I'm going to leave that and let's
concentrate now on types so we figured out what this function does
it just applies four times how to write it well if we just write twice
of twice things don't work the way we don't we see that as it doesn't
work if we do ctrl shift P that's the type it inferred and that's
no good it's nothing nothing nothing nothing so that's no good that's
no good now that's better so the type why is that type what is that
so let's reason about the type so twice has the type which we derived
over there the easiest way of thinking about the type of a function
is to write it like this as a kind of a value with parameter with
no explicit arguments and just as a value but of this type so this
is the type of twice now if I want to apply it twice to itself what
does it mean well this is that argument so the first argument is actually
t - t and in here it's again the same type so this cannot be the same
T in here and in here because the first argument of twice is T - t
and twice is this so the first realization is that these two instances
of twice cannot have the same type parameter let's call this type
parameter a let's call this type parameter B or or what's called as
T just to make it a little easier all right so twice T is is of this
type now what is a what can it be well the first argument of twice
a is of type so let me repeat this comment and say what is twice a
twice a has this type if it's a function whose first argument is this
the second argument is that so the first argument of twice a is this
so it has this type I remind you by convention this is how we interpret
this syntax with these parentheses so clearly this can be a to a only
when a is actually T to T and then a to a is this alright so now we
figured out that the type parameter a here must be actually t to t
okay so what is the result of : twice with argument twice well we
put this in here so this is twice of a right so we put this argument
here the result is this type so it's a to a a is this so it's again
that so a to a is equal to this now I'm this is not Scala code in
their comments that I wrote while I was thinking about the type of
this expression in Scala code you don't write this you cannot do that
so you can write that but you cannot write what I need so these are
just my notes for myself {[}Music{]} so now intelligent agrees there
are two ways to make or even three ways to make IntelliJ in further
types if I just write this IntelliJ doesn't infer the type if I write
this it doesn't improve the type which I verify here because this
example doesn't compile this example doesn't compile so three ways
that it works in terms of syntax first well obviously we have to specify
a type somewhere first we specify this parameter so we specify a remember
this was our a a second is we specified T the third is we specified
the entire thing on the left now maybe this is the best way because
it serves at the same time as documentation and as code if somebody
reads this code later it will be very helpful if they don't have to
go through this consideration again this is a calculation really what
we just did here is a calculation and I would say it's very nice that
we determine things by calculation in our programs we're not guessing
anything but once this calculation is done it is nice to document
its results so that other people don't have to repeat this calculation
again so however all these three things define exactly the same function
which I verify here in the tests calling this function again requires
you since this function is fully generic it doesn't have any non generic
parameters non parametric types generic and parametric is synonymous
it's I prefer to say parametric but in some programming languages
people use the word generic in terms of generic type nice parametric
type or type parameters so when you use this function you supply anonymous
function as the argument Scala needs to know the type of that function
either that type is inferred because you specify this where the type
is inferred because you specify this you can specify both that's not
an error it's just not necessary so that's how it works so now we
found out the type of twice twice let's go on the next example is
that we need to take a function with two arguments fix the value of
the first argument and return the function of the remaining one argument
we define this operation as a function with fully parametric types
and here is what we were supposed to do so we are supposed to have
a function first Arg with three type parameters the first argument
is a function from two arguments of types x and y to a z value x z
a second argument of the function is a value of type x and the result
must be a function from Y to Z so we fix the value of the first argument
of this F we call F on the x0 and the supplied Y so that results in
a new function that takes Y as an argument and returns Z so that is
the most general type that we could imagine for this kind of function
the code for this function is extremely simple just for readability
sometimes I like to put the code of the function on the next line
and also sometimes I like to put this into curly brackets in the curly
braces you saw that it's written or readable this is the code of the
function and this is the type of this return value but it's just syntax
it's exactly the same thing if the function is very simple like this
function for example it's okay to have it in all in one line all right
so this is it this is the reason this is the solution so how do we
reason about this problem how do we derive this solution so let me
see if I didn't know how to write this what would I do well I would
write the first part so we're supposed to have two arguments first
is a function of some type x and y sub two arguments function and
this is a general type of any function so that has two arguments then
the general type is XY going to Z I remind you this is not a to problem
XY Scala has a distinction if I wanted the tuple of XY I would have
done this and that means that the function f takes a single argument
that is a tuple of F of two parts having types x and y but in my problem
statement i was not supposed to do that i was supposed to define a
first arc whose argument is a function of two organs so that's a general
type and then the value x0 has type X and I'm then I'll return what
you can I return well there is not much I can return really all the
types are fully parametric so I cannot return Y for example because
I don't have any values of type Y I could return X but that's not
what I mean required to do and require to call F put in X 0 in it
so I don't know what that is but I need to call f of X 0 and y and
Y should be an argument of this new function so what is the type of
Y well obviously it must have type capital y there's no by the way
so this must be a capital y going to Z because F returns Z so in this
way I'm forced to have this as a return type and then I need to introduce
X Y Z as type variable since I'm using them here so in this way I
derive first well the code of the function is kind of clear from the
problem statement but then I derived the most general type signature
for this function a fully parameterised function type it does not
have any specific types like integers or emporia string or anything
all types are parameters alright so that is how I reason about this
solution and then I just check that this works so for example I have
a function try one that prints this and then I fix the argument of
the first argument the integer argument of try 1 to be 123 and then
the result is that so that works and also this syntax works so I can
call first Arg with two sets of arguments the first set of arguments
having two arguments the second set of arguments having a single argument
so now but by now I hope you can understand how this works and how
the syntax works alternative way of writing this function would be
to have a value like let's just call it first Arg 1 to make things
don't break a value like syntax where this doesn't have parameters
but it equals an expression which is a function so f okay so first
I say the type which is why point to Z and X so this is the first
set of arguments when I have a second set of arguments which is just
Y and then Z and then I say it's f going to Y going to I'm sorry F
comma X zero going to Y going to type of X near Y alright so this
is an alternative syntax for exactly the same function it has two
sets of arguments which means that here's the first set of arguments
arrow second set of arguments arrow result type this is a typical
thing for curried functions with multiple arrows and then the first
set of arguments actually has two arguments of these types the first
type is less the second type is this this is a bit harder to read
than that so I would not prefer to write things this way but this
is equivalent so it's important that you understand why this is equivalent
and how this works I can exactly the same syntax can be used for the
second function let's run this test to make sure this runs so this
is a solution of this example let me just wait until tests finish
component and then we'll go on moment the next example is we need
to implement a function that applies given function f repeatedly to
an initial value x0 until a given condition function returns true
so a hint is that this would be a signature so we'll have parameter
X type parameter X and then three arguments a function from X to X
initial value of type X and the condition from X to boolean now notice
we have this boolean so this is not a fully parametric function it
has a specific logic which is tied to this boolean type so that's
fine boolean is a very special type and that's probably okay so this
function is still widely usable for many different types X how would
we implement this function so there are two ways of implementing it
if you have gone through the previous tutorial we have seen how that
works the first way is using iterator which is a library function
second is to write explicit recursion let me go through the iterator
first I use the iterated function which is a standard library parameterize
by type X which is given to me here and iterate has two sets of arguments
it's a curried function the first argument is start which is of type
T which I called X here you see X is a type variable so this is like
an argument or a function as well so the type argument so these are
value arguments and this is a type argument and like in any function
argument I can rename this to anything I want I can rename this to
T I can reveal this to LA I can rename this to anything so by convention
these are single capital letters but it doesn't have to be this way
so iterate is the first argument which is start second argument is
a function from T to T so we have exactly the same arguments here
so I give X 0 and f as the arguments so what does iterate do it creates
an iterator that takes X 0 as the first value and then applies f repeatedly
getting the next value and again getting the next value and so on
that's exactly what we want however this will generate an infinite
iterators will never stop so what we need is to stop when this condition
is true so we filter which means we skip all the elements in the iterator
sequence for which the condition is not true after the filter I still
have an iterator of T but now all elements in this iterator sequence
are such that the condition returns true we actually only need the
first one so let's take one element the result of this is still an
iterator you see to see that I press command and I hover my mouse
over this symbol and then I see what type it would definition it has
an type in everything so after this I have a still an iterator of
T with just one element in it so now I can convert this to sequence
the result of that would be sequence of T having just one element
now I take a head of this sequence which is just one value it's safe
to take head because well either this condition is never true or it's
true sometimes if it's true sometimes we'll get that value this sequence
will be non-empty and head can be done on an on two secrets if this
condition is never true and this iterator will iterate forever the
filter will be never true so this entire thing will never return we
cannot do very much about this while we could in principle first specify
some max number of elements take that so we could do this to be safe
take 1 million elements and but that would is not what we were told
to do in the problem setting so in some real situation we should think
about limiting the number of iterations of course let's just forget
about this for now ok so this is a reasonably simple implementation
the second implementation uses tail recursion so that is a direct
translation of mathematical induction which is if the condition is
true already on the initial element that's the base case of the induction
and then we return that initial element otherwise we apply one function
f1 x and a function f to x0 and call the same function again so we
do the inductive step the the value is the same procedure applied
to the next value after the initial value now this is a little more
difficult to reason about perhaps than this and also what if I wanted
to limit the number of iterations then I have to complicate this code
significantly so tail recursion is not very nice to maintain its or
rather not Taylor just recursion at all recursive functions generally
take more work to write and more work to maintain that is to to make
the curve to make changes if you want to do other things or make things
different I'll do things differently it's more difficult to make changes
to recursive function then it is - this kind of code which is on the
surface it's not recursive was just calling library functions and
working directly on a sequence so iterator produces a sequence and
we just call functions on a sequence - nothing is recursive here recursion
is hidden somewhere in the library and it's safe so it's very easy
to change this code to do whatever we want so I would prefer this
implementation in production code it's easier to maintain how do I
test this well I test this using the procedure to compute approximate
square roots with the saturation so this is known from numerical methods
i iterate a square root sequence so given X and if given Y which is
an approximation to the square root then this formula gives a better
approximation to the square root of x and I iterate that with initial
value equal to 1/2 of X just randomly kind of chosen as Peralta strongly
not equal to x over 2 min most cases but anyway that's okay as an
initial guess and then the condition the condition is that Y times
y minus x is less than the given precision so I need to specify the
precision just write it like this with all s spaced out how do I test
this so I say precision is 10 to the minus 8 and then let's compute
square root of 25 that should be equal to 5 plus - precision so this
is test library that gives me their syntax and this test passes final
- worked examples are to infer types in a function so this is something
we have done when we were reasoning about twice and let's just do
this a little more to get practice to understand how types work in
functions alright so what is this let's reason about this I already
wrote of course the solution before but let's pretend we don't see
that so how do we reason about it so here's what we need right this
code is given what is the type of this value well obviously it's a
function what is the type of the argument of this function well this
is some kind of a so f has type a and then the code of the function
the expression is f of 2 well that can only work if F is itself a
function and this function has an argument that is an integer so a
must be actually int going to some B that's the only way that this
could make any sense all right now what is the result of applying
F to 2 the result is of type D so this will return B so what is the
type of this entire expression it's a function that takes F this and
returns this which is of type B so so it returns B and it takes this
does it type okay so it's let me write it a little more formatted
so this is the actual type now it has an unknown type B which is not
fixed by the code it could be any type so therefore we need this B
as a type parameter here this is how we derive the type now I wrote
this solution exactly the same code except I used T instead of B so
as we know this is a type variable so it can be renamed and we know
in mathematics our function arguments can be renamed type arguments
are the kind of function arguments as well although there are different
different sort so it's the same if we write B or T or any other letter
so in this way we infer the missing types so let's find out if this
works so how do we test well we need to supply a function of type
into T so I'd say T is boolean so let's supply this function which
is defined as this again this is a function I used before which checks
that X is even X going to this expression which is true only when
X is even so then P of F should equal true because 2 is even so the
function P applies def to a fixed argument which is 2 fixed value
which is 2 so then P of F is true because F in our example is checking
whether its argument is even another question here is whether P of
P works and I have a test here that says it does not compile so why
does P of P not compile let's reason about the type of P of P so P
of P let's write out the type arguments well it could be different
type arguments correct we don't know that so it could be different
type arguments since they're not written in this expression could
we find any type arguments a and B that would make this expression
well typed let's reason about this let's replace the expression P
of B with its type just for our own notes this is not going to be
Scala code it's going to be our reasoning which is some mathematical
like notation so we replace this with the type so the type of P of
B is this int going to be going to be and P of a is int going to a
going to a so this type is a function which is applied to this type
now the only way this can work is when the type of the argument of
this function is the same as the type of the expression that we are
applying it to so can it be that int int going to a is the same as
int going to be to be so the left hand side is something going to
something a function type mapping into a lab the right hand side is
again against something going to something so the only way this can
work is if a is the same as be yeah but now it doesn't worry because
int is not the same type as int going to 84 no possible type a we
could have int equal to this because this is a function type and this
is not a function type so there are no more type variables there is
nothing we can do to make this match so for this reason there are
no possibilities to find types or type parameters in this expression
so that the types match the compiler finds this and gives us an error
so if I write something like this I will have read type mismatch cannot
resolve reference be with such signature so it tries to find values
of type parameters that would match but it can't and so it gives me
read right away now this read is different from the read I would get
if I emitted type parameters somewhere because actually this can never
work whatever type parameters you put in it just will never match
because what we just saw well we just saw int cannot be equal to int
to me so that's why it will not compile the final worked example is
this inferred types in this code and ask questions about key of Q
and Q of Q of Q so this is kind of a puzzle a bit so let's go through
this just as an illustration of type directed reason so Q is given
as sowhat's again start reasoning q is given as f going to G going
to G of F f going to G going to G of F so this is the code that we
are given we should be able to put types on it so let me put this
into braces and let's put types so f is some type a maybe G is of
some type B I don't know what types so right now I just put some arbitrary
type various type of variables okay now G of F is in the code so it
means G must be a function such that its first argument is of type
a so this B must be of type a going to some C so let's call let's
put it more explicitly it's not helping to so okay so we assumed F
is of type a therefore G must be of type a to something so let's call
that something C without loss of generality the result is of type
C therefore so this whole thing is F type a going to a to C going
to C so the first argument is a the second is well it's a curried
function so it's the syntax with something arrow something arrow something
where the parentheses around the last pair are assumed so I'm not
going to write down but these parentheses are assumed and therefore
the type of Q must be this and that's the most general type I have
not assumed anything about any type so for example G is a function
it must be a function because it's applied to F but the result of
genius any types it could be any complicated type could be itself
a function I don't care so Q must have two type parameters a and C
therefore and that is the solution so this is exactly what I wrote
here except I used letters F and T instead of a and C and I have two
versions of syntax first is when I specify the types on the right
hand side the second if I specify the types on the left hand side
other than that it's exactly the same thing so Scala would not work
if I don't specify types it cannot infer type parameters so I have
to specify type parameters and I have to specify either this or this
if I specify this it will infer the right type if I specify that it
will infer types of F and type on G I prefer the second form because
it shows what type this value is so this computation doesn't have
to be repeated by people who will look at this code later and actually
I would even prefer q1 let's say which has two sets of our means one
is f-type F 1 and G of type F going to T and that's of type T so I
would actually prefer to write code like this because that is easier
to read I clearly show there two groups of arguments two sets of arguments
we are not actually mathematical sets leader sequences of arguments
as arguments are ordering here so first and second and I clearly separate
languages easier to read the types are given the result type in women
and the code is shortened so I don't have this longer expression which
is somewhat more difficult to read and I don't have this expression
however it's nice to understand that this is actually the function
value unworthiness all right so when I use this skill I need to specify
type parameters fmt otherwise it doesn't know what to do and that
means don't work well as we have seen before so let me just emphasize
two things first these three syntactic forms define exactly equivalent
code there's no difference between the code except the syntax second
thing I'd like to emphasize is that this function exists once in the
code and it can be used many times specifying different type parameters
so I could use it in this place in the code with these type parameters
at another place in the code with different type parameters I don't
have to repeat the definition of this function and it's not duplicate
it's called with different types so really type parameters are like
arguments where you call function call this function many times of
different audience you can call this function many tests also with
different type turnovers and different arguments here I show different
ways of using it so a has the type insta boolean - boolean and then
I use it on a function which is into the boolean I get boolean now
if I don't specify types then this doesn't work I have to specify
types here it works because I specified that this is returning boolean
so it could get both type parameters so you see sometimes you have
to specify types at other times if you give both arguments at once
then this is fine you don't have to specify type it's never an error
to specify type parameters just makes your code longer so it knows
this is an int because this is an int and then that is a boolean so
it knows this is a boy so in this case you didn't have to specify
this in here but you could and if you do the error message would probably
be easier to understand so I put here for example 10.0 type mismatch
expected int actual double but why does it expect int it's because
of this int that I wrote if I didn't write this this error wouldn't
be noticed so it's never a mistake to write types it's safer it guards
against mistakes in the code earlier but it's just more typing so
it's not always necessary alright now what is Q of Q well that's an
interesting same Q of Q and you see things are getting very long very
easy but I will not go through a key of P of Q perhaps because it's
very similar so let us go through Q of Q so that's reason about it
so what is Q of Q well first of all we need to put type parameters
Q has to type parameters a and B and this is C only right because
the two instances of Q could have completely different type parameters
the Scala compiler will try to find combinations of type parameters
that work together it may fail or it may succeed let's see if this
succeeds just as I did before I'm going to replace this expression
with its type just for the purposes of reasoning so what is the type
the easiest way to see the type is to look at this syntax so this
is the type F T is f f - t t so so this is going to be C going to
see me going through D and this is I'm just going to copy this in
here and replace CMD with mV so now I'm going to try to fit the types
together this function has the first argument which is of type a and
it's applied to this expression of this type so the only way this
can work if a equals this that's already something but once that is
so once a is equal to this then there is no more constraint so this
argument has been substituted it has the right type B is an is not
fixed so B can be anything so actually we have three arbitrate type
parameters now B CMD now I can rename them but I have three type parameter
so Q Q which is Q of Q has this type which is applying Q to this expression
and the result is this type now a is equal to that so I thought it
I'm dating just recently didn't it so let me copy this and instead
of a let me copy this now I need parentheses around it let me do it
slower put parentheses and then copy it in there now there are no
mistakes so this is actually the type of Q of Q it's a complicated
type it's a higher order function a function that takes as its first
argument this function that takes as its first argument this function
that takes as its first argument this of type C and the second argument
this so the complicated matching of types that we'll all be performed
automatically by the Scala compiler so the way that scholar can do
it for you is first you write this so we have just figured out it
a must be equal to this so let's put that in the type parameters and
then put all the type parameters explicitly and then say option enter
and add type annotations so the Scala compiler will infer this type
correctly in the same way reasoning about Q of Q of Q gives you this
set of type parameters and so I'm not going to go through this but
you're encouraged to travel now something that doesn't work is this
now you see there's a difference between this and this here I take
Q of Q and I apply Q to the result here I take Q of Q and apply that
again this does not compile because Scala cannot infer the correct
type arguments for these two cues so I can put arbitrary a and B but
it's complicated so how do it reason about this it is not clear actually
at the beginning whether this expression can be typed so in programming
languages such as Oh camel and Haskell I'm pretty sure this would
be done by the compiler because their type systems are different and
these examples can be typed automatically but in Scala this is not
automatic so let's reason about it and this is also a good exercise
so what is this expression is the same as QQ applied to Q here we
have a type of QQ with three type argument so let's put it here and
this Q has two type arguments so let's take this expression so basically
our {[}Music{]} question is can we find ABC F and T so that this is
a well typed expression now expression is well typed when all functions
receive arguments of the correct types that's basically the definition
here we have this function which is applied to this argument of this
type so the only condition is whether this function QQ o of types
ABC has the first argument which can be matched with this type let's
check what is the type of QQ of ABC it is this this is a function
whose first argument is yes parentheses so let's copy that so this
must equal this if that can be matched with some choice of ABC F T
then we're done there are no other problems how can this be matched
so again the only way that this can be matched is when the left is
a function of some X to some Y and the right also is a function from
some X to some one the same x and y now the left is a function from
this to see the right is a function from F to this remember there's
implicitly there are these parentheses here therefore f must equal
this and C must equal this okay can we do this of course we can F
can be equal to this and once that is true C must be equal to that
so let's put parentheses here and I'll paste it in alright so now
we can put this instead of F in here so that's right code actually
some parameters C T whatever actually C will be equal to that so a
B and T will remain and then we get QQ of a B C of Q of F C right
now let's paste so f is equal to this and C is equal to that and now
everything is green so now we can do option enter here and it will
infer the type so the type of this expression is actually this which
is the same as the type of QQ up to changing C 2 T so we can rename
this to C and it will be exactly the same type so this is very interesting
we we have Q of Q of this type and Q of Q of Q is again of this type
so clearly we can continue doing this Q of Q of Q of Q of Q and it
will still have the same type up to some complicated substitutions
in the types now this example I admit is quite artificial but this
serves to show you how type reasoning works here are some exercises
for you and you can apply the typed reasoning as I just showed you
in the same way and I encourage you to do these exercises 
\end{comment}


\subsection{Curried functions}

Consider a function with type signature \lstinline!Int => (Int => Int)!.
This is a function that takes an integer and returns a \emph{function}
that again takes an integer and then returns an integer. So, we obtain
an integer result only after we apply the function to \emph{two} integer
values, one after another. This is, in a sense, equivalent to a function
having two arguments, except the application of the function needs
to be done in two steps, applying it to one argument at a time. Functions
of this sort are called \textbf{curried} functions\index{curried function}.

One way of defining such a function is
\begin{lstlisting}
def f0(x: Int): Int => Int = { y => x - y }
\end{lstlisting}

The function takes an integer argument \lstinline!x! and returns
the expression \lstinline!y => x - y!, which is a function of type
\lstinline!Int => Int!. So the type of \lstinline!f0! is written
as \lstinline!Int => (Int => Int)!.

To use \lstinline!f0!, we must apply it to an integer value. The
result is a value of function type:

\begin{lstlisting}
scala> val r = f0(20)
\end{lstlisting}
The value \lstinline!r! can be now applied to another integer argument:
\begin{lstlisting}
scala> r(4)
\end{lstlisting}

In Scala, \lstinline!Int => Int => Int! means the same as \lstinline!Int => (Int => Int)!,
and \lstinline!x => y => x - y! means the same as \lstinline!x => (y => x - y)!.
In other words, the function symbol \lstinline!=>! associates to
the right. Thus, the type signature of \lstinline!f0! may be equivalently
written as \lstinline!Int => Int => Int!.

An equivalent way of defining a function with the same type signature
is
\begin{lstlisting}
val f1: Int => Int => Int = x => y => x - y
\end{lstlisting}

Let us compare the function \lstinline!f1! with a function that takes
its two arguments at once. Such a function will have a different type
signature, for instance
\begin{lstlisting}
def f2(x: Int, y: Int): Int = x - y
\end{lstlisting}
has type signature \lstinline!(Int, Int) => Int!.

The syntax for calling the functions \lstinline!f1! and \lstinline!f2!
is different:
\begin{lstlisting}
scala> f1(20)(4)
scala> f2(20, 4)
\end{lstlisting}
The main difference between the usage of \lstinline!f1! and \lstinline!f2!
is that \lstinline!f2! must be applied \emph{at once} to both arguments,
while \lstinline!f1(20)! can be evaluated separately, \textendash{}
that is, applied only to its first argument, \lstinline!20!. The
result of \lstinline!f1(20)! is a \emph{function} that can be later
applied to another argument:
\begin{lstlisting}
scala> val r = f1(20)
scala> r(4)
\end{lstlisting}

Applying a curried function to some but not all of possible arguments
is called \textbf{\index{partial application}partial application}.

More generally, a curried function may have a type signature of the
form \lstinline!A => B => C => P => Z!, where \lstinline!A!, \lstinline!B!,
\lstinline!C!, ..., \lstinline!Z! are some types. I prefer to think
about this kind of type signature as having arguments of types \lstinline!A!,
\lstinline!B!, ..., \lstinline!P!, called the \textbf{\index{curried arguments}curried
arguments} of the function, and the ``final'' return value of type
\lstinline!Z!. The ``final'' return value of the function is returned
after supplying all arguments.

A function with this type signature is, in a sense, equivalent to
an \textbf{uncurried\index{uncurried function}} function with type
signature \lstinline!(A,B,C,...,P) => Z!. The uncurried function
takes all arguments at once. The equivalence of curried and uncurried
functions is not \emph{equality}  \textendash{}  these functions are
\emph{different}; but one of them can be easily reconstructed from
the other if necessary. One says that a curried function is \index{isomorphic}\textbf{isomorphic}
or \textbf{equivalent} to an uncurried function.

From the point of view of programming language theory, curried functions
are ``simpler'' because they always have a \emph{single} argument
(and may return a function that will consume further arguments). From
the point of view of programming practice, curried functions are sometimes
harder to read.

In the syntax used e.g.\ in OCaml and Haskell, a curried function
such as \lstinline!f2! is applied to its arguments as \lstinline!f2 20 4!.
This departs further from the mathematical tradition and requires
some getting used to. If the two arguments are more complicated than
just $20$ and $4$, the resulting expression may become significantly
harder to read, compared with the syntax where commas are used to
separate the arguments. (Consider, for example, the Haskell expression
\lstinline!f2 (g 10) (h 20) + 30!.) To improve readability of code,
programmers may prefer to first define short names for complicated
expressions and then use these names as curried arguments.

In Scala, the choice of whether to use curried or uncurried function
signatures is largely a matter of syntactic convenience. Most Scala
code tends to be written with uncurried functions, while curried functions
are used when they produce more easily readable code.

One of the syntactic features for curried functions in Scala is the
ability to give a curried argument using the curly brace syntax. Compare
the two definitions of the function \lstinline!summation! described
earlier:
\begin{lstlisting}
def summation1(a: Int, b: Int, g: Int => Int): Int =
  (a to b).map(g).sum

def summation2(a: Int, b: Int)(g: Int => Int): int =
  (a to b).map(g).sum
summation1(1, 10, x => x*x*x + 2*x)
summation2(1, 10){ x => x*x*x + 2*x }
\end{lstlisting}

The code that calls \lstinline!summation2! may be easier to read
because the curried argument is syntactically separated from the rest
of the code by curly braces. This is especially useful when the curried
argument is itself a function, since the Scala curly braces syntax
allows function bodies to contain their own local definitions (\lstinline!val!
or \lstinline!def!).

A feature of Scala is the ``dotless'' method syntax: for example,
\lstinline!xs map f! is equivalent to \lstinline!xs.map(f)!. The
``dotless'' syntax works only for infix methods, such as \lstinline!.map!,
defined on specific types such as \lstinline!Seq!. Do not confuse
Scala's ``dotless'' method syntax with the short function application
syntax such as \lstinline!fmap f xs!, used in Haskell and some other
languages.

\subsection{Calculations with nameless functions}

We now need to gain experience working with nameless functions.

In mathematics, functions are evaluated by substituting their argument
values into their body. Nameless functions are evaluated in the same
way. For example, applying the nameless function $x\Rightarrow x+10$
to an integer $2$, we substitute $2$ instead of $x$ in \textquotedblleft $x+10$\textquotedblright{}
and get \textquotedblleft $2+10$\textquotedblright , which we then
evaluate to $12$. The computation is written like this, 
\[
(x\Rightarrow x+10)(2)=2+10=12\quad.
\]
Nameless functions are \emph{values} and can be used as part of larger
expressions, just as any other values. For instance, nameless functions
can be arguments of other functions (nameless or not). Here is an
example of applying a nameless function $f\Rightarrow f(2)$ to a
nameless function $x\Rightarrow x+4$:
\[
\left(f\Rightarrow f(2)\right)(x\Rightarrow x+4)=(x\Rightarrow x+4)(2)=6\quad.
\]
In the nameless function $f\Rightarrow f(2)$, the argument $f$ has
to be itself a function, otherwise the expression $f(2)$ would make
no sense. In this example, $f$ must have type \lstinline!Int => Int!.

There are some standard conventions for reducing the number of parentheses
when writing expressions involving nameless functions, especially
curried functions:
\begin{itemize}
\item Function expressions group everything to the right:\\
 $x\Rightarrow y\Rightarrow z\Rightarrow e$ means the same as $x\Rightarrow\left(y\Rightarrow\left(z\Rightarrow e\right)\right)$.
\item Function applications group everything to the left:\\
 $f(x)(y)(z)$ means $\big((f(x))(y)\big)(z)$. 
\item Function applications group stronger than infix operations:\\
 $x+f(y)$ means $x+(f(y))$, just like in mathematics.
\end{itemize}
To specify the type of the argument, I will use a colon in the superscript,
for example: $x^{:\text{Int}}\Rightarrow x+2$.

The convention of grouping functions to the right reduces the number
of parentheses for curried function types used most often. It is rare
to find use for function types such as $(\left(a\Rightarrow b\right)\Rightarrow c)\Rightarrow d$
that require many parentheses with this convention.

Here are some more examples of performing function applications symbolically.
I will omit types for brevity, since every non-function value is of
type \texttt{}\lstinline!Int! in these examples.
\begin{align*}
\left(x^{:\text{Int}}\Rightarrow x*2\right)(10) & =10*2=20\quad.\\
\left(p\Rightarrow z\Rightarrow z*p\right)\left(t\right) & =(z\Rightarrow z*t)\quad.\\
\left(p\Rightarrow z\Rightarrow z*p\right)(t)(4) & =(z\Rightarrow z*t)(4)=4*t\quad.
\end{align*}
Some results of these computation are integer values such as $20$;
in other cases, results are \emph{function} \emph{values}\index{function value}
such as $z\Rightarrow z*t$.

In the following examples, some function arguments are themselves
functions:
\begin{align*}
\left(f\Rightarrow p\Rightarrow f(p)\right)\left(g\Rightarrow g(2)\right) & =\left(p\Rightarrow p(2)\right)\quad.\\
\left(f\Rightarrow p\Rightarrow f(p)\right)\left(g\Rightarrow g(2)\right)\left(x\Rightarrow x+4\right) & =\left(p\Rightarrow p(2)\right)\left(x\Rightarrow x+4\right)\\
 & =2+4=6\quad.
\end{align*}
Here I have been performing calculations step by step, as usual in
mathematics. A Scala program is evaluated in a similar way at run
time.

\subsection{Short syntax for function applications}

In mathematics, function applications are sometimes written without
parentheses, for instance $\cos x$ or $\text{arg}\,z$. There are
also cases where formulas such as $\sin2x=2\sin x\cos x$ imply parentheses
as $\sin\left(2x\right)=2\cdot\sin\left(x\right)\cdot\cos\left(x\right)$.

Many programming languages (such as ML, OCaml, F\#, Haskell, Elm,
PureScript) have adopted this ``short syntax'', in which parentheses
are optional for function arguments. The result is a concise notation
where \lstinline!f x! means the same as \lstinline!f(x)!. Parentheses
are still used where necessary to avoid ambiguity or for readability.\footnote{The no-parentheses syntax has a long history. It is used in Unix shell
commands, for example \lstinline!cp file1 file2!, as well as in the
programming language Tcl/Tk. In LISP and Scheme, each function application
is enclosed in parentheses but the arguments are separated by spaces,
for example \lstinline!(+ 1 2 3)!.}

The conventions for nameless functions in the short syntax become:
\begin{itemize}
\item Function expressions group everything to the right:\\
 $x\Rightarrow y\Rightarrow z\Rightarrow e$ means $x\Rightarrow\left(y\Rightarrow\left(z\Rightarrow e\right)\right)$.
\item Function applications group everything to the left:\\
 $f\,x\,y\,z$ means $\big((f\,x)\:y\big)\:z$. 
\item Function applications group stronger than infix operations: $x+f\,y$
means $x+(f\,y)$, just like in mathematics $x+\cos y$ groups $\cos y$
stronger than the infix ``$+$'' operation.
\end{itemize}
So, $x\Rightarrow y\Rightarrow a\,b\,c+p\,q$ means $x\Rightarrow\left(y\Rightarrow\left(\left(a\,b\right)\,c\right)+(p\,q)\right)$.
When this notation becomes hard to read correctly, one needs to add
parentheses, e.g.\ to write $f(x\Rightarrow g\,h)$ instead of $f\,x\Rightarrow g\,h$.

In this book, I will sometimes use this ``short syntax'' when reasoning
about code. Scala does not support the short syntax; in Scala, parentheses
need to be put around every curried argument. The infix method syntax
such as \texttt{}\lstinline!List(1,2,3) map func1! does not work
with curried functions in Scala.

\subsection{Higher-order functions}

The \textbf{order\index{order of a function}} of a function is the
number of function arrows ``\lstinline!=>!'' contained in the type
signature of that function. If a function's type signature contains
more than one function arrow, the function is called a \textbf{\index{higher-order function}higher-order}
function. A higher-order function takes a function as argument and/or
returns a function as its result value.

Examples:
\begin{lyxcode}
def~f1(x:~Int):~Int~=~x~+~10
\end{lyxcode}
The function \texttt{f1} has type signature \texttt{Int $\Rightarrow$
Int} and order 1, so it is \emph{not} a higher-order function.
\begin{lyxcode}
def~f2(x:~Int):~Int~$\Rightarrow$~Int~=~z~$\Rightarrow$~z~+~x
\end{lyxcode}
The function \texttt{f2} has type signature \texttt{Int $\Rightarrow$
Int $\Rightarrow$ Int} and is a higher-order function, of order 2. 
\begin{lyxcode}
def~f3(g:~Int~$\Rightarrow$~Int):~Int~=~g(123)
\end{lyxcode}
The function \texttt{f3} has type signature \texttt{(Int $\Rightarrow$
Int) $\Rightarrow$ Int} and is a higher-order function of order 2.

Although \texttt{f2} is a higher-order function, its higher-orderness
comes from the fact that the return value is of function type. An
equivalent computation can be performed by an uncurried function that
is not higher-order:
\begin{lyxcode}
scala>~def~f2u(x:~Int,~z:~Int):~Int~=~z~+~x
\end{lyxcode}
The Scala library defines methods to transform between curried and
uncurried functions:
\begin{lyxcode}
scala>~def~f2u(x:~Int,~z:~Int):~Int~=~z~+~x

scala>~val~f2c~=~(f2u~\_).curried

scala>~val~f2u1~=~Function.uncurried(f2c)
\end{lyxcode}
The syntax \lstinline!(f2u _)! is used in Scala to convert methods
to function values. Recall that Scala has two ways of defining a function:
one as a method\index{Scala method} (defined using \lstinline!def!),
another as a function value\index{function value} (defined using
\lstinline!val!). 

The methods \lstinline!.curried! and \lstinline!.uncurried! can
be easily implemented in Scala code, as we will see in the worked
examples.

Unlike \lstinline!f2!, the function \lstinline!f3! cannot be converted
to a non-higher-order function because \lstinline!f3! has an argument
of function type, rather than a return value of function type. Converting
to an uncurried form cannot eliminate an argument of function type.

\subsection{Worked examples: higher-order functions}
\begin{enumerate}
\item Using both \lstinline!def! and \lstinline!val!, define a function
that...
\begin{enumerate}
\item ...adds $20$ to its integer argument.
\begin{lyxcode}
def~fa(i:~Int):~Int~=~i~+~20

val~fa\_v:~(Int~$\Rightarrow$~Int)~=~k~$\Rightarrow$~k~+~20
\end{lyxcode}
It is not necessary to specify the type of the argument \texttt{k}
because we already fully specified the type \texttt{(Int => Int)}
of \texttt{fa\_v}. The parentheses around the type of \texttt{fa\_v}
are optional, I added them for clarity.
\item ...takes an integer $x$, and returns a \emph{function} that adds
$x$ to \emph{its} argument.
\begin{lyxcode}
def~fb(x:~Int):~(Int~$\Rightarrow$~Int)~=~k~$\Rightarrow$~k~+~x

val~fb\_v:~(Int~$\Rightarrow$~Int~$\Rightarrow$~Int)~=~x~$\Rightarrow$~k~$\Rightarrow$~k~+~x

def~fb\_v2(x:~Int)(k:~Int):~Int~=~k~+~x
\end{lyxcode}
Since functions are values, we can directly return new functions.
When defining the right-hand sides as function expressions in \texttt{fb}
and \texttt{fb\_v}, it is not necessary to specify the type of the
arguments \texttt{x} and \texttt{k} because we already fully specified
the type signatures of \texttt{fb} and \texttt{fb\_v}. The last version,
\texttt{fb\_v2}, may be easier to read and is equivalent to \texttt{fb\_v.}
\item ...takes an integer $x$ and returns \texttt{true} iff $x+1$ is a
prime. Use the function \texttt{is\_prime} defined previously.
\begin{lyxcode}
def~fc(x:~Int):~Boolean~=~is\_prime(x~+~1)

val~fc\_v:~(Int~$\Rightarrow$~Boolean)~=~x~$\Rightarrow$~is\_prime(x~+~1)
\end{lyxcode}
\item ...returns its integer argument unchanged. (This is called the \textbf{identity
function\index{identity function}} for integer type.)
\begin{lyxcode}
def~fd(i:~Int):~Int~=~i

val~fd\_v:~(Int~$\Rightarrow$~Int)~=~k~$\Rightarrow$~k
\end{lyxcode}
\item ...takes $x$ and always returns 123, ignoring its argument $x$.
(This is called a \textbf{constant function\index{constant function}}.)
\begin{lyxcode}
def~fe(x:~Int):~Int~=~123

val~fe\_v:~(Int~$\Rightarrow$~Int)~=~x~$\Rightarrow$~123
\end{lyxcode}
To emphasize the fact that the argument $x$ is ignored, use the special
syntax where $x$ is replaced by the underscore:
\begin{lyxcode}
val~fe\_v1:~(Int~$\Rightarrow$~Int)~=~\_~$\Rightarrow$~123
\end{lyxcode}
\item ...takes $x$ and returns a constant function that always returns
the fixed value $x$. (This is called the \textbf{constant combinator\index{constant combinator}}.)
\begin{lyxcode}
def~ff(x:~Int):~Int~$\Rightarrow$~Int~=~\_~$\Rightarrow$~x

val~ff\_v:~(Int~$\Rightarrow$~Int~$\Rightarrow$~Int)~=~x~$\Rightarrow$~\_~$\Rightarrow$~x

def~ff\_v2(x:~Int)(y:~Int):~Int~=~x
\end{lyxcode}
The syntax of \texttt{ff\_v2} may be easier to read, but then we cannot
omit the name \texttt{y} for the unused argument.
\end{enumerate}
\item Define a function \texttt{comp} that takes two functions $f:$\texttt{Int
$\Rightarrow$ Double} and $g:$\texttt{Double $\Rightarrow$ String}
as arguments, and returns a new function that computes $g(f(x))$.
What is the type of the function \texttt{comp}?
\begin{lyxcode}
{\footnotesize{}def~comp(f:~Int~$\Rightarrow$~Double,~g:~Double~$\Rightarrow$~String):~(Int~$\Rightarrow$~String)~=}{\footnotesize\par}

{\footnotesize{}~~x~$\Rightarrow$~g(f(x))}{\footnotesize\par}

{\footnotesize{}scala>~val~f:~Int~$\Rightarrow$~Double~=~x~$\Rightarrow$~5.67~+~x}{\footnotesize\par}

{\footnotesize{}scala>~val~g:~Double~$\Rightarrow$~String~=~x~$\Rightarrow$~f\textquotedbl x=\%3.2f\textquotedbl}{\footnotesize\par}

{\footnotesize{}scala>~val~h~=~comp(f,~g)}{\footnotesize\par}

{\footnotesize{}scala>~h(10)}{\footnotesize\par}
\end{lyxcode}
The function \texttt{comp} has two arguments, of types \texttt{Int
$\Rightarrow$ Double} and \texttt{Double $\Rightarrow$ String}.
The result value of \texttt{comp} is of type \texttt{Int $\Rightarrow$
String}, because \texttt{comp} returns a new function that takes an
argument $x$ of type \texttt{Int} and returns a \texttt{String}.
So the full type signature of the function \texttt{comp} is written
as
\begin{lyxcode}
///~(Int~$\Rightarrow$~Double,~Double~$\Rightarrow$~String)~$\Rightarrow$~(Int~$\Rightarrow$~String)
\end{lyxcode}
This is an example of a function that both takes other functions as
arguments \emph{and} returns a new function.
\item Define a function \texttt{uncurry2} that takes a curried function
of type \texttt{Int => Int => Int} and returns an uncurried equivalent
function of type \texttt{(Int, Int) => Int}.
\begin{lyxcode}
{\footnotesize{}def~uncurry2(f:~Int~$\Rightarrow$~Int~$\Rightarrow$~Int):~(Int,~Int)~$\Rightarrow$~Int~=}{\footnotesize\par}

{\footnotesize{}~~(x,~y)~$\Rightarrow$~f(x)(y)}{\footnotesize\par}
\end{lyxcode}
\end{enumerate}

\section{Discussion}

\subsection{Scope of bound variables}

A bound variable is invisible outside the scope of the expression
(often called \textbf{local scope\index{local scope}} whenever it
is clear which expression is being considered). This is why bound
variables may be renamed at will: no outside code could possibly use
them and depend on their values. However, outside code may define
variables that (by chance or by mistake) have the same name as a bound
variable inside the scope.

Consider this example from calculus: In the integral
\[
f(x)=\int_{0}^{x}\frac{dx}{1+x}\quad,
\]
a bound variable named $x$ is defined in \emph{two} local scopes:
in the scope of $f$ and in the scope of the nameless function $x\Rightarrow\frac{1}{1+x}$.
The convention in mathematics is to treat these two $x$'s as two
\emph{completely} \emph{different} variables that just happen to have
the same name. In sub-expressions where both of these bound variables
are visible, priority is given to the bound variable defined in the
closest inner scope. The outer definition of $x$ is \textbf{shadowed\index{name shadowing}},
i.e.\ hidden, by the definition of the inner $x$. For this reason,
mathematicians expect that evaluating $f(10)$ will give
\[
f(10)=\int_{0}^{10}\frac{dx}{1+x}\quad,
\]
rather than $\int_{0}^{10}\frac{dx}{1+10}$, because the outer definition
$x=10$ is shadowed, within the expression $\frac{1}{1+x}$, by the
closer definition of $x$ in the local scope of $x\Rightarrow\frac{1}{1+x}$.

Since this is the prevailing mathematical convention, the same convention
is adopted in FP. A variable defined in a local scope (i.e.\ a bound
variable) is invisible outside that scope but will shadow any outside
definitions of a variable with the same name. 

It is better to avoid name shadowing, because it usually decreases
the clarity of code and thus invites errors. Consider this function,
\[
x\Rightarrow x\Rightarrow x\quad.
\]
Let us decipher this confusing syntax. The symbol $\Rightarrow$ associates
to the right, so $x\Rightarrow x\Rightarrow x$ is the same as $x\Rightarrow\left(x\Rightarrow x\right)$.
So, it is a function that takes $x$ and returns $x\Rightarrow x$.
Since the returned nameless function, $\left(x\Rightarrow x\right)$,
may be renamed to $\left(y\Rightarrow y\right)$ without changing
its value, we can rewrite the code to
\[
x\Rightarrow\left(y\Rightarrow y\right)\quad.
\]
It is now easier to understand this code and reason about it. For
instance, it becomes clear that this function actually ignores its
argument $x$.

\section{Exercises}
\begin{enumerate}
\item Define a function of type \texttt{}\lstinline!Int => List[List[Int]] => List[List[Int]]!
similar to Exercise~\ref{subsec:ch1-transf-Exercise-4} except that
the hard-coded number $100$ must be a \emph{curried} first argument.
Implement Exercise~\ref{subsec:ch1-transf-Exercise-4} using this
function.
\item Define a function $q$ that takes a function $f^{:\text{Int}\Rightarrow\text{Int}}$
as its argument and returns a new function that computes $f(f(f(x)))$
for any given $x$. What is the required type of the function $q$?
\item Define a function \lstinline!curry2! that takes an uncurried function
of type \texttt{}\lstinline!(Int, Int) => Int! and returns an equivalent
but \emph{curried} function of type \texttt{}\lstinline!Int => Int => Int!.
\end{enumerate}

