%% LyX 2.2.0 created this file.  For more info, see http://www.lyx.org/.
%% Do not edit unless you really know what you are doing.
\documentclass[english]{beamer}
\usepackage[T1]{fontenc}
\usepackage[latin9]{inputenc}
\setcounter{secnumdepth}{3}
\setcounter{tocdepth}{3}
\usepackage{babel}
\usepackage{amsmath}
\ifx\hypersetup\undefined
  \AtBeginDocument{%
    \hypersetup{unicode=true,pdfusetitle,
 bookmarks=true,bookmarksnumbered=false,bookmarksopen=false,
 breaklinks=false,pdfborder={0 0 1},backref=false,colorlinks=true}
  }
\else
  \hypersetup{unicode=true,pdfusetitle,
 bookmarks=true,bookmarksnumbered=false,bookmarksopen=false,
 breaklinks=false,pdfborder={0 0 1},backref=false,colorlinks=true}
\fi

\makeatletter

%%%%%%%%%%%%%%%%%%%%%%%%%%%%%% LyX specific LaTeX commands.
%% Because html converters don't know tabularnewline
\providecommand{\tabularnewline}{\\}

%%%%%%%%%%%%%%%%%%%%%%%%%%%%%% Textclass specific LaTeX commands.
 % this default might be overridden by plain title style
 \newcommand\makebeamertitle{\frame{\maketitle}}%
 % (ERT) argument for the TOC
 \AtBeginDocument{%
   \let\origtableofcontents=\tableofcontents
   \def\tableofcontents{\@ifnextchar[{\origtableofcontents}{\gobbletableofcontents}}
   \def\gobbletableofcontents#1{\origtableofcontents}
 }
 \newenvironment{lyxcode}
   {\par\begin{list}{}{
     \setlength{\rightmargin}{\leftmargin}
     \setlength{\listparindent}{0pt}% needed for AMS classes
     \raggedright
     \setlength{\itemsep}{0pt}
     \setlength{\parsep}{0pt}
     \normalfont\ttfamily}%
    \def\{{\char`\{}
    \def\}{\char`\}}
    \def\textasciitilde{\char`\~}
    \item[]}
   {\end{list}}

%%%%%%%%%%%%%%%%%%%%%%%%%%%%%% User specified LaTeX commands.
\usetheme[secheader]{Boadilla}
\usecolortheme{seahorse}
\title[Chapter 1 - Expressions]{Chapter 1: Values, types, expressions, functions}
\author{Sergei Winitzki}
\date{October 21, 2017}
\institute[ABTB]{Academy by the Bay}

\makeatother

\begin{document}
\frame{\titlepage}
\begin{frame}{What is ``Functional Programming''?}

Functional programming...
\begin{itemize}
\item treats programs as mathematical expressions
\item uses age-old mathematical intuition to design software
\item is natural and effective in OCaml, Haskell, F\#, Scala, Swift, etc.
\item ... but not in C, C++, JavaScript, Java (before version 8), or Python!
\end{itemize}
\bigskip{}
We will be using Scala for all examples...
\begin{itemize}
\item ...but the same material looks very similar in the other languages 
\end{itemize}
\end{frame}

\begin{frame}{Examples of functional programs}

Compute the factorial of a natural number:
\[
n!=\prod_{k=1}^{n}k
\]

Check whether a natural number is a prime:
\[
prime(n)=\forall i\text{ such that }2\leq i<n\ :\ n\mod i\neq0
\]

Count how many even numbers there are in a given set $S$ of integers:
\begin{align*}
count\_even & =\sum_{k\in S}is\_even(k)\\
\text{where we defined }is\_even(k) & =\begin{cases}
1 & \text{if }k\mod2=0\\
0 & \text{otherwise}
\end{cases}
\end{align*}

\begin{itemize}
\item Scala programs implementing this are similar to the math
\begin{itemize}
\item Programs in Java or Python are \emph{not} similar to the math!
\end{itemize}
\end{itemize}
\end{frame}

\begin{frame}{What exactly is ``math-like'' about that code?}

\begin{itemize}
\item The code represents a \emph{mathematical expression} that we want
to compute
\item Each value is immutable and has a fixed \emph{type} (integer, set,
function, etc.)
\item The code can define new names or new functions \emph{within an expression}
\item There are no loops, no ``goto'' or ``repeat''
\begin{itemize}
\item Have you ever seen a math book that says things such as,\\
\emph{~ ``now change $k$ to $k-1$ and repeat Equation 123 until
$k=0$''~?}\\
or\\
~ \emph{``at this point, if $x>0$ then go back to page 208, else
assign $x=0$''?}
\end{itemize}
\end{itemize}
\end{frame}

\begin{frame}{Adapting math to programming I}


\framesubtitle{Values, expressions, and types}

In math, there are two kinds of ``variables'':
\begin{itemize}
\item named constant values,
\end{itemize}
\begin{lyxcode}
\textcolor{blue}{\footnotesize{}val~a~=~123}{\footnotesize \par}
\end{lyxcode}
\begin{itemize}
\item function arguments,
\end{itemize}
\begin{lyxcode}
\textcolor{blue}{\footnotesize{}def~f(x:~Int,~y:~Int)~=~x~+~y~-~1}{\footnotesize \par}
\end{lyxcode}
Math texts never try to ``modify'' a value, so...
\begin{itemize}
\item ``\texttt{\textcolor{blue}{\footnotesize{}val}}''s and function
arguments are immutable
\end{itemize}
Each value has a fixed type (\texttt{\textcolor{blue}{\footnotesize{}Int}},
\texttt{\textcolor{blue}{\footnotesize{}Boolean}}, \texttt{\textcolor{blue}{\footnotesize{}Set{[}Int{]}}},
etc.)
\begin{itemize}
\item Type represents the set of possible values of the function argument
\item Type is automatically assigned to named constants
\end{itemize}
\end{frame}

\begin{frame}{Adapting math to programming II}


\framesubtitle{Anonymous functions vs.~named functions}

There are two ways of defining a function in Scala:
\begin{itemize}
\item \textbf{named} function \textendash{} using \texttt{\textcolor{blue}{\footnotesize{}def}}
with a name
\end{itemize}
\begin{lyxcode}
\textcolor{blue}{\footnotesize{}def~merge(x:~Int,~y:~Int):~Int~=~\{~x~+~y~-~1~\}}{\footnotesize \par}
\end{lyxcode}
\begin{itemize}
\item \textbf{anonymous} function \textendash{} in math notation, $x\mapsto f(x)$:
\end{itemize}
\begin{lyxcode}
\textcolor{blue}{\footnotesize{}(x:~Int,~y:~Int)~=>~x~+~y~-~1}{\footnotesize \par}
\end{lyxcode}
Anonymous functions are \emph{values}:
\begin{itemize}
\item they are immutable and have a fixed type, e.g.~\texttt{\textcolor{blue}{\footnotesize{}(Int,
Int) => Int}}{\footnotesize \par}
\item they can be assigned a name and used later in an expression:
\end{itemize}
\begin{lyxcode}
\textcolor{blue}{\footnotesize{}val~double:~(Int~=>~Int)~=~\{~x~=>~x~{*}~2~\};~double(y)}{\footnotesize \par}
\end{lyxcode}
\begin{itemize}
\item or they can be used directly as arguments of other functions:
\end{itemize}
\begin{lyxcode}
\textcolor{blue}{\footnotesize{}(1~to~100).map(x~=>~x~{*}~x)}{\footnotesize \par}
\end{lyxcode}
\end{frame}

\begin{frame}{Some collections in Scala}

What is \textcolor{blue}{\footnotesize{}(1 to 100)}? What type does
it have?
\begin{lyxcode}
{\footnotesize{}scala>~(1~to~100)}{\footnotesize \par}

{\footnotesize{}res0:~scala.collection.immutable.Range.Inclusive~=~Range~1~to~100}{\footnotesize \par}
\end{lyxcode}
\begin{itemize}
\item Sequence, \texttt{\textcolor{blue}{\footnotesize{}Seq}} and its subtypes:
\texttt{\textcolor{blue}{\footnotesize{}List}}, \texttt{\textcolor{blue}{\footnotesize{}IndexedSeq}},
etc.
\end{itemize}
\begin{lyxcode}
\textcolor{blue}{\footnotesize{}val~a:~Seq{[}Int{]}~=~Seq(2,~4,~6,~8)}{\footnotesize \par}

\textcolor{blue}{\footnotesize{}val~b~=~a(0)~//}~now\textcolor{blue}{\footnotesize{}~b:~Int~==~2~}{\footnotesize \par}
\end{lyxcode}
\begin{itemize}
\item Set: \texttt{\textcolor{blue}{\footnotesize{}Set}}{\footnotesize \par}
\end{itemize}
\begin{lyxcode}
\textcolor{blue}{\footnotesize{}val~b:~Set{[}String{]}~=~Set(\textquotedbl{}x\textquotedbl{},~\textquotedbl{}y\textquotedbl{},~\textquotedbl{}z\textquotedbl{})}{\footnotesize \par}
\end{lyxcode}
\begin{itemize}
\item Dictionary: \texttt{\textcolor{blue}{\footnotesize{}Map}}{\footnotesize \par}
\end{itemize}
\begin{lyxcode}
\textcolor{blue}{\footnotesize{}val~c:~Map{[}String,~Int{]}~=~Map(\textquotedbl{}x\textquotedbl{}~->~5,~\textquotedbl{}y\textquotedbl{}~->~3,~\textquotedbl{}z\textquotedbl{}~->~1)}{\footnotesize \par}
\end{lyxcode}
Note the \textbf{parameterized} types \texttt{\textcolor{blue}{\footnotesize{}Seq{[}Int{]}}}
and \texttt{\textcolor{blue}{\footnotesize{}Map{[}String, Int{]}}}{\footnotesize \par}

Collections can have any element type: \texttt{\textcolor{blue}{\footnotesize{}String}},
\texttt{\textcolor{blue}{\footnotesize{}Boolean}}, \texttt{\textcolor{blue}{\footnotesize{}Set{[}Seq{[}Int{]}{]}}},
...
\end{frame}

\begin{frame}{Adapting math to programming III}


\framesubtitle{Encoding sums, products, quantifiers using anonymous functions}

The methods \textcolor{blue}{\footnotesize{}map}, \textcolor{blue}{\footnotesize{}filter},\textcolor{blue}{\footnotesize{}
forall}, \textcolor{blue}{\footnotesize{}exists} are defined on all
collections

The methods \textcolor{blue}{\footnotesize{}sum}, \textcolor{blue}{\footnotesize{}product}
are defined on collections of \emph{numbers}
\begin{center}
\begin{tabular}{|c|c|}
\hline 
\emph{Mathematical notation} & \emph{Scala code}\tabularnewline
\hline 
\hline 
$x\mapsto\sqrt{x^{2}+1}$ & \texttt{\textcolor{blue}{\footnotesize{}x => math.sqrt(x {*} x + 1)}}\tabularnewline
\hline 
sequence $\left[1,~2,~...,~n\right]$ & \texttt{\textcolor{blue}{\footnotesize{}(1 to n)}}\tabularnewline
\hline 
sequence $\left[f(1),~...,~f(n)\right]$ & \texttt{\textcolor{blue}{\footnotesize{}(1 to n).map(k => f(k))}}\tabularnewline
\hline 
$\sum_{k=1}^{n}k^{2}$ & \texttt{\textcolor{blue}{\footnotesize{}(1 to n).map(k => k{*}k).sum}}\tabularnewline
\hline 
$\prod_{k=1}^{n}f(k)$ & \texttt{\textcolor{blue}{\footnotesize{}(1 to n).map(f).product}}\tabularnewline
\hline 
$\forall k\text{ such that }1\leq k\leq n:p(k)~\text{holds}$ & \texttt{\textcolor{blue}{\footnotesize{}(1 to n).forall(k => p(k))}}\tabularnewline
\hline 
$\exists k,\:1\leq k\leq n\text{ such that }p(k)~\text{holds}$ & \texttt{\textcolor{blue}{\footnotesize{}(1 to n).exists(k => p(k))}}\tabularnewline
\hline 
${\displaystyle \sum_{k\in S:\,p(k)\:\text{holds}}}f(k)$ & \texttt{\textcolor{blue}{\footnotesize{}s.filter(p).map(f).sum}}\tabularnewline
\hline 
\end{tabular}
\par\end{center}

\end{frame}

\begin{frame}{Adapting math to programming IV}


\framesubtitle{Higher-order functions}

Derivatives and integrals could be implemented as functions:
\begin{lyxcode}
\textcolor{blue}{\footnotesize{}def~deriv(f:~Double~=>~Double):~(Double~=>~Double)~=~\{~???~\}}{\footnotesize \par}

\textcolor{blue}{\footnotesize{}def~integ(f:~Double~=>~Double,~range:~(Double,~Double)):~Double~=~???}{\footnotesize \par}
\end{lyxcode}
\textbf{Higher-order functions} take function arguments and/or return
functions
\begin{itemize}
\item Many computations with sequences, sets, dictionaries can be done using
higher-order functions, \emph{without loops}, concisely and error-free
\item The Scala standard library has many more higher-order methods for
collections
\begin{itemize}
\item \texttt{\textcolor{blue}{\footnotesize{}size}}, \texttt{\textcolor{blue}{\footnotesize{}reduce}},
\texttt{\textcolor{blue}{\footnotesize{}zip}}, \texttt{\textcolor{blue}{\footnotesize{}zipWithIndex}},
\texttt{\textcolor{blue}{\footnotesize{}flatten}}, \texttt{\textcolor{blue}{\footnotesize{}flatMap}},
\texttt{\textcolor{blue}{\footnotesize{}foldLeft}}, \texttt{\textcolor{blue}{\footnotesize{}foldRight}},
\texttt{\textcolor{blue}{\footnotesize{}scanLeft}}, \texttt{\textcolor{blue}{\footnotesize{}collect}},
\texttt{\textcolor{blue}{\footnotesize{}distinct}}, \texttt{\textcolor{blue}{\footnotesize{}groupBy}},
...
\end{itemize}
\item Write code by formulating the problem as a mathematical expression
\end{itemize}
\end{frame}

\begin{frame}{Examples}

\begin{itemize}
\item Using both \texttt{\textcolor{blue}{\footnotesize{}def}} and \texttt{\textcolor{blue}{\footnotesize{}val}},
define a function that...
\begin{enumerate}
\item adds 20 to its integer argument
\item takes an integer \texttt{\textcolor{blue}{\footnotesize{}x}}, and
returns a function that adds \texttt{\textcolor{blue}{\footnotesize{}x}}
to its argument
\item takes an integer \texttt{\textcolor{blue}{\footnotesize{}x}}, and
returns \texttt{\textcolor{blue}{\footnotesize{}true}} iff \texttt{\textcolor{blue}{\footnotesize{}x+1}}
is prime
\end{enumerate}
\item What are the types of the functions in Examples 1 - 3?
\item Compute the average of all numbers in a sequence of type \texttt{\textcolor{blue}{\footnotesize{}Seq{[}Double{]}}}.
Use \texttt{\textcolor{blue}{\footnotesize{}sum}} and \texttt{\textcolor{blue}{\footnotesize{}size}}
but no loops.
\item Given $n$, compute the Wallis product truncated up to $\frac{2n}{2n+1}$:
\[
\frac{2}{1}\frac{2}{3}\frac{4}{3}\frac{4}{5}\frac{6}{5}\frac{6}{7}...\frac{2n}{2n+1}.
\]
Use a sequence of \texttt{\textcolor{blue}{\footnotesize{}Int}} or
\texttt{\textcolor{blue}{\footnotesize{}Double}} numbers, \texttt{\textcolor{blue}{\footnotesize{}map}},
and \texttt{\textcolor{blue}{\footnotesize{}product}}.
\item Given \texttt{\textcolor{blue}{\footnotesize{}s:~Seq{[}Set{[}Int{]}{]}}},
compute the sequence containing the sets of size at least 3. Use \texttt{\textcolor{blue}{\footnotesize{}map}},
\texttt{\textcolor{blue}{\footnotesize{}filter}}, \texttt{\textcolor{blue}{\footnotesize{}size}}.
The result must be again of type \texttt{\textcolor{blue}{\footnotesize{}Seq{[}Set{[}Int{]}{]}}}.
\end{itemize}
\end{frame}

\begin{frame}{Summary}

What problems can we solve now?
\begin{itemize}
\item Compute mathematical expressions involving sums, products, and quantifiers,
based on integer ranges (such as $\sum_{k=1}^{n}f(k)$ etc.)
\item Implement functions that return other functions
\item Work on collections using \texttt{\textcolor{blue}{\footnotesize{}map}}
and other library methods
\end{itemize}
What kinds of problems are not solved with these tools?
\begin{itemize}
\item Compute the smallest $n$ such that $f(f(f(...f(1)...)>1000$, where
the function $f()$ is applied $n$ times.
\item Find the median in an array of integers.
\end{itemize}
Why can't we solve such problems yet?
\begin{itemize}
\item Because we can't yet put \emph{mathematical induction} into code
\end{itemize}
\end{frame}

\end{document}
