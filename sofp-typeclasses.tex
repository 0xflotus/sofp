
\chapter{Type-level functions and typeclasses}

\section{Slides}

\subsection{Motivation for type classes I: Restricting type arguments}

We would like a generic \texttt{\textcolor{blue}{\footnotesize{}sum}}
implementation for \texttt{\textcolor{blue}{\footnotesize{}Seq{[}Int{]},
Seq{[}Double{]}}}, etc.

but we cannot generalize \texttt{\textcolor{blue}{\footnotesize{}sum}}
to arbitrary types \texttt{\textcolor{blue}{\footnotesize{}T}} like
this:

\textcolor{blue}{\footnotesize{}def sum{[}T{]}(s: Seq{[}T{]}): T =
???}{\footnotesize\par}

this can work only for \texttt{\textcolor{blue}{\footnotesize{}T}}
that are ``summable'' in some sense

We would like to define \texttt{\textcolor{blue}{\footnotesize{}fmap}}
for functors, using the available \texttt{\textcolor{blue}{\footnotesize{}map}}
\ 

but we cannot generalize \texttt{\textcolor{blue}{\footnotesize{}fmap}}
to arbitrary type constructors \texttt{\textcolor{blue}{\footnotesize{}F{[}\_{]}}}:

\textcolor{blue}{\footnotesize{}def fmap{[}F{[}\_{]}, A, B{]}(f: A
$\Rightarrow$ B): F{[}A{]} $\Rightarrow$ F{[}B{]} = ???}{\footnotesize\par}

this can work only for type constructors \texttt{\textcolor{blue}{\footnotesize{}F{[}\_{]}}}
that are functors

We would like to define functions whose type arguments, such as \texttt{\textcolor{blue}{\footnotesize{}T}}
or \texttt{\textcolor{blue}{\footnotesize{}F{[}\_{]}}}, are required
to belong to a \emph{certain subset} of possible types

We could then use the known properties of these type arguments

We would also like to add new supported types as needed

This is similar to the concept of \emph{partial functions} \textendash{}
applied to types


\subsection{Motivation for type classes II: Partial functions on types}

Functions can be \textbf{total} or \textbf{partial}

Total function: has a result for all argument values

Partial function: has \emph{no result} for \emph{some} argument values

Also, functions can be, in principle, \{from/to\} \{values/types\}:
\begin{center}
\begin{tabular}{|c|c|c|}
\hline 
function: & from value & from type\tabularnewline
\hline 
\hline 
to value & \texttt{\textcolor{blue}{\footnotesize{}def f(x:\ Int):\ Int}} & \texttt{\textcolor{blue}{\footnotesize{}def point{[}A{]}: A $\Rightarrow$
List{[}A{]}}}\tabularnewline
\hline 
to type & \emph{dependent type} & \texttt{\textcolor{blue}{\footnotesize{}type MyData{[}A{]} = Either{[}Int,
A{]}}}\tabularnewline
\hline 
\end{tabular}
\par\end{center}

Evaluation: value-to-value \textendash{} run time, type-to-value \textendash{}
compile time

if we use type casts, type-to-value can become run-time (\emph{yuck!})
\begin{center}
\begin{tabular}{|c|c|c|}
\hline 
partial function: &  value-level (PVVF) & from type (PTVF)\tabularnewline
\hline 
\hline 
example: & \texttt{\textcolor{blue}{\footnotesize{}\{ case Some(x) $\Rightarrow$
x-1 \}}} & \texttt{\textcolor{blue}{\footnotesize{}implicitly{[}T{]}}}\tabularnewline
\hline 
when misapplied: & exception at run time & error at compile time\tabularnewline
\hline 
\end{tabular}
\par\end{center}

Type classes provide a systematic way of managing PTVFs

we can apply a PTVF to type \texttt{\textcolor{blue}{\footnotesize{}T}}
if \texttt{\textcolor{blue}{\footnotesize{}T}} ``belongs to a certain
type class''


\subsection{Example of using value-level PFs: The caveats}

Filter a \texttt{\textcolor{blue}{\footnotesize{}Seq{[}Either{[}Int,
Boolean{]}{]}}}, then apply \texttt{\textcolor{blue}{\footnotesize{}map}}
with a PF:

\textcolor{blue}{\footnotesize{}val s: Seq{[}Int{]} = Seq( Left(1),
Right(true), Left(2) )}{\footnotesize\par}

\textcolor{blue}{\footnotesize{}  .filter(\_.isLeft) }\textcolor{gray}{\footnotesize{}//
result here is still of type }\textcolor{blue}{\footnotesize{}Seq{[}Either{[}...{]}{]}}{\footnotesize\par}

\textcolor{blue}{\footnotesize{}  .map \{ case Left(x) $\Rightarrow$
x \} }\textcolor{gray}{\footnotesize{}// result is of type }\textcolor{blue}{\footnotesize{}Seq{[}Int{]}}\textcolor{gray}{\footnotesize{}
but unsafe}{\footnotesize\par}

``We know'' it is okay to apply this PF here...

but the types do not show this, \textendash{} compile-time checking
doesn't help

if refactored, the code may become wrong and break \emph{at run time}

The type-safe version uses \texttt{\textcolor{blue}{\footnotesize{}.collect}}
instead of \texttt{\textcolor{blue}{\footnotesize{}.filter().map()}}:

\textcolor{blue}{\footnotesize{}val s: Seq{[}Int{]} = Seq( Left(1),
Right(true), Left(2) )}{\footnotesize\par}

\textcolor{blue}{\footnotesize{}  .collect \{ case Left(x) $\Rightarrow$
x \}}\textcolor{gray}{\footnotesize{}  // result is safe, of type
}\textcolor{blue}{\footnotesize{}Seq{[}Int{]}}{\footnotesize\par}

PFs are only safe in certain places, such as within \texttt{\textcolor{blue}{\footnotesize{}.collect()}}{\footnotesize\par}

make functions total: either add code, or use \emph{more restrictive
types}

e.g.\ types such as ``non-empty list'', ``positive number'',
\texttt{\textcolor{blue}{\footnotesize{}Some{[}T{]}}}, etc.

\textcolor{blue}{\footnotesize{}def f(xs: NonEmptyList{[}Int{]}) =
\{}{\footnotesize\par}

\textcolor{blue}{\footnotesize{}  val h = xs.head}\textcolor{gray}{\footnotesize{}
 // safe and checked at compile time}{\footnotesize\par}

\textcolor{blue}{\footnotesize{}\}}{\footnotesize\par}

Can we restrict the \emph{type} \emph{parameter(s)} of a PTVF to a
subset of types?


\subsection{Managing PTFs by hand I: Using GADTs}

PTTFs: Partial Type-to-Type Functions

A type constructor that accepts only certain types as parameters:

\textcolor{blue}{\footnotesize{}sealed trait }\texttt{\textcolor{blue}{\footnotesize{}MyTC}}\textcolor{blue}{\footnotesize{}{[}A{]}
}\textcolor{gray}{\footnotesize{}// ``sealed'' GADT \textendash{}
user code can't add cases}{\footnotesize\par}

\textcolor{blue}{\footnotesize{}final case class Case1(d: Double)
extends }\texttt{\textcolor{blue}{\footnotesize{}MyTC}}\textcolor{blue}{\footnotesize{}{[}Int{]}}{\footnotesize\par}

\textcolor{blue}{\footnotesize{}final case class Case2() extends }\texttt{\textcolor{blue}{\footnotesize{}MyTC}}\textcolor{blue}{\footnotesize{}{[}String{]}}\textcolor{gray}{\footnotesize{}
// whatever}{\footnotesize\par}

It looks like we have defined \texttt{\textcolor{blue}{\footnotesize{}MyTC{[}}}\textcolor{blue}{\footnotesize{}A}\texttt{\textcolor{blue}{\footnotesize{}{]}}}
for any type \textcolor{blue}{\footnotesize{}A} ...

actually, we can only ever create values of \texttt{\textcolor{blue}{\footnotesize{}MyTC{[}Int{]}}}
or \texttt{\textcolor{blue}{\footnotesize{}MyTC{[}String{]}}}{\footnotesize\par}

see example code

$\text{MyTC}^{A}$ is a PTTF because its values exist only for $A\in\left\{ \text{Int};\,\text{String}\right\} $

The \textbf{type domain} $\left\{ \text{Int};\,\text{String}\right\} $
is defined \emph{at compile time}

Note: $\text{MyTC}^{A}$ cannot be a functor since it is not defined
for all types $A$

When to use GADTs:

for domain modeling (e.g.\ queries with a fixed set of result types)

for DSLs that represent typed expressions

Alternatively, a PTTF can be a \texttt{\textcolor{blue}{\footnotesize{}trait}}
with some implementation code


\subsection{Managing PTFs by hand II: Using OO method overriding}

PTVFs: Partial Type-to-Value Functions \textendash{} the object-oriented
way

A trait with \texttt{\textcolor{blue}{\footnotesize{}def}} methods
that are overridden:

\textcolor{blue}{\footnotesize{}sealed trait HasPlus{[}A{]} \{}{\footnotesize\par}

\textcolor{blue}{\footnotesize{}  def plus(a1: A, a2: A): A}{\footnotesize\par}

\textcolor{blue}{\footnotesize{}\}}{\footnotesize\par}

\textcolor{blue}{\footnotesize{}final case class CaseInt() extends
HasPlus{[}Int{]} \{}{\footnotesize\par}

\textcolor{blue}{\footnotesize{}  override def plus(a1: Int, a2: Int):
Int = a1 + a2}{\footnotesize\par}

\textcolor{blue}{\footnotesize{}\}}{\footnotesize\par}

\textcolor{blue}{\footnotesize{}final case class CaseString() extends
HasPlus{[}String{]} \{}{\footnotesize\par}

\textcolor{blue}{\footnotesize{}  override def plus(a1: String, a2:
String): String = a1 + a2}{\footnotesize\par}

\textcolor{blue}{\footnotesize{}\}}{\footnotesize\par}

\emph{Similar} to having defined \texttt{\textcolor{blue}{\footnotesize{}plus{[}A{]}}}
for $A\in\left\{ \text{Int};\,\text{String}\right\} $

Limitations:

We can only call \texttt{\textcolor{blue}{\footnotesize{}plus() }}via
a value of type \texttt{\textcolor{blue}{\footnotesize{}HasPlus{[}A{]}}}
\ 

All PTVFs must be declared up front in the trait

Not extensible \textendash{} cannot add new PTVFs later

Not compositional \textendash{} cannot use this in other PTVFs defined
later


\subsection{Managing PTFs by hand III: ``Type Evidence'' arguments}

PTVFs: Partial Type-to-Value Functions \textendash{} the general case

To define a function \texttt{\textcolor{blue}{\footnotesize{}func{[}A{]}(...)}}
only for certain types \texttt{\textcolor{blue}{\footnotesize{}A}}:

create a PTTF defined only for the relevant types \texttt{\textcolor{blue}{\footnotesize{}A}},
e.g.\ \texttt{\textcolor{blue}{\footnotesize{}IsGood{[}A{]}}} \ 

create some values of types \texttt{\textcolor{blue}{\footnotesize{}IsGood{[}A{]}}}
for relevant types \texttt{\textcolor{blue}{\footnotesize{}A}} as
needed

add an \emph{extra argument} \texttt{\textcolor{blue}{\footnotesize{}ev:\ IsGood{[}A{]}}}
(\textbf{type evidence}) to \texttt{\textcolor{blue}{\footnotesize{}func{[}A{]}(...)}}\ 

What we gained:

it is now impossible to call \texttt{\textcolor{blue}{\footnotesize{}func{[}A{]}}}
with an unsupported type \texttt{\textcolor{blue}{\footnotesize{}A}}{\footnotesize\par}

trying to do so will fail \emph{at compile time} \ \textendash{}
TE values won't type-check

new supported types can be added in user code if \texttt{\textcolor{blue}{\footnotesize{}IsGood}}
is not \texttt{\textcolor{blue}{\footnotesize{}sealed}}{\footnotesize\par}

The cost:

all calls to \texttt{\textcolor{blue}{\footnotesize{}func{[}A{]}(args)}}
will now become \texttt{\textcolor{blue}{\footnotesize{}func{[}A{]}(args,
ev)}}{\footnotesize\par}

one TE value \texttt{\textcolor{blue}{\footnotesize{}ev}} needs to
be created for \emph{each} supported type \texttt{\textcolor{blue}{\footnotesize{}A}}\ 

we now need to keep passing all these TE values around the code

Mitigate these issues in Scala by using \texttt{\textcolor{blue}{\footnotesize{}implicit}}
values:

TE arguments are explicit only at \texttt{\textcolor{blue}{\footnotesize{}func}}
declaration site

once defined as \texttt{\textcolor{blue}{\footnotesize{}implicit}},
TE values are passed around invisibly

new \texttt{\textcolor{blue}{\footnotesize{}implicit}} values can
be built up automatically from previous ones


\subsection{Scala's mechanism of ``\texttt{implicit} values''}

Implicit values are:

declared as \texttt{\textcolor{blue}{\footnotesize{}implicit val x:\ SomeType
= ...}}{\footnotesize\par}

also have \texttt{\textcolor{blue}{\footnotesize{}implicit def f{[}T{]}(...)\ =
...}} and \texttt{\textcolor{blue}{\footnotesize{}implicit class(...)}}{\footnotesize\par}

automatically passed into functions that declare extra arguments as

\texttt{\textcolor{blue}{\footnotesize{}def f(args...)(implicit x: SomeType)
= ...}}{\footnotesize\par}

searched in local scope, imports, companion objects, parent classes

having $\geq2$ \texttt{\textcolor{blue}{\footnotesize{}implicit}}
values of the same type is a compile-time error!

Special short syntax for declaring implicit TE arguments in a PTVF:

\texttt{\textcolor{blue}{\footnotesize{}def func{[}A: TC1, B: TC2{]}(args...)
= ...}}{\footnotesize\par}

This is entirely equivalent to this longer code:

\texttt{\textcolor{blue}{\footnotesize{}def func{[}A, B{]}(args...)(implicit
evA: TC1{[}A{]}, evB: TC2{[}B{]})= ...}}{\footnotesize\par}

standard library has \texttt{\textcolor{blue}{\footnotesize{}def implicitly{[}A{]}(implicit
x:\ A):\ A = x}}{\footnotesize\par}

We still need to:

declare \texttt{\textcolor{blue}{\footnotesize{}MyTypeClass{[}A{]}}}
as a PTTF elsewhere

create TE values of various types and declare them as \texttt{\textcolor{blue}{\footnotesize{}implicit}}
\ 


\subsection{Type classes I}

The general definition

A \textbf{type class} is a set of PTVFs that all have the same type
domain

In terms of specific code to be written, a type class is:

a PTTF, e.g.\ \texttt{\textcolor{blue}{\footnotesize{}MyTypeClass{[}T{]}}}
with some code that creates TE values, \emph{and}

the desired PTVFs that use this PTTF to define their type domain

for many important use cases, the PTVFs must also satisfy certain
laws

A type \texttt{\textcolor{blue}{\footnotesize{}T}} ``\textbf{belongs
to} the type class \texttt{\textcolor{blue}{\footnotesize{}MyTypeClass}}''
if a TE value exists

i.e.\ if \emph{some} value of type \texttt{\textcolor{blue}{\footnotesize{}MyTypeClass{[}T{]}}}
can be found

A function \texttt{\textcolor{blue}{\footnotesize{}func{[}T{]}}} ``\textbf{requires}
the type class \texttt{\textcolor{blue}{\footnotesize{}MyTypeClass}}
for \texttt{\textcolor{blue}{\footnotesize{}T}}'' if one of \texttt{\textcolor{blue}{\footnotesize{}func}}'s
arguments is a value of PTTF-constructed type \texttt{\textcolor{blue}{\footnotesize{}MyTypeClass{[}T{]}}}
\ 

that argument is the \textbf{type class instance} for the type parameter
\texttt{\textcolor{blue}{\footnotesize{}T}}{\footnotesize\par}

this \textbf{constrains} the type parameter \texttt{\textcolor{blue}{\footnotesize{}T}}
to \textbf{belong to} the type class

this is how we know that \texttt{\textcolor{blue}{\footnotesize{}func{[}T{]}}}
is a PTVF


\subsection{Type classes II}

Implementation in Scala

A type class is typically implemented as:

a trait with a type parameter, e.g.\ \texttt{\textcolor{blue}{\footnotesize{}trait
MyTC{[}T{]}}}{\footnotesize\par}

code that creates values of type \texttt{\textcolor{blue}{\footnotesize{}MyTC{[}T{]}}}
for various \texttt{\textcolor{blue}{\footnotesize{}T}}{\footnotesize\par}

these values are declared as \texttt{\textcolor{blue}{\footnotesize{}implicit}}
and made available via imports or in the companion objects for the
specific types \texttt{\textcolor{blue}{\footnotesize{}T}}{\footnotesize\par}

some functions with implicit argument(s) of type \texttt{\textcolor{blue}{\footnotesize{}MyTC{[}T{]}}}{\footnotesize\par}

these functions are usually \texttt{\textcolor{blue}{\footnotesize{}def}}
methods in a trait, but don't have to be

laws for these functions may need to be enforced by property tests

A TE value should carry all information the PTVFs need about the type
\texttt{\textcolor{blue}{\footnotesize{}T}}{\footnotesize\par}

usually, the trait \texttt{\textcolor{blue}{\footnotesize{}MyTC{[}T{]}}}
contains all the PTVFs as \texttt{\textcolor{blue}{\footnotesize{}def}}
methods

in simpler cases, TE can be a data type (not a trait with \texttt{\textcolor{blue}{\footnotesize{}def}}
methods)

a trait with \texttt{\textcolor{blue}{\footnotesize{}def}} methods
is necessary for \emph{higher-order} type functions

additional PTVFs (with unchanged PTTF) can be added later

no need to modify the code of \texttt{\textcolor{blue}{\footnotesize{}MyTC{[}T{]}}}
if the type domain is unchanged

can combine with other PTTFs/PTVFs defined later

See example code


\subsection{Examples of type classes I}

Some simple PTFs and their use cases

A type \texttt{\textcolor{blue}{\footnotesize{}T}} is a \textbf{semigroup}
if it has an \emph{associative} binary operation

\texttt{\textcolor{blue}{\footnotesize{}def op{[}T{]}(x: T, y: T): T}}{\footnotesize\par}

a bare-bones operation, no inverse \textendash{} just ``can combine'' 

A type \texttt{\textcolor{blue}{\footnotesize{}T}} is \textbf{pointed}
if there exists a function \texttt{\textcolor{blue}{\footnotesize{}point{[}T{]}:\ T}}{\footnotesize\par}

This is a special, somehow ``naturally'' selected value of that
type

Examples: \texttt{\textcolor{blue}{\footnotesize{}0:\ Int}}; \texttt{\textcolor{blue}{\footnotesize{}\textquotedbl\textquotedbl :\ String}};
\texttt{\textcolor{blue}{\footnotesize{}identity{[}A{]}:\ A $\Rightarrow$
A}}{\footnotesize\par}

A type \texttt{\textcolor{blue}{\footnotesize{}T}} is a \textbf{monoid}
if there exist functions

\texttt{\textcolor{blue}{\footnotesize{}def empty{[}T{]}: T}}{\footnotesize\par}

\texttt{\textcolor{blue}{\footnotesize{}def combine{[}T{]}(x: T, y: T): T}}{\footnotesize\par}

such that the usual algebraic laws hold:

\texttt{\textcolor{blue}{\footnotesize{}combine}} is associative

\texttt{\textcolor{black}{\footnotesize{}$\forall x:\text{combine}(\text{empty},x)=\text{combine}(x,\text{empty})=x$ }}{\footnotesize\par}

Monoids are an abstraction for any sort of data aggregation

See example code for implementing the \texttt{\textcolor{blue}{\footnotesize{}Monoid}}
type class:

by using a case class as a PTTF (instance from scratch)

by assuming \texttt{\textcolor{blue}{\footnotesize{}Pointed}} and
\texttt{\textcolor{blue}{\footnotesize{}Semigroup}} (``derived''
instance)


\subsection{Examples of type classes II}

Higher-order PTFs

A type constructor $F^{A}$ is a functor if it has a \texttt{\textcolor{blue}{\footnotesize{}map}}
operation

or, equivalently, \texttt{\textcolor{blue}{\footnotesize{}fmap}} 

that satisfies the functor laws (identity law, composition law)

We would like to write a generic function that tests the functor laws

\texttt{\textcolor{blue}{\footnotesize{}def checkFunctorLaws{[}F{[}\_{]},
A, B, C{]}(): Assertion = ???}}{\footnotesize\par}

Need to get access to the function \texttt{\textcolor{blue}{\footnotesize{}map}}
defined for the given \texttt{\textcolor{blue}{\footnotesize{}F}}{\footnotesize\par}

We treat \texttt{\textcolor{blue}{\footnotesize{}map}} as a PTVF whose
type domain is all functors \texttt{\textcolor{blue}{\footnotesize{}F}}:

\texttt{\textcolor{blue}{\footnotesize{}def map{[}F{[}\_{]}, A, B{]}(fa: F{[}A{]},
f: A $\Rightarrow$ B): F{[}B{]}}}{\footnotesize\par}

We constrain \texttt{\textcolor{blue}{\footnotesize{}F}} to belong
to the \texttt{\textcolor{blue}{\footnotesize{}Functor}} type class

by adding \texttt{\textcolor{blue}{\footnotesize{}implicit ev:\ Functor{[}F{]}}}
as extra argument to \texttt{\textcolor{blue}{\footnotesize{}map}}{\footnotesize\par}

note: \texttt{\textcolor{blue}{\footnotesize{}Functor}} is a \emph{higher-order}
PTTF \textendash{} its type argument is \texttt{\textcolor{blue}{\footnotesize{}F{[}\_{]}}}{\footnotesize\par}

See test code for implementation and functor laws


\subsection{Overview: Types and kinds}

Compare value-to-value functions (VVFs) vs.\ type-to-value functions:

the \textbf{domain} of a VVF is the set of admissible argument \emph{values}

a ``value domain'' (subset of values) is called a \textbf{type}

the VVF is applied safely only to argument values of the right \textbf{type}

\texttt{\textcolor{blue}{\footnotesize{}def f(x: Option{[}Int{]})
= ...; f(y);}}{\footnotesize\par}

the \textbf{type domain} of a PTVF is the set of admissible argument
\emph{types}

a ``type domain'' (subset of types) is called a \textbf{kind}

the PTVF is applied safely only to type arguments of the right \textbf{kind}

\texttt{\textcolor{blue}{\footnotesize{}def f{[}T: MyTypeClass{]}(args...)
= ...; f{[}A{]}(args);}}{\footnotesize\par}

In both cases, the function call's safety is guaranteed \emph{at compile
time}

Kinds are the ``type system for types''

a type class \texttt{\textcolor{blue}{\footnotesize{}MyTypeClass}}
defines a new kind (as a subset of types)

suggested \textbf{kind} notation:{\footnotesize{} $(*:\text{MyTypeClass})$}{\footnotesize\par}

another existing kind is the \textbf{type function} kind (notation:
$*\rightarrow*$)

in \texttt{\textcolor{blue}{\footnotesize{}F{[}T{]}}}, the \texttt{\textcolor{blue}{\footnotesize{}F}}
and the \texttt{\textcolor{blue}{\footnotesize{}T}} are types of different
\textbf{kinds} ($*\rightarrow*$ and $*$ resp.)

define \texttt{\textcolor{blue}{\footnotesize{}type Ap{[}F{[}\_{]},
T{]} = F{[}T{]}}}, then wrong kinds will fail in \texttt{\textcolor{blue}{\footnotesize{}Ap{[}A,
B{]}}}{\footnotesize\par}

suggested \textbf{kind} notation:{\footnotesize{} $\text{Ap}:(*\rightarrow*,\,*)\rightarrow*$
}(``higher-kinded type'')

See test code


\subsection{Scala's ``implicit method'' syntax for PTVFs}

Two sorts of available syntax for Scala functions:

as in ordinary math: \texttt{\textcolor{blue}{\footnotesize{}func(x,
y)}} or \texttt{\textcolor{blue}{\footnotesize{}func(x, y)(z)}} etc.

as ``method'': \texttt{\textcolor{blue}{\footnotesize{}x.func(y)}}
or equivalently \texttt{\textcolor{blue}{\footnotesize{}x func y}}
\ 

this is similar to \texttt{\textcolor{blue}{\footnotesize{}func(x)(y)}}
but is implemented differently

It is often convenient to use functions syntactically as methods:

\texttt{\textcolor{blue}{\footnotesize{}def +++{[}T: HasPlusPlusPlus{]}(t: T,
arg: ...) = ...}}{\footnotesize\par}

\texttt{\textcolor{blue}{\footnotesize{}val t: T = ...}}{\footnotesize\par}

\texttt{\textcolor{blue}{\footnotesize{}+++(t, arg) }}\textcolor{gray}{\footnotesize{}//
that's how we have to call this function}{\footnotesize\par}

\textcolor{gray}{\footnotesize{}// but instead we want to be able
to write t +++ arg}{\footnotesize\par}

Implementing the ``implicit method syntax'' for a PTVF \texttt{\textcolor{blue}{\footnotesize{}func}}:

declare \texttt{\textcolor{blue}{\footnotesize{}func}} as a method
on a new trait or class, say \texttt{\textcolor{blue}{\footnotesize{}MyTCSyntax{[}T{]}}}{\footnotesize\par}

declare an \emph{implicit conversion }function from \texttt{\textcolor{blue}{\footnotesize{}T}}
to \texttt{\textcolor{blue}{\footnotesize{}MyTCSyntax{[}T{]}}}{\footnotesize\par}

to make the code shorter, use an \texttt{\textcolor{blue}{\footnotesize{}implicit
class}}{\footnotesize\par}

see example code

What we gained:

the PTVF appears as a method \emph{only} on values of the relevant
types

the new syntax is defined automatically on \emph{all} the relevant
types \texttt{\textcolor{blue}{\footnotesize{}T}}{\footnotesize\par}


\subsection{Worked examples}

Define a PTVF \texttt{\textcolor{blue}{\footnotesize{}def bitsize{[}T{]}
= ...}} such that \texttt{\textcolor{blue}{\footnotesize{}bitsize{[}Int{]}}}
returns $32$ and \texttt{\textcolor{blue}{\footnotesize{}bitsize{[}Long{]}}}
returns $64$; otherwise \texttt{\textcolor{blue}{\footnotesize{}bitsize{[}T{]}}}
is undefined

Define a monoid instance for the type $1+\left(\text{String}\Rightarrow\text{String}\right)$

Assuming that $A$ and $B$ are monoids, define monoid instance for
$A\times B$

Show: If $A$ is a monoid and $B$ is a semigroup then $A+B$ is a
monoid

Define a functor instance for \texttt{\textcolor{blue}{\footnotesize{}type
F{[}T{]} = Seq{[}Try{[}T{]}{]}}}{\footnotesize\par}

Define a Cats' \texttt{\textcolor{blue}{\footnotesize{}Bifunctor}}
instance for $Q^{X,Y}\equiv X+X\times Y$

Define a \texttt{\textcolor{blue}{\footnotesize{}ContraFunctor}} type
class having \texttt{\textcolor{blue}{\footnotesize{}contrafmap}}:

\texttt{\textcolor{blue}{\footnotesize{}def contrafmap{[}A, B{]}(f: B
$\Rightarrow$ A): C{[}A{]} $\Rightarrow$ C{[}B{]}}}{\footnotesize\par}

Define a \texttt{\textcolor{blue}{\footnotesize{}ContraFunctor}} instance
for type constructor $C^{A}\equiv A\Rightarrow\text{Int}$

Define functor instance for recursive type $Q^{A}\equiv\left(\text{Int}\Rightarrow A\right)+\text{Int}+Q^{A}$

{*} If $F^{A}$ and $G^{A}$ are functors, define functor instance
for $F^{A}+G^{A}$


\subsection{Exercises}

Define a PTVF \texttt{\textcolor{blue}{\footnotesize{}def isLong{[}T{]}:\ Boolean}}
that returns \texttt{\textcolor{blue}{\footnotesize{}true}} for \texttt{\textcolor{blue}{\footnotesize{}Long}}
and \texttt{\textcolor{blue}{\footnotesize{}Double}}; returns \texttt{\textcolor{blue}{\footnotesize{}false}}
for \texttt{\textcolor{blue}{\footnotesize{}Int}}, \texttt{\textcolor{blue}{\footnotesize{}Short}},
and \texttt{\textcolor{blue}{\footnotesize{}Float}}; otherwise undefined

Define a monoid instance for the type $\text{String}\times(1+\text{Int})$

If $A$ is a monoid and $R$ any type, define monoid instance for
$R\Rightarrow A$

Show: If \texttt{\textcolor{blue}{\footnotesize{}S}} is a semigroup
then \texttt{\textcolor{blue}{\footnotesize{}Option{[}S{]}}} is a
monoid

Define a functor instance for \texttt{\textcolor{blue}{\footnotesize{}type
F{[}T{]} = Future{[}Seq{[}T{]}{]}}}{\footnotesize\par}

Define a Cats' \texttt{\textcolor{blue}{\footnotesize{}Bifunctor}}
instance for $B^{X,Y}\equiv\left(\text{Int}\Rightarrow X\right)+Y\times Y$

Define a \texttt{\textcolor{blue}{\footnotesize{}ProFunctor}} type
class having \texttt{\textcolor{blue}{\footnotesize{}dimap}}:

\texttt{\textcolor{blue}{\footnotesize{}def dimap{[}A, B{]}(f: A $\Rightarrow$
B, g: B $\Rightarrow$ A): F{[}A{]} $\Rightarrow$ F{[}B{]}}}{\footnotesize\par}

Define a \texttt{\textcolor{blue}{\footnotesize{}ProFunctor}} instance
for $P^{A}\equiv A\Rightarrow\left(\text{Int}\times A\right)$

Define a functor instance for recursive type $Q^{A}\equiv\text{String}+A\times Q^{A}$

{*} If $F^{A}$ and $G^{A}$ are functors, define functor instance
for $F^{A}\times G^{A}$ 

{*} Define a functor instance for $F^{A}\Rightarrow G^{A}$ where
$F^{A}$ is a contrafunctor (use Cats' \texttt{\textcolor{blue}{\footnotesize{}Contravariant}}
type class for $F^{A}$) and $G^{A}$ is a functor


\subsection{Further directions}

What problems can we solve now?

Define arbitrary PTTFs and use them to define type classes (PTVFs) 

Define them together or separately, combine them at will

Use the Cats library to define instances for standard type classes

Derive type class instances automatically from previous ones

Reason about higher-order type functions, types, and kinds as necessary

What problems cannot be solved with these tools?

Automatically derive type class instances for polynomial data types

see \href{https://github.com/underscoreio/shapeless-guide}{The guide to ``shapeless''},
chapter 3

Derive a recursive type generically from an arbitrary type function

Given a type function \texttt{\textcolor{blue}{\footnotesize{}F{[}\_{]}}},
define a recursive type \texttt{\textcolor{blue}{\footnotesize{}R}}
via \texttt{\textcolor{blue}{\footnotesize{}R = F{[}R{]}}}{\footnotesize\par}

This \texttt{\textcolor{blue}{\footnotesize{}R}} will be a function
of \texttt{\textcolor{blue}{\footnotesize{}F}}; denote that type function
by \texttt{\textcolor{blue}{\footnotesize{}Y{[}F{[}\_{]}{]}}}{\footnotesize\par}

This \texttt{\textcolor{blue}{\footnotesize{}Y}} must be defined by
a type equation like this,

\texttt{\textcolor{blue}{\footnotesize{}type Y{[}F{[}\_{]}{]} = F{[}Y{[}F{]}{]}}}\textcolor{gray}{\footnotesize{}
// does not compile (``cyclic type'')}{\footnotesize\par}

Automatically derive type class instances for such recursive types

That requires type-level recursion (type-level fixpoints), see \href{https://github.com/slamdata/matryoshka}{matryoshka}

This and other advanced topics are found in \href{https://apocalisp.wordpress.com/2010/06/08/type-level-programming-in-scala/}{this blog post from 2010}

\section{Combining typeclasses}

\section{Inheritance}

\section{Functional dependencies}

\section{Discussion}

\begin{comment}
this tutorial is about typeclasses and type level functions to motivate
why we want to talk about this let's consider what happens if we would
like to implement the sum function generically so that the same implementation
code will work for sequence of integers for sequence of doubles and
so on if we try to do that we find that it doesn't quite work we cannot
generalize the sum function like this with a type parameter T and
an argument of type sequence of T because there is no way for us to
sum or to add values of type T where T is unknown it's arbitrary unknown
type obviously the sum function can only work for types T that in
some sense or summable another very similar situation happens if we
wanted for instance to define the F map function for factors that
already define the map function as we know F map is equivalent to
map so for each factor somebody defined already the map function we
would like to define F map for all of them at once by in the same
generic code but we cannot generalize F map to arbitrary type constructors
F here is what we would have to write suppose that we tried we would
have to put F as a type parameter F being the type constructor of
the function just as an aside in Scala this syntax is necessary if
you want to put a type parameter that is itself a type construction
you can do that just by using this syntax so if we try to write F
map like this we will have F me up with type parameters and we cannot
write this code because F in here is an arbitrary type constructor
parameter and there's no way for us to get the map function for that
F we don't even know if it exists just like here there's no way for
us to get the addition operation with some more plus or something
for the type T and just like in that case the F map could work only
for certain type constructors F namely for those that are factors
so our desire to write code more generically leads us to the need
to define functions whose type arguments for example the T and F are
not just arbitrary types but there are types that are required to
have certain properties or to belong to a certain subset of possible
types here for instance it will be integer double long and so on maybe
others string could be considered as well and here will be the filter
types or rather to be precise the factor type constructors that should
so somehow we should require require that this type parameter in the
F map implementation is not just any type but it is a sub is belongs
to a subset a certain subset of possible types which we also have
to define somehow so if we are able to do that which is the whole
focus of this tutorial if we're able to do that then we can use known
properties of this type parameters in implementation of these functions
also we would like to be able to add new supported types whenever
it's needed so integer double maybe a library will define this property
for the type T but we want to add new ones later so wanted to be extensible
so this is our goal now the first step towards solving it and we will
be solving this systematically the first step is to realize that this
is similar to the concept of partial functions except that applied
to types let me then look at what partial functions are generally
functions can be total or partial the total function always has a
result for any argument value the partial function sometimes has no
result for some argument values it does have a result for others it
does not an example of a total function would be an integer function
taking an integer argument with returning that integer multiplied
by 3 so for any input integer you always have an output integer the
example of a partial function will be real valued square root it has
no result for negative arguments no real valued square root from negative
arguments but it does have a result for non-negative arguments so
that's a partial function partial function has a domain of its definition
which is smaller than the all values of that type now we should also
consider that here we have type parameters so what does it mean that
we have a function with type parameters it's like a function with
argument is a type so let's consider functions both from value and
from type to values in two types and that is the most most general
kind of situation so we will have this table of all possible functions
function from value or from type and to value and to type so the function
fill value to value were value level function or value to value function
that's just an ordinary function like that it takes a value argument
of type integer let's say and returns a value of type integer so that's
a value to value function a function from type to value could be visualized
like this so it has a type parameter and the result is a value of
this type now this type depends on the type parameter so first you
have to give it the type parameter then it knows what type it value
needs to return and then it creates somehow that value and returns
it now one important observation here is that these functions are
the same kind the same sort this function has a type parameter and
then you can rewrite this in a different syntax by writing some column
sequence t rlt which will be quite similar to this just a different
syntax so just like in the example here first you have to get the
type parameter T then you know which types you're taking in this function
so the function some can be considered as a function film type from
t2 value which is a function from sequence T 2 T so this is a value
which is itself a function on a function at the value level but that's
a value right so functions at value level or values sequence T 2 T
where T is already fixed fixed type this type parameter finally we
also have functions from type 2 type for example if we define this
type then my data is a function from type 2 type so later in the code
you could say my data of string and that would be the same as evaluating
this function substituting string instead of a and the result would
be either of int and string and there also is a function from value
to type in principle now in Scala it would be quite hard to come up
with useful examples of this sort because these are actually dependent
types so this is this kind of function is called a dependent type
which means that it's a type that depends on a value and that is not
very well supported in Scala or in Haskell for that matter so there
are more advanced experimental aim which is like a dress or Agda not
have better support for dependent types so we will not talk about
dependent types here our concern will be mostly this part of the table
functions from type either from type to value or from type 2 type
now consider how these functions are being evaluated the value to
value functions are evaluated at runtime type 2 value functions are
evaluated at compile time because you cannot really call these functions
without specifying a type parameter and that is evaluated at compile
time the compiler will know that you are calling this function with
string here say instead of a and that would be known at compile time
as you write your code one caveat here is that if you use type casts
in the code then you could make a run time evaluation based on types
so type T value functions can become runtime which is a problem because
it is possible to have a bug and a bug in a type 2 value or type 2
type function at compile time that's okay because you you will find
it before you run your program at runtime it's not okay you already
deployed your program it's running and three months later it crashes
because of some bug that's not a good outcome at all so we will avoid
using type casts or any kind of runtime computation with types we
want to keep all type computations compile time in this way they are
safe consider now partial functions so these so far we just considered
any kind of functions here of all kinds value to value value to type
type to a value type to type consider now partial functions so partial
value to value function is something like this it's a partial function
that takes an argument if the argument here is an option that is not
empty when it returns the value inside this option but if the argument
of this function were the empty option which is to say the none value
then there is no case to match that so this function cannot process
that argument only so this function considered as a functional options
is a partial function if you apply this to the room kind of argument
you get an exception at runtime consider now functions from types
of partial type the value function is a function that takes a type
argument and it only works for certain types for certainty types T
in the view call this function so this is an example of this would
be this function which I will talk about it later but basically the
idea is that it's a function that has a type parameter it returns
a value and when you call this function with the wrong type parameter
it's an error at compile time otherwise it works so that's a partial
function at type two value level so partial type two value function
is what we would like to have like to have those functions and the
goal of this tutorial is to explain systematically how to create and
manage partial type two value functions typeclasses is the mechanism
for doing that and the idea roughly speaking is that the type T should
belong to a certain typeclass or a certain subset of types or a certain
type of domain there are certain kind of types and then you can apply
that partial type to value function to that type if we are able to
implement this then these problems would be solved we will just say
well some is not just total type the value function is a partial type
the value function it's only defined on types T that belong to the
correct kind or typeclass which are summable in some sense and of
course this code it won't work it has to be done honest it's like
a different way which we will find out how to do in a systematic manner
before we do that let's look at an example of using value level partial
functions and let's see what's the issue there so imagine we have
a situation we have a sequence of either end boolean we want to find
the {[}Music{]} elements in that sequence that are in the left part
of the disjunction having an integer and we just get one to get a
sequence of those integers out of this sequence how do we do that
well we just say take in a sequence we filter those that are on the
left which is like this maybe and then we map with this partial function
now after this step the result is still of type sequence of either
so now we are applying this partial function to arguments of type
either which is unsafe what about with a right element and there will
be nothing to match well actually we know in this case it's okay to
apply this partial function because we just filtered all the elements
to be on the left side of the disjunction so we know it's safe however
the types don't show that it is safe the compile-time checking can
tell you maybe that there is some problem a warning but certain cases
will not be matched but it doesn't know it's safe and so if you refactor
this code in some way but say you put this part in one module in this
part in another module and then different people start to modify this
and eventually this condition has changed and it has more complicated
conditions whatever features we need to implement and finally it's
not always the left and so the other part of the code is also changed
in some way and you get a runtime error after some refactoring sometime
later the type side type safe version of this code does not use filter
milk and sort of filter map you do collect so collect is basically
a safe version of filtering only those that fit into this partial
function and then applying the partial function so partial functions
in Scala are special a special type and this type has a method to
decide whether the partial function can apply to us and give an argument
and so that's what we collect is doing and the result is safe it is
of type sequence int and there's nothing you can refactor here to
break the safety of this code so partial functions are safe but only
in certain places where you take care to encapsulate the possible
breakage so usually we make functions total we either add more code
to handle other cases or we use more restrictive types so that we
know that this is not just an either it's really always left and in
other cases for instance there types such as non-empty list or positive
number or or even this this is a subtype of option and if you have
this than you you know it's not empty so instead of using option you
use this in a specific function or you use known empty list and sort
of just list and if you do that you can do it dot head on a non empty
list and it's always safe and the safety is checked at compile time
you cannot pass here instead of non empty list just arbitrary list
so that's how we handle the unsafe team partial functions for value
level functions so for type level functions we would like to restrict
the type parameters to some restrictive subset of types so that this
is safe that's our motivation so let's see how this can be done here
is the first example of a partial type level function consider this
data type so this is a disjunction the disjunction is paralysed by
a type parameter a and has two cases in the one case gives you a my
type constructor of int and the other case gives you my type constructor
of string well this is just an artificial example but this could occur
in some applications now notice we have defined the type constructor
my TC my type constructor as if it's defined for any parameter a as
any type a actually in code you can only ever create values either
of my TC event or of type my DCF string you will never be able to
define any values of any other type like my TC of bullying or mighty
sea or anything else only either string or integer there is no other
case and everything is sealed and final and so there's no way to add
any more cases let's see how a code looks like for this so here's
the code if I wanted to pretend that I have a value of type my TC
of a for arbitrary a I cannot do that I have to create it instantly
the only way to credit is to use case one or case two neither one
would work so none of them type checks action for example if I did
this I mean it may get no no no way I can do anything here I could
say no this one is my t siient so the only thing that can work here
is this but I cannot put here a equal to int either I have a parameter
here or not so you can get rid of this parameter there's no way I
can use this mighty sea with an arbitrary time frame type round you
can get a value now I can define types such as this one which looks
like I'm applying the type level function to int and get a type so
you see I using this type constructor as as if it were a function
of types I apply this to type into my data type t1 up out of that
and this works and then I can use t1 and t2 for example in my code
and it's all compiles when it works so this is to illustrate but the
type constructor like my TC is quite similar to a function at type
level since a functional takes types as arguments and returns types
just that I have to say type instead of well because it's a type level
function not a value level function otherwise it's very similar to
a function so that's why I keep calling this a type 2 type function
or type the value function because I'm I want to think about these
type constructors in a general way I don't want to think about them
in some kind of special magic fashion these are just functions their
arguments are types the results are types and now I can use my mathematical
intuition about functions to reason about them now here's a way to
use this in code so for example x2 is of this type I can match this
and I get a result so it works now what if I wanted to apply this
function to a different argument so I am applying it to doing now
there is no way to create a value of type boolean but the Scala compiler
doesn't know this so in some sense I consider this function as a function
that is only defined so it's a type 2 type function my TC that's only
defined for int or string as type arguments but if I write my TC of
boolean and I evaluate this it compiles so the compiler doesn't know
doesn't check but there is no way to create this value but actually
no code would ever compile it creates this value correctly like this
for example this won't compile now what I can do is I can force the
type like this using this construction as instance of which breaks
type checking its runtime typecast and it's completely breaking the
type checking there is nothing the compiler will check here the results
are going to be wrong like this is going to be wrong I can't really
use this in any useful way in other words it doesn't help if you break
the type checking type checking is your friend this is to be used
only in very rare situations when the type system is not strong enough
not powerful enough to do some extremely complicated things and in
most cases it's not necessary so this is cheating in terms of type
checking and even this cheating pretend that I created a value of
type t3 but actually I have not created such a value I cannot use
it in the code still there's nothing I can do with this in code that
would be compiled invented working so this motivates me to consider
that a function my TC is actually a partial type function it is meaningless
to apply this function to arguments such as doing the type argument
such as bullying it's only meaningful to apply to arguments integer
and string so in other words this type function has a domain type
domain it is set of types to which it can be applied meaningfully
and this domain contains only two types int and string so become the
compiler as we have just seen does not enforce this it does not flag
it immediately as an error however it does enforce that the values
of this type only exist for the type parameter within the correct
domain or the type domain and this type domain is defined at compile
time and is what I mean by the domain of the partial type 2 type function
just a side note here the type constructor my TC cannot possibly be
a factor since it's not defined for all type arguments a factor must
be a total type 2 type function partial type 2 type functions cannot
be factors so these kind of types where you have partial type 2 type
functions and case classes they're called generalized algebraic data
types or generalized means that these types are not the type parameter
a there are just specific types so there are some cases when they're
useful they're useful for domain modeling of certain kinds like for
example you have some kind of query and the result types can be a
fixed set of different result types then it's useful to have all these
possible result types modeled directly via cases in the disjunction
like this for domain-specific languages representing some kind of
type expressions in some way then it's also useful these are specific
problems where people have used GE T's in a useful way so this is
our first example of a partial type function another way of doing
this is to have a trait with code now here this is just a data type
there is no code in here so let's see how that works when you have
trade with code so that's more the object-oriented way because it
uses method overriding so we start with the trait that has a method
the trait has a type parameter as well and there are some case classes
so it's very similar to the previous slide except the trade has code
the DEF method the inside and the case class has override that def
method other than that it's good could be case class just like in
the previous slide and again you see this class extends heads plus
with integer time from type parameter and not with an arbitrary type
parameter and this extends with a string not with arbitrary type parameter
so just as in the previous slide this defines a partial type 2 type
function and now this is a function that's only defined for a certain
a for a either int or string so this code is quite similar to having
defined a function plus with a type parameter a having this type signature
except that a must be from this type domain it's not quite the same
as having defined such a function directly because plus doesn't have
a type parameter and we can only access it by first creating a value
of this type and I'm doing that value dot plus so because it's a def
method there must be the syntax value dot plus another limitation
is that all the functions support so we see this plus is one of the
partial type to value functions it's a type to value function which
is defined only for certain types all of these must be defined up
front in the code of this trait so if I want plus minus times or divide
or whatever I have to define all of those as def methods up front
in this train I cannot later defined further partial type two value
functions here and also I cannot use this partial type two value function
in a different one defined later so that is quite a significant limitation
so this mechanism kind of works for implementing a partial type two
value function but it is limited so the object-oriented way it gives
us some leverage but it's it's limited we will be using this mechanism
in many cases but we need to have a more general mechanism for implementing
partial type two value functions so what is that mechanism to understand
this mechanism remember these values of type has plus of a here only
be of two different types as plus event or as possib stream there
are no other values that your code could possibly define so let's
use that fact now there are only certain values that your code can
define and require these values to exist and in this way we will define
a partial type two value function so here's how it works suppose we
want to define a function func with type argument a and some whatever
value arguments and we want to define this only for certain types
a first we define a partial type two type function that is defined
only for these types a so we do it in the same way here there's really
only one way in Scala of doing this make a trait and make some classes
that extended with specific type parameters like this or like that
that's it's the same thing it just there's only one way of doing this
in Scala so it first to create this partial type two type function
which defines your type domain its defined only for the specific types
you want let's call this for now is good so it is this type parameter
good and if so then we can proceed with it we create some specific
values of these types for all relevant types a would create a specific
value so this would be to say we create a value of this type you have
to create a value of this type just you know Val a equals this without
B equals that and that's what we will do now we will add an extra
argument to this function func in addition to all the arguments it
already has or needs we add another argument which we will call the
type evidence and this argument will be of type is good of a so this
function cannot be called unless you have a value of this type and
you can only have values of this type if the type is supported by
the partial type 2 type function in other words if the type is from
the correct type domain that you are defining it becomes now impossible
to call this function with an unsupported type parameter a because
you will never be able to produce this value type evidence here you
you can only produce values of these two types even if you define
the type for a different type argument you can produce the value of
that type so you could you could define a type of my TC of booing
but you will never be able to produce a value of that type and calling
func requires a value of that type so because we added an extra argument
to funk so the type evidence is that value that is required now with
a new argument that we have it and that guarantees we will always
call func value with correct types a trying to call with wrong types
will failed compile-time you can put some type evidence value for
the wrong type at the wrong type check so that's a compile-time guarantee
now you can make the trait not sealed here it was sealed because we
were just thinking in terms of disjunctions but this trade doesn't
have to be a disjunction it's just a type constructor it's just a
type two type function and it doesn't have to be sealed these don't
have to be final case classes they could be actually traits themselves
or they could be objects or the case objects or they could be just
classes not case it could be anything it doesn't have to be a disjunction
we're not using this as a disjunction really we're using this as a
partial type to type function so the only thing we want to define
is a type and some values of that type with specific type parameters
how we do this is implementation detail one easy way of doing it is
by using traits and classes that extend the trait we will see in the
sample code another way of doing this or several other ways of doing
this that might or might not be convenient for different situations
let's implementation detail the important thing is if we make this
a tree that's not sealed we can add new values for new types later
so in in the libraries say we provide this with some standard types
in an application we want more types so we just add more values there's
no need to modify the library code so that's extensible new supported
types can be added in user code and that's a great advantage so that
actually is the general solution of the problem of defining a partial
type two value function now the solution has its cost here is what
we have to do now first all calls to this function will now become
calls of this kind we have one more argument for each call this tantrum
you have to put one more argument now this value this evidence or
type evidence is the value that we need to create her each supported
type a so if you have 15 different types that we want to use need
to create 15 different values a lot of code and that code is probably
going to be very straightforward just automatic kind of code which
software engineers call boilerplate boring that is still missed it
needs to be written and finally all these evidence values have to
be passed around all the time because you call these functions in
different places in your code have to get these these type evidence
values you created created these values in one place but you have
to pass them all around your code whenever they're needed so that's
a lot of extra work for if you want to use these functions many times
them in different places of your code that would be a lot of work
in Scala these issues can be mitigated and mechanism for mitigating
these issues is by using implicit values which we will look at very
soon then we look at specific code the result of using implicit values
is that type evidence arguments are only explicitly mentioned at the
declaration side of the function func and when you call it you don't
write them all so you don't have to pass them around they are already
passed around invisibly once they are defined as input set these type
evidence values and also you don't have to write a lot of code to
define type evidence values for all kinds of types because you can
define rules by which new implicit values can be built up automatically
from previously built up implicit values and the compiler will do
that recursively as much as necessary so that's more or less solves
all these three issues however I would like to emphasize that these
are cosmetic issues which make make code better and easier to write
they do not change the character of the code in a qualitative way
these gains will stand whether we use this or not so we can use this
mechanism the partial type 2 type functions so in languages that did
not support implicit values we can even use this in Java these things
are just cosmetic issues with their positi of the code and it's not
changing the fact that we have implemented partial type 2 value functions
in the most general way possible so what is the scholars mechanism
of implicit values so here's how they work you declare a value of
some type as implicit and also there are implicit def methods and
implicit classes that you can define once you have defined it like
this you can start defining functions with implicit arguments so like
this for example and then once you have defined a value of this type
as implicit you can call this function by you're saying F of args
you don't have to pass X into it that would be found so the compiler
at compile time will search in the local scope for this definition
or in imports or in companion objects or in parent classes and it
will find a definition like this and it will substitute that value
silently and invisibly into your function so by the way having more
than two implicit values of the same type is a compile time here so
you cannot say implicit well X some type and possess well Y some type
next to it that would be a compile time here in addition to this Scala
has a short syntax for declaring type evidence arguments which is
which is specifically very useful for declaring partial type to value
functions and the syntax looks like this so it looks like you're saying
that type parameter a is itself of some type well it's like a type
4 type notation but actually it's completely equivalent to this code
take all these arguments that you have and add another list of arguments
which are all implicit and they will have automatically generated
names and the types will be this type constructor of a this type constructor
will be and so on so this code is just shorter easier to read and
this is longer but in exactly the same is less it it just rewritten
so you could have some arguments implicit written out like this some
other arguments written out like that we can combine both they will
be already written to one big list of implicit arguments at the end
so the difference between these two is that if you want to use the
evidence argument in the code of the function you need its name and
in this definition you have the name you define it yourself in this
definition you don't see that name so that name would be automatically
generated you don't see it don't matter another problem there's a
special method defined in the standard library which covers a scope
which is called implicitly and that method can grab the implicit argument
defined for you when they define functions of the syntax the definition
of implicitly is very simple and you could have defined it yourself
it's just defined for convenience for you in the standard library
we'll see how that works in one of the worst examples later so these
things make the code shorter we still need to declare the trait or
the type constructor as a partial type two type function and we still
need to create type of evidence values I am declaring them as implicit
but passing them around is very simple and calling the function is
much shorter because of the implicit mechanism so now having done
all this to let me say what typeclasses are so right now we were talking
about partial type two value functions and partial type two type functions
but this is not really the terminology that most people use it is
just the terminology that I find the most illuminating it tells you
exactly what we're doing but actually people say typeclass what sorts
a tight class a typeclass is basically a set of partial type two value
functions that all have the same time dummy which means that for example
you can have plus minus times divide let's say and they all have the
same type domain which is numbers that can be added subtracted divided
in something so instead of saying typeclass we could say some partial
type two value functions that somebody has defined that's the same
thing so in terms of specific code that needs to be written the typeclass
is two things first it's a partial type two type function so it's
a type constructor with some code that creates specific type evidence
values of specific types T and that defines the type domain second
it is the code for the desired PT lives so whatever functions you
need to define your write code for them and you use that PT TF in
that code to define the type domain also for many important use cases
these functions must satisfy certain laws certain equations for mathematical
reasons otherwise it would be unusable or programs will have difficult
bugs so that's what typeclasses are that's a definition of a tag class
let's fix some terminology people say that a type T belongs to the
typeclass my tag class what does it mean it means that there is a
PTT F partial type two type function called the called this and this
partial type of tag function has a domain to which type T belongs
so for example integer could be T and then integer belongs to this
if that type domain includes the integer which means that there is
a type value sorry type evidence value of type MyType plus int so
if some value of this type can be found then the type belongs to the
type cost that's what what it means when people say that now a function
with type parameter requires for this type parameter the typeclass
might a class what does it mean well it means that actually one of
the arguments of this function usually implicit argument is the value
of a type like this which is constructed using a partial type two
type function so if you if you see that there is some partial time
to time function lying around then you know that this is a partial
type 2 value function that requires F typeclass constraint or it constrains
a type parameter to belong to the typeclass otherwise it cannot be
called with that type now another important terminology is that the
evidence value the type evidence value which is this extra argument
of this type that value itself is is called not only type evidence
is also called the typeclass instance now that terminology is completely
equivalent to type evidence so I prefer evidence but instance is also
standard terminology I will explain how how to create these things
in many examples now so typically in Scala you create a trait with
a type parameter like this then you create code that makes values
of this type for various types T all these values are declared as
implicit and somehow you make them available for user code either
while imports or put them into companion objects for the types T which
is quite convenient because then you don't need any imports then you
also write some functions not have implicit arguments of this type
usually these metals are def methods these functions are def methods
in this trait we don't have to be but often they are because that's
sometimes necessary and often convenient so as a rule a type evidence
value should carry all the information they need about the typed you
will see why let's look at the examples so semigroup is a typeclass
that actually let me show you what it is a typeclass that has an associative
binary operation so in other words there must be this function which
is a partial type 2 value function it takes two arguments of type
T and returns a value of type T so it's a any kind of combined operation
so combine two elements and get another element of the same type combine
two values in any way as whatsoever as long as it's associative so
it doesn't have to be commutative doesn't have to be any inverse operation
to this just can combine two values into one larger value in some
way doesn't even have to be larger value we don't know anything about
how it works so that's semigroup it's a very weak very bare-bones
kind of operation binary operation it just is associative that's the
only thing we know so let's implement that operation is a partial
type to value function first step is to implement PDGF so let's just
do that semi group with parameter T then we have two K subjects let's
say well we have semi group and semi group string that's all nothing
else fine so that actually defines PTT F that defines a type constructor
of which you can only have value semigroup event and semigroup of
string so that's sufficient in principle to create our partial target
value function so let's see how we can do this so we can define up
here the type parameter T it has an x and y as required it returns
the t as required it also has an implicit argument which is the type
evidence of type semigroup t so now this is it this is the partial
type T value function you could not call this function with a type
parameter T for which you cannot have cannot have a value of semigroup
T so you could not call this with type parameter T other than int
instance so this test for example says you can combine ends you can
combine strings but option event cannot be combined this doesn't even
compile because the type option int is not within the type domain
of this partial type function and so you cannot have any type evidence
for it and so it just won't compile so that's great now how do we
implement this function well we have to add somehow integers or strings
but how do you know which is which so let's see this evidence is the
argument so the only thing we know about type T is this evidence value
so we can imagine it where two cases so it could be an int or it could
be a string evidence so if it's an int evidence we know that T is
actually int so well what can we do well the only thing we can do
with the force X to be event Y to be event add them together and then
force their is out to be again of T because we're supposed to return
T and that's what we have to do and we have to do we can do the same
for string and actually this code works as the test shows the test
passes it's a very unsafe code is not great not not not great at all
because of these as instance of operations which are unsafe and maintaining
this code is very hard and so don't do that we will never do this
I'm just showing you an example of how we could do it in some way
and that actually is an implementation of a typeclass so semigroup
is a typeclass up rather op is a type course is a partial type to
value function with a type domain we defined and it works for that
type domain that's that's the typeclass a semigroup we have implemented
seven rules on very ugly but we'll do it much better in just a few
minutes this is our very first example note we derived the requirements
for this completely systematical there is no there's no guessing so
we must have a type dummy we must have a type function type constructor
and we must have the implicit evidence argument well implicit is just
convenient so we don't have to write this argument here in principle
we could so the OP function if you look at it it has the second of
arguments which is this one their evidence and we could put this evidence
here in parentheses for example sending group into evidence it's exactly
the same so this is how the function actually must be called and the
compiler writes that for us we don't have to write that in our code
so that just makes our cold sugar other than that that's what's going
on this all right now why is it that we had all this trouble it's
because we have no idea what tedious we know that it's either int
or string so in this case it is int in this case it's string but we
don't have any functions that help us to work with that so the way
to solve this is to put more information into the evidence so right
now here evidence is just an empty object it has no information except
the type we put more information into the evidence we can get it out
and work on these x and y's so why don't we just put the implementation
into the evidence instead of putting it here and having all this trouble
with type custom so that's what we do next we again implement semigroup
again for empty string now the partial type to type function in our
case is not going to be a sealed straight it's just going to be a
type function which we define and it's going to carry this data so
actually you see what we need is to combine these to fund these two
elements and that's very natural to put that information directly
onto the type constructor so now it's very easy for us to do find
the evidence values so we just say this function which is adding two
integers has this type and that's correct this function has this type
that's correct so we have defined values of type semi group in the
semi group string so that together with the type function defines
a partial type two type function so now semi group int and semi group
string are defined and others are not defined we put them as implicit
now we define the partial type two value function let's say we define
a sum that's what we wanted there is a sequence of values T and there's
a default value for empty sequences say then there is an implicit
type evidence argument so now the implementation is actually very
easy we just do a fold left with the default element as the initial
value and the fold has a function argument which is this OP an evidence
has exactly the right type it is the hope so we just put it in there
and we're done with no typecasting completely typesafe implementation
and yet this is a partial type 2 value function because it can only
be called on integer instrum and this is a test that shows how it
works so this is a minimum implementation very very bare-bones implementation
of a typeclass which is the sum a semigroup is the type 2 type function
and sum is the type 2 value function we can define further type to
value function functions if we want to use them this type 2 type function
so that's a second attempt at defining a typeclass and the code is
very easy as you see now let's try to use traits instead now this
is a third implementation of second group we put this method which
used to be here we put it as a def mounted on the trade so we override
this then in the two specific implementations for integer and string
evidence so that defines our PTT F and just as in the previous example
all the data it's necessary for implementing what we want is carried
by values of this type so all the data is carried by values of this
straight now we do implicit case object just so that we don't have
to write more implicit Val's in principle what we could do is we could
say for example here is taste object not imported and then we said
well maybe C equals that equals singing group in evidence we could
do this just as well it's just shorter to write implicit case object
right here so we have defined a PTT after now we define a PTV F which
is adding of three numbers let's say and we define it in the same
way so we have arguments at once and then additionally an implicit
evidence argument and then we use the evidence dot up to perform the
computation we see evidence dot what is this op that has been defined
in the trade and since it's a deft method on the trade we have to
use it in the syntax F dot op whereas here we didn't so here we could
define add three C we can define new PTV F very easily so this must
be of type same in Group G so how do we do that we do F F is a function
right so that type is this function so we just do have of X or Y Z
for example random so we can define any number of DVF's externally
user code just like what we wanted so that's exactly how we expect
this to work and here instead of saying just F will you have dot off
every time because here op is a death metal in your trait same things
should work alright so as we have seen it is quite useful if the type
evidence value carries all information but the ETF's need to know
about the type team and in many cases the trait contains these materials
directly in some simple cases it can be a data type not a trait but
actually as we will see a treatable death metals is necessary if we
need higher order type functions it is convenient it's not particularly
inconvenient to heaven it's just more writing to do trade with death
metals if you can avoid it you don't have to do it and as we just
have seen additional partial type 2 value functions using the same
unchanged type 2 type function can be added later so no need to modify
the code of this trade if we want to add new partial type 2 value
functions without changing the type domain so add 3 as an example
we don't change the type deleting some and add 3 are defined for the
same type domain for things that can be added no need to modify the
code of library if we want to add more partial type 2 value functions
with the same type domain and as we'll see later we can combine this
with other partial type functions that we define in other code so
that's exactly what we wanted let's look at some more examples of
typeclasses semigroup we just saw now another interesting typeclass
is pointed it's a very simple time code typeclass it just has a function
point that returns a value of that type so that means somehow there
is a special and naturally selected value of that type and the function
gives you that value examples would be 0 integer type or empty string
so these are naturally selected values of the type or for a function
type it will be the identity fun for this function type that's also
naturally selected value so I mean it's still selected in a sense
that somebody selected it it's just that it's natural to consider
that value it's useful to to have some kind of value of this sort
and maybe it will have interesting properties and so if you have a
value like this then the type is pointed an example of a type that's
not pointed is a function from A to B there's no way to have some
kind of natural function from A to B but from a to a there is you
have the identity function another important typeclass is Mohammed
monoid is a typeclass that represents data aggregation so data aggregation
means you have some aggregator that is initially empty and you put
data into it and it combines everything you put it into it into some
aggregate value so it has two functions empty and combined so empty
is just like the point it just gives you the value of that type and
combined is just like the semigroup takes two values of the type of
returns one additionally though we require that combine is associative
just like in a semigroup and this law must hold so the empty value
combined with anything either to the left or to the right should again
return the same anything that you passed in so combining an empty
with X does not change X that is a requirement so the value of the
empty aggregate is special in the sense that you can combine empty
aggregate with data and that doesn't change that data so monoids are
used a lot in different contexts so one obvious context is log aggregation
so you add logs and it's still log so you can add more you know put
two logs together still it's a bigger log but it's still a log so
that's you know you have an empty log or you can combine two logs
that are maybe empty and you still have a big log as a result so logging
is one application of this and there are other applications many many
other occasion so one know it is a is a very useful abstraction so
now we'll see examples of how to implement the monoi typeclass by
either just doing it from scratch sorry implementing the PT TF and
so on or in a very interesting way if we assume pointed and semigroup
first so so pointed typeclass has this point function seven group
has the OP function and you know it has these two functions of the
same types while the names of the functions are completely material
of course they can rename all you want the types are important so
types are the same and so monoid is basically a semigroup combining
is pointed and we can write code exactly saying that and save us the
trouble of implementing everything from scratch so now we'll see how
that works so first let's implement monoid from scratch we will not
use trades will not use any names just the bare-bones implementation
so the idea of implementation is that first we do a partial type 2
type functions so we define a type function and the value of this
type should carry all the information we need to know this information
consists of two pieces first the empty value remember the specification
of monoid we need an empty value and we need a combined so empty value
is just T combine as a function from T and T 2 T so let's put these
two values in the tuple so we'll have a tuple of T and a function
from T and T 2 T that's good enough we don't really need any names
or trades or anything we just need the data the fact that they are
called empty or combined this fact is immaterial this is not important
we can rename this you can call us zero instead of empty we can call
this append instead of combined or add or whatever or op it doesn't
really matter what matters is the types so let's define the type domain
for this partial type function that means we defined implicitly of
type monoid int and 108 string so we define these values these are
the tuples having the selected int value and the function that combines
integers and the selected string value and the function that combines
strings let's do it this way done so now we have the partial type
2 type function defined now let's define the function sum well we
have finished defining the monoid typeclass let's define some partial
functions so the function sum will take an argument of sequence of
T and T will be constrained to have a monoid instance so or to have
an evidence of belonging to the monoid typeclass so how do we do the
implementation while we fold down the sequence the first argument
of the fold is the initial value which is the first element of the
tuple the second argument of the fold is the combining function which
is the second element of the tuple that's it we're done so some is
implemented and it works let's implement a monoid tie plus in a slightly
more verbose manner which might be not bad because it documents what
we're doing it also helps read the code later because this code looks
a bit cryptic with T comma T comma T parenthesis it's a bit cryptic
so let's define names so instead of just using a tuple let's use names
to pull name to post the case class so let's use it in case close
call it monoid put names empty combined so exactly the same type and
we define a type function in exactly the same way except now has names
and these names help to help us document what we're doing exactly
the same definition of the time to type domain except with the name
exactly the same definition of the function some except now we have
F dot empty and F dot combined because the typeclass has names if
you compare with the previous definition that was f dot underscore
one F dot underscore two so evidence now is a case closed and we can
use noise names that's exactly the same code now good let's look at
defining the pointed and the seven group in the same way so we define
it as case classes with so this is the partial type two type function
with its domain this is another partial type two type function with
its domain now this is the partial type two type function we want
to define and we want to automatically derive we don't want to repeat
implicit values at all here how do we do that we use this with we
use this implicit death so blessed F is a scholar feature which is
that this function will be called by the compiler whenever necessary
if it needs to produce an implicit value of this type it will see
if it can call this function and if implicit value of these types
are available if so it will insert the code to call this function
and create these this value and put that into your implicit argument
all of this will be happening silently and invisible so we define
the instance is a function with two implicit arguments which returns
the type evidence so it takes type evidence for semigroup and type
evidence for pointed and returns type evidence from unalloyed well
how do we do that well the type evidence from a Lloyd is a value of
this type so we need to just create a case class value of this sorry
of this type so we just say monoid which creates a case class value
and we need to provide it with two values with the empty and that
is the point from the pointed evidence and the combined function which
is the cop from the sum of your packages that's it so we have defined
the monoid instance the test will show that it works so we have not
defined two different monoid instances for in ten string nevertheless
we can use int and string in our code because this implicit def will
generate them so imagine here we have 15 different types it's a big
savings in code that we can automatically your life cause instances
typeclass instances from previously defined evidence values now one
other thing I'd like to show is that all these things are actually
already defined in the library called cats and also in another library
called scholar Z so all these typeclasses are pretty standard the
semigroup the maloik and so on and they're defined in these libraries
and the difference between these libraries may be names because types
are the same your standard mathematical structures but the names might
be different so in the cats library the names are empty and combined
in the scholars e-library they're different names types are the same
so how do we use the library to define Illinois so here's a here's
a test so we want to say that if we have a semigroup and we have appointed
we want to define a cation or instance so then we do it like this
we define a value of type Katzman on it and that's a class so we do
a new and then we override the functions empty and combine now unlike
what we did here the cat's library does not use keys classes it uses
traits and death methods so it means you need to override things but
that's how it is otherwise it's exactly the same thing and then we
can define some for the cats monoid and we can do the same test as
before now one another thing that is interesting is that we can check
laws in a generic way so I implemented this function to check cats
monoid laws which is a function with a type parameter that takes an
evidence that M is is from the cat monoid typeclass so these things
are necessary for law checking all these assertions and arbitrary
either property checks library notice I'm combining all kinds of syntax
here I have typeclass syntax like this I have implicit argument of
another type cost like this it's completely up to me and so this function
will check their left and right identity laws that combine of empty
and M is equal to M combined of M and M T is equal to M when the function
data is equal is a general way of comparing two elements of type M
it might be non-trivial if M our function types so that's why I have
this data is equal by definite by default this is just a comparison
but it could be different for function types we'll see examples for
this social tivity law also needs to be checked up for all ABC and
combine a combined BC is equal to combined combined ABC so that's
associated and then I can call this function for interest ring but
that won't compile because we didn't define them an audience in school
double isn't a side here's how we can check laws for the partial type
two value functions in a typeclass I'm using the Scala check library
and the typeclasses it uses itself for instance or the arbitrary typeclass
that typeclass means that the type has a function that produces arbitrary
values of it or some set of values and that allows me to write for
all function but of course cannot check all possible values but it
checks the large number of them so then I write a function like this
which is a generic function taking any type T as long as this type
has an arbitrary instance and also a semigroup instance and then I'm
imposing the condition that for all X Y \& Z of type T the operation
is associative so the operation of X with Y Z and gives me the same
result as first combining x and y and then combining that result with
Z now see I could have rewritten this function in a different way
in a different syntax like this I can combine the typeclass connotations
and this is just a different syntax and then I don't need to write
this however now I need to replace this with something I need to access
this operation which is the partial type 2 value function in the semigroup
typeclass previously I had an implicit argument with named semi group
F now the name is hidden I don't know what that name is so instead
of doing this I just say implicitly same in group t that's and that's
exactly the same thing so implicitly fetches the value of the pro
of the implicit parameter of the given type and there's only one because
there is a compile time error if they were two different implicit
values of the same type so that's how I would write the code if I
didn't want to put an explicit argument argument but actually it's
exactly the same function that if I look at this function here I use
it in exactly the same way I don't call it with any arguments so this
is just syntax I can use this syntax or I can use the syntax I had
before so once I have defined this function I call it by substituting
different type parameters and this itself is also a partial type to
value function that takes a type parameter and returns an assertion
value and that better be success if it's a failure the test would
fail so that's the way I can use the the Scala check library to test
laws similarly I have a function that tests the monoid laws it takes
an annoyed evidence and then tells me that there is a combined with
empty which returns me the same value so notice this function is defined
separately from the typeclass but it uses the same type domain as
a time code typeclass it does not change the monoid or arbitrary type
domain so I can combine two different type domains so there's no problem
in defining partial type two value functions later in the code without
changing the code of monoid typeclass later I can define further partial
type two value functions like this one which will work on any type
belonging to the relevant typeclass so that's the kind of extensibility
we wanted use that to make my tests that verify with my defined instances
for the moon the weight of integer in the noid of string they satisfy
the correct mathematical law laws here's an example when this does
not happen it does not satisfy the laws I create an instance of semigroup
for a boolean but I use the operation which is the boolean implication
if X than Y now this operation is not associative if I define this
as my semigroup evidence which is the same as to say if I define this
as my typeclass instance for boolean of semigroup typeclass then my
law will not be satisfied so the associativity law is not going to
be satisfied when I call this function this test will fail and it
will print a counter example let me run this actually the Scala check
library is such that it doesn't just check the laws for a large number
of values of the parameters but if it finds some values that do not
satisfy the law it tells me what those values are it will print the
counter example values and then I can write a test where I can debug
to see if this was about to see why it did not satisfy the law or
I can write a test and that will correctly pass the law so let me
just wait until this compiles there's a number of things it needs
to be it needs to do and wellness test runs yeah so it fails it says
true did not Eagle falls the current past generated values false false
false so it prints a counter example in the previous run I had a different
counter example which is false true false but false false cause apparently
is also a counter example so then I have a test here that verifies
that the laws are not holding and I use again this implicitly just
to fetch in this but I could just instead of this since I know what
that evidence is I can just say that the bad sending your evidence
with exactly the same thing that's to say I made I declare that as
implicit so fetching the implicit of this type is the same as just
using your value of the type so now tests pass so let me undo my change
alright so now let's consider a different example of typeclass which
is a typeclass for a type instructor the example is that of a factor
so the type constructor is a function if it has a map operation equivalently
in earth map that satisfies the function laws which were they had
identity law in the composition law now we would like to write a generic
function that tests the functor laws so this function would look like
this it will take the type constructor as as well as three types say
ABC has type parameters and it will then run the tests with various
values and check that all the laws hold for this type constructor
this generic function should require that the type constructor F be
a factor and we need to also access the function map that is defined
for the given type constructor F so therefore we say that the map
must be a typeclass so it is a partial type to value function whose
type domain is following the filter type constructors while the factors
F that we have declared and then we constrain F to belong to that
typeclass in this function unit check frontier laws where did we do
this in exactly the same way as before the only difference is that
the type construction now is of a different kind is a type parameter
F is a type constructor and not just a type based type nevertheless
we just declare a partial type 2 type function called factor which
will be called like this it's type parameter is this F which is this
type constructor and this will be well defined only for type constructors
F which we have defined a functor instance in other words will define
this implicit evidence value for each functor instance and we will
then require that this implicit argument be one of the arguments of
the check functor loss function we will see the implementation in
a second and for now just note that functor is a higher-order type
two type function it's a function whose argument it's a type function
but it's argument is a itself a type function whose effort self the
type function so just as we do with ordinary functions and we say
that a function whose argument is itself the function is a higher-order
function so here we just say this is a higher-order type function
because it's argument is a type function it's already let's look at
the test code them so we proceed in exactly the same way if you want
to implement the Thunderer typeclass there's no matter that it's a
higher-order type function we do exactly the same thing we define
a trait let's say with a type rounder which is this F and we need
to use the syntax with square brackets and underscore to show that
this type is itself a typed function so this type parameter is itself
a type function the trade has this def method which has itself to
further type parameters because the factors map function has this
type round now having these two type parameters pretty much forces
us to use the trait with the DEF method for implementing the partial
type two type function we could not do this within a data type because
in Scala data values cannot have it themselves type parameters only
def methods can have type parameters there are some ways of circumventing
this limitation but they do not significantly change this fact they're
just hiding the fact that somewhere there is a definite end with type
parameters so let's not try hide this fact but use the trade with
definite as very necessary so this is a partial type two type function
what what is its type domain what are the type constructors for which
we can have some values of type funky of F well let's say data one
is one such type constructor so let's define some type constructor
like this simple one let's define the evidence that this type constructor
belongs to the type domain of this type function so the evidence means
that we create a value of type filter of data 1 and this value could
be anything could be evolved it could be an object so one way to write
it is to say implicit object another way would be to say implicit
well and then to say new here it's another different way of doing
the same thing which one is better it's not so clear right now let's
just use one of them now in this value when we extend this trait with
a specific value of the typewriter we need to override this death
metal because this def method has a different implementation for each
type for each type constructor here so we need to override it so here
we just write the implementation of the f map for the data 1 constructor
so we already know how to do these things and I will not dwell on
how to implement the factor instance for case closed now this is this
is the entire definition of the partial function for type 2 type it
now let's define our type 2 value function the map will use the F
map will define generically the map so how do we do that well the
F is one type parameter and we say this must be a functor that close
and alb are two other type parameters and we define the F map as the
functors F map so we fetch the implicit value which is this value
or take the F map of it which is this sniff method apply distill F
we get F of A to F of B and then apply this function to F a this gives
us a value of f of B and that's the map method so the map method basically
by hand we just use the F map and call it first of F and then on data
and one away but you see this implementation is generic for all factors
it is not specific to data one so in this way we have defined a generic
partial type 2 value function where the type constructor is constrained
to be a factor this is exactly what we wanted it would be impossible
to call this with a type parameter that is not a factor because there
is the implicit requirement so it required implicit argument it wouldn't
be available other words so let's see how to use this so okay so we
have this map function how do we use it let's create some data of
this type and we just call that function on that date just an ordinary
function the fact that this is a partial function is transparent we
don't have to say what the type is the types are infrared data has
a specific type data one of int this is a specific function from integer
string so we transform this and we get data one with values one a
B C 2 a B C or D 2 this transformation the result is of type data
one of string so that's a typical result of applying a santur to data
so as you see using the function map is no different from using a
loan generic function but the definition of map is a generic one it
would not have to be extended if we have more data types like data
to that's safe with another function instance won't have to extend
this function so this is very good for library design because we can
put these functions into the library and we'll never have to extend
them it's only the users code that needs to extend this needs to declare
functor instances for new data types but this has to be done once
together with each data type and you're done here's a function that
can check functor laws it's a bit involved because functor laws involve
arbitrary functions from A to B and from B to C and so I explicitly
write down all the arguments that I need so these are the implicit
arguments of this function factor f is a typeclass for functor then
there are these arbitrary instances of typeclass so the Scala check
library needs to use arbitrary values of these types to run the checks
and if I don't put them here that it doesn't know that they exist
but finally when I use this function I just give it some specific
types and I don't have to specify anything more than that all these
arbitrary instances are implicit values that are defined in the library
and I don't want to talk about them in my code so for library design
this is great the library becomes much more powerful and know his
record needs to be written so just to be sure this is clear what this
is doing it's checking the factory laws for specific types so data
one is a factor and for example a the type parameter a here is int
so here the identity law will be only checked for the int type so
that into the int identity on data one instances of type data one
int are preserved after F map that's the only thing that can it will
check now I can obviously add some more checking here with different
types with other types it's up to me I I can do this or I can neglect
doing this but no more no necessary extra code so so let us see how
the same works with a cat slightly you know the cats library has a
standard function typeclass which is the cats not funky it works in
a very different way except that it uses makeup rather than death
map so the order of arguments is that first there is the data or the
functor and then there is a function whereas the f map has the opposite
order first the function and then the data now a convention that I
follow is that the implicit values should be in the companion object
of the data type now this is a useful convention because Scala has
a mechanism for searching for implicit values and it will search in
the companion objects of the types you are using so there is some
function that tries to find implicit value of functor theta1 the compiler
will automatically search in the companion objects of function and
in the companion object of later one now it makes sense to put data
one specific stuff in a companion object of data one and functor specific
stuff into a companion objective factor the factor is in the library
later one is my own code so that's why I first defined my data 1 as
as my own custom type and then I define companion object which is
in Scala just an object with the same name as the type it's a special
convention in Scala so in this object I have my implicit value I can
again I can do implicit Val and then equals new if I want or I can
do implicit object extends this it's up to me there is not a big difference
and not a very different amount of code to be written when I write
the implementation of the function instance so I implementing lab
and then I check the factor laws so this function is very similar
to what we saw above except it uses the cat's function typeclass within
map so that's implemented in my test code very similar code checks
the identity law in the composition law given three arbitrary types
and a type constructor that must be effective so now having seen this
let us take an overview of what we have achieved we have been working
with type 2 value functions and type 2 type functions let's compare
value to value functions and type 2 value functions now value to value
functions as the ones that are the ordinary functions the domain of
a value to value function is the set of its admissible argument values
we call that a type so the type is the subset of values that are admissible
as an argument of functions of for example if the function takes an
integer type of the argument when strings are not admissible arrays
of something are not admissible nothing else except integers are admissible
so in the set of all integer values is the value domain of that function
and that's what we call a type so the function can be applied safely
only if the argument is of the correct type and this is the example
of having a function we declare its type and then we apply to some
Y value and unless Y has this type this is not safe to do so so not
safe to apply and if Y is not of this type there will be a compile
time error that's how we use types types prevent applying functions
to incorrect arguments now let's look at the partial type 2 value
function it has a type domain which is the set of admissible argument
types or type parameters the type domain is a sub set of types a sub
set of type parameters that the function can accept so that is actually
called a kind this is terminology it is accepted in functional programming
so we say that the type 2 value function can be applied only to type
arguments of the right kind similarly to a value to value function
that can be only applied safely if the argument values are of the
right type a function like this where we declare it's type parameter
constrained to be in some typeclass this can be safely applied to
a type parameter a only if a belongs to this typeclass we say only
if alias of the right kind and in both cases the error will be caught
at a compile time so if this is not the right type if this type does
not belong to this typeclass writing this code will not compile in
the case of PT BFS that will happen because the implicit argument
will not be found but no matter it will still be a compile time error
so kinds are the type system for types values are of the right type
type arguments are of the right kind so a typeclass such as my typeclass
defines a new a new kind as a subset of types there is a type notation
which we use and there's also a kind notation now the kind notation
is not part of the Scala language but we need a notation over the
last to talk about kinds so I suggest to use this notation so star
is a standard kind notation for any type basic type like integer or
string an actual type that has values so star is a type that has values
any type that has values and if I denote it like this with a typeclass
a notation that means I'm only considering types that belong to the
typeclass another existing available kind is a type function kind
notation for that is this so let's look at an example consider this
type and here F and T are types of different kinds F is a type constructor
T is a type basic type not a type constructor so F is a type constructor
or type function and so f has this kind is a function kind which is
similar to function type except for types so it takes a type and returns
a type a kind of T is the star T is just a basic type the kind of
F is the function from basic type to basic type so this is the kind
notation star means the basic type error means a function and colon
with typeclass I suggest to use that for typeclasses here's an example
we can define a type like this this is a type this is also type but
this is a type function this is also type function so the kind of
this is actually this expression it has two arguments the first argument
is a type function which is this kind the second argument is a basic
type which is a star kind and the return of this type function is
a basic type which is a star called a return is a basic type pretty
much always and so that would be the kind notation for app if I define
the up like this note that Scala compiler will not compile if I put
our own kinds so for example if I put up with two type parameters
but a and B but a is not a type constructor so the kind of a must
be the function kind or the type function kind if this is not so the
Scala compiler will given there let's look at the test code so here
I define type function just any type function here I define the app
as in the slide now if I define type X like this I'm applying the
type function app to the arguments G and int G is itself a type function
defined here and so that's okay because app has the right kinds of
arguments here so after I apply this X becomes the result of applying
G to int which is this that's verified of X indeed is this so let's
say X of type X is left of that and that compiles now if I were to
try writing code like this it won't type check because the kinds are
wrong the first argument of F must be a type function but it is not
it is a basic type similarly here the second argument of app must
be a basic type but it is not it is a type function so neither of
these two will type check let's look at a little more complicated
example we will write the tag current notation explicitly here again
first we define a type constructor then we define app which has two
arguments and the kind of app is this and we can write more verbose
Li with argument names if we wish so f has the kind start to start
a has the kind star so actually Scala requires the syntax that shows
the kind you cannot just say F comma a if you write out of F comma
a it will assume that F has star kind so you have to write this syntax
if you want a higher kind higher order type now let me just say the
terminology in functional programming has higher kind of types I don't
think it's very limiting types are types are not kind it we don't
say it's a higher type function we say higher order function because
it's a function that takes functions as arguments here we have functions
of types that take other functions of types of arguments why should
we call it anything else than higher-order type function or higher-order
type higher kind it is not very illuminating here so here is another
different where I define up with two peas which is three parameter
function the first parameter is itself a higher-order type function
of the same kind as app so I can express it like this in Scala it's
a little harder to read I have two arguments the first argument is
a type constructor the second argument is a basic type and the function
P is a type function of these two arguments so the kind of P is this
the kind of Q is this and the kind of R is this basic typing so then
I can define this you see I'm applying when I define the type function
I said I need to apply P to Q and R and so Q is of the right kind
to be put into P so now I can say define X the same X that I had here
defined like this F by G int I can say app of app G and it will be
exactly the same thing I verified the same things as I verified before
namely that putting arguments of the wrong kinds does not work now
there's one other thing that we would have to do if we want to be
completely free dealing with type functions remember that in functional
programming it's important to have anonymous functions not just named
functions anonymous functions enable a lot of freedom in programming
and without them things are difficult so functional programming really
wouldn't work well without anonymous functions quite similarly we
need anonymous type functions until now all we've seen were named
typed functions all these names but we want also anonymous type functions
because without them things are just not always possible to express
anonymous type functions are more difficult to write in Scala but
it is possible to write and there is a compiler plugin that makes
the syntax easier it's just the syntax plug-in and I use that plug-in
which is available and this address it's called the kind projector
plug-in so let me show you how it works I define a type function just
a simple one and I define the app just as before now I want to define
a type function higher-order type function that takes a type constructor
and applies it twice to its type argument so app two should take a
P and the Q and then it will put Q twice into P so to speak so you
see the first argument of P is a type constructor now previously I
put Q as the first argument but now I want to put Q twice and on the
SEC U of Q of a type argument to express this I need an anonymous
type function because I want to put that anonymous type function as
the first argument of the type function P unless I have that I cannot
express this behavior this is the syntax of the kind projector it
has the special single lambda and then it allows me to write my anonymous
type function like this where X is now a new type variable anonymous
functions type variable so this is a very similar syntax except under
lambda and that is what the kind projector provides now instead of
lambda I can write instead of the Greek letter lambda I can write
lambda in Latin alphabet like that which is just more verbose no difference
so this app to has this kind it's the same kind of app from the previous
example except that here it instead of using Q once it uses Q twice
it's it composes Q with itself q is a type function it composes that
type function with itself in order to express such behavior we need
anonymous type functions so now if I apply this app to using the o2
type constructor which is just an option of a tuple a a then this
option of a tuple will be applied to itself twice so it will become
option of a tuple of option of tuple option of tuple and indeed that's
what it is x2 has this type and type checks so this is a bit of an
advanced topic the anonymous type functions but this is similar to
anonymous functions so just the syntax has to be a little different
so as a result of this we have a uniform view of values and types
types are two values as kinds are two types typeclasses are just a
certain sort of kind function current type function is another sort
of kind so there are different kinds there's type function kinds and
typeclass kinds and you can combine them you can have a typeclass
for type functions which is for instance for the factor factor is
a type function and we can have a typeclass for it you can of course
have typeclasses for higher order types as well so the syntax becomes
a bit more involved but you can still do it if there is a use case
for it so I showed you that there are some use cases for all these
generic functions main idea being that you want to save your code
you want to implementing generically right code once and have it work
for all factors for all types from a certain typeclass and another
thing that Scala provides by way of convenience is the implicit method
syntax so let me show you what that is this is very often used together
with typeclasses but this is a purely syntactic convenience there's
no more power that this brings us the partial type 2 value functions
are exactly the same as before all of these implicit things are just
syntactic conveniences that make code shorter the basic idea of partial
type T value functions remains the same so here is how it works in
Scala in principle there are two sorts of syntax available for functions
the first is the syntax similar to that of the ordinary mathematics
where you say function name in parentheses you list the arguments
Scala also allows you several lists or one or more lists of arguments
another way is to have a syntax like this X dot func of Y which is
the method kind of syntax like object-oriented method all pretty much
object-oriented languages use this syntax in Scala there is another
equivalent syntax whenever you can say X dot funk of Y you can also
say X space funk space Y sometimes that's easier to read now these
functions are similar but they're just implemented differently and
in some cases you can put you might prefer one in another case another
so here's an example when it is convenient to have the method like
syntax imagine we define a function called plus plus plus it is a
partial type two value function and the type domain for it is this
typeclass has plus plus plus and then there are some arguments and
one argument is of type T so you will have to call this function like
this plus plus plus of T and art now you would like to write this
instead this is much more readable T plus plus plus Arg whatever plus
plus plus means in your application this is certainly easier to read
than this Scala provides a way of implementing that syntax this is
what I call the implicit method syntax extension syntax so what is
the implicit method syntax suppose you want to convert func which
is a partial type to value function to the syntax you declare that
as a method on a new trait or class well a new trait is necessary
usually a class is sufficient and you declare an implicit conversion
function from T to this new class and this implicit conversion function
is a PTV F using the same typeclass so in this way you automatically
extend your existing typeclass with new syntax you do not change a
type domain of your typeclass when you do this since you don't change
the domain you don't need to change the code of the existing typeclass
so this can be done in a different piece of code or a different library
different module and the implicit conversion function can be done
as an implicit class I will show you the code in a second and that
makes called shorter action here's how it works suppose we implements
the monoid typeclass so this is our partial type 2 type function which
has some methods in the trait here is our syntax so this is this new
trait or class that we need to implement for the syntax purposes I
called an implicit class and it has one argument which is so it's
implicit class is really a function that at the same time defines
a new type of this name and the function that creates values of this
type the function has this argument it also has an implicit argument
of the partial type 2 type per function which means that this is a
partial type to value function so like I said implicit class means
you define at once and you type and a function that creates a value
of this type with this argument so in this way it's doing what I said
here to do in one step I define a new trailer class a new type and
I define a function that creates values of that type from values of
type T so here's type M at the same time I constrain the type parameter
m to belong to the typeclass la nuit so now it's a very short code
comparatively all I need to do now is to implement a method that I
want the method I want is the syntax method so let's call it append
log I can change the name if I want because this is an extension so
I do not actually change the functionality of monoid as it was previously
defined I'm adding new syntax to the previous syntax the previous
index will still keep working and so I can have a new name if I wish
for the new syntax so this is the method that will be available on
108 types and the method will be append log with an argument Y I will
just use the combined method from the moon wind and there's no new
functionality here I'm just repackaging the combined in the new syntax
so how does this work then here's a data type in my application which
is log data as I mentioned before logging is a typical monoid example
because there isn't there's an empty log and there's appending to
log so you can put two logs together and it's a bigger log again so
let's define what empty log means is the string that says no log so
far and then let's create a menorah instance which is a monoid of
type my log data I'm using implicit Val here why not so the combined
function will check if one of the logs is emptied and I just take
the other one if not empty I will concatenate logs and add a new line
between those log lines so this is a little more sophisticated than
just concatenating strings so my logs will be well formatted will
always be new lines separated then I import this monoid syntax now
this is the new thing that I defined here for syntax I import and
now here's the code of my application I have some log values here
one log message another log message and then I just say initial log
a pen log log data wand a pen log log data - so I'm appending all
these logs so this is the new syntax but I have just defined and it
works the log result is equal to this string what I could do for instance
very easily now cuz I could rename the second log to {[}Music{]} plus
plus plus what happens then I have code like this so this might be
more readable depending on my application I am free to add new syntax
without changing the typeclass that was defined before so I'm completely
open to extension without having to change any of the library code
for Edmund okay so to summarize what we have gained is that the partial
function that we defined appears as emitted on values of the relevant
types and only on those types so the plus plus plus will be only defined
on those types that have a monoid instance in my previous example
and the new syntax is defined automatically on all the types I did
not have to define plus plus plus specifically for my log data not
like that at all I define it generally for any type M it has a monoid
instance so this is the power of typeclasses let's now go through
some worked examples this is going over the entire material of this
tutorial the first example is that we want to define a partial type
value function of this kind the bit size has a type parameter T and
if that's an int type it returns 32 and if it's a long net return
64 otherwise this type the value function remains undefined how would
we implement that we go systematically so a partial type 2 value function
requires first of all the partial type 2 type function to define a
type domain so let us define this type constructor as a case class
we could not just use a type definition because that's all going to
be int for all the types T and so that's going to be a type collision
so let's do a case class instead so this is a type constructor that
will be our personal type two type function now we want to define
this only for intent law so we do implicit Val bit-sized int evidence
has bit sized with value 34 and long evidence has bit sized with value
64 32 64 notice we put into the type that will be our type function
we put the information that this function needs that the function
we want is the bit size that will return to our 64 this information
is carried by this type so the f-type evidence value here is the information
we need that's almost always going to be the case for any partial
type functions the partial type to type function represents a type
that carries all the information we'll need about the type that the
type 2 value functions we'll use later and the only information we're
required to return is this number so let's put this number as our
type in here now we define the partial type 2 value function so bit
size has an implicit argument of type with size T and that returns
just this evidence value dot size that's enough that's all we need
to do and indeed this works as expected the test verifies that the
next example is to define the mono it instance for this type now notice
this is an option of a function from string to string so I'm going
to use that as a type and I'm going to use the cats library for the
standard memory typeclass so my tests here are all going to use the
law checking for the cats library that I implemented in a different
file so this is my data type I just define this as an option from
string to string since this is such a simple type no need to do case
classes here myself so defining a typeclass instance for the standard
typeclass in the library means that the type constructor that PT TF
is already defined in the library it is the monoid already defined
in the cat's library the type constructor is there but I'm going to
add more types to the domain so that's always going to be possible
adding north types to the domain of this portion function and this
so because the mono it is not sealed trait is just a trait not sealed
here and so I can always add new new instances or new types to the
type domain of that partial function so I'm going to so add this type
to the type domain in order to do this all I need to do is to create
value of the type monoid of data that's all that is required now so
here's one value I create on this type I need to override two functions
empty and combine the empty is a value of this type it combined is
a function from two values of this type to a third value of this type
so what is a naturally selected value of this type well I can think
of many one could be it's an option of identity function so for some
identity function so that's what I write here and the combined would
have to be such that if there is a non-empty option then I compose
these two functions so I have some of one function string to string
some of another function I can post these two functions so this is
the code I'm writing here so if there's a sum and the sum and I compose
these two functions in all other cases I return none so clearly the
identity function here is a neutral element for this operation so
if I compose identity with anything I get the same function again
in all other cases I get none anyway so this will satisfy the laws
hopefully it's kind of a little boring instance because it very often
produces none whenever one of the two parts of the combined one of
the two arguments of the combined function isn't none it will produce
no another instance would produce not much less frequently how would
that work well if one of the two arguments of the combined function
is none then I return the other one and so none becomes the neutral
element and then I do compose the functions if they have two functions
so basically I prefer to keep the function if I have one is non-empty
option I prefer to keep it unless both are empty then the result won't
be none so that's perhaps a more interesting instance but we're free
to choose one or the other it's not really clear at this point which
one would be more interesting so there could be different implementations
of the same typeclass and there could be use cases for both of them
because of this I don't make them implicit in my test code I want
to keep them explicitly implicitly be possible with only one of them
because I cannot have two positive values of the same type and I check
that both of these satisfy the monoid laws no the satisfying minora
Clause requires two implicit arguments of the arbitrary data and non-oil
instance so I since I'm not doing the implicit I have to put these
arguments by hand if I look at this function it has an arbitrary typeclass
and the monoid typeclass so I need to present arguments and the arbitrary
data I don't know where to get that but it's implicitly available
so I just say possibly arbitrary data and that's how it works both
are valid monoid instances it turns out the third example is by assuming
that a and B are types with monoi oil instance I define a monoid instance
for the product type so this is a interesting example because it shows
that I don't have to implement everything from scratch if there are
monoid instances for some previous types I can just define monoi instance
for a new type and don't have to write a lot of code maybe those monoid
instances for the previous types had a lot of code on them but this
code is not very large so how do I do that well I use this implicit
death mechanism which is I define a function that produces the type
evidence for the product type so the type evidence is the value of
a type monoid of product type so i produce this type evidence given
the type evidence for monoid a and one going to be so this is what
it means sure to define the noid instance for product a B if I have
monoid instances for a and B it means to make a function it takes
the two monoid instances I remind you that what noid instance is exactly
the same as a type evidence value it's just for from annoyed so typeclass
instance is exactly the same as a type evidence value it's a value
of the type that is the partial type two type function applied to
the type that were producing evidence for so if you want producing
evidence for type a and type evidence is something of type one noid
a and that's something we'll typically carry the entire information
when to implement know it for type a so that is also called for that
reason monoid the monoid instance for type a and so our task is to
implement a function that takes monoid instances for types a and B
and produces a monoid instance for type product a B well so how do
we do that we'll just say we're all new monoid which is the way to
extend the trait very quickly and we override two methods so we need
to produce a value of type tuple a B and the evidence will give us
a value of type a and the value of type B so we just put them into
a tuple and the combine works in a similar way so we know how to combine
two A's and we know how to combine two B's so let's shows us how to
combine a tuple a B and another tuple a B or just combine a separately
and you combine these separately and you use the evidence values to
fetch the combined functions from the one monoi oil and another monoid
so that's all let's test so this test is just going very slowly to
make it clear exactly what we're having achieved well we have achieved
so first we have no entered uh Balma lloyd instances in scope just
make sure we don't have them so this would not compile and this would
not compile now we declare these instances so for int let's say the
empty value is 1 and combined is product that's fine that works for
double-amputee is zero and combined as a sum that works to multiplication
actually wouldn't work because it's not precise enough and it would
violate associativity by precision errors by by roundoff errors so
I use addition not multiplication for double alright so after I have
defined instances as implicit values of these types we have both int
and double monoid instance event scope in this will compile and work
whereas here it did not alright so now that should be sufficient for
us to be able to derive the Malloy instance for the tuple int double
automatically and that works so the monoid law has work that's this
test runs and that's how it works the next example is to show that
if a is a monoid and B is a semigroup then the disjunction a plus
B is a mohamed now to show this means again to write a function it
takes a type evidence for monoid of a in the type evidence for semi
group of b and produces the type evidence of monoid of a plus B were
using equivalent terminology typeclass instance for monoid of a typeclass
instance for B as a semi group these are two values these are going
to be arguments of my function and the result of my function must
be the typeclass instance from annoyed for type a a plus B just either
a B so showing this is not some kind of theoretical mathematical exercise
but it's a specific coding exercise I need to write a function that
produces the evidence of monoid typeclass given these two evidences
and I want to check laws so here's how it works this is this function
has two type parameters and two implicit arguments one is the evidence
that a E is a monoid and the other is the evidence that B is a semigroup
remind you the semigroup doesn't have the selected element mono it
has both a selected element and the binary operation a semigroup just
has the binary operation no select an element so it's interesting
that if we combine the two of them with the disjunction and the result
can be monoid while in the previous example we had a product both
of them needed to be a monoid for the product okay I'm annoyed because
we need to produce the empty value so both of them must have an empty
value but here with disjunction we don't need to produce both empty
values just one is sufficient so that's why one thing I'm annoyed
and another is the semigroup is sufficient alright so let's see how
that works so empty element needs to be of type either a B well obviously
we don't have selected naturally selected element of B because it's
a seven group we do have a naturally selected element of monoid so
we produce that okay now how does combine work all takes X of either
a B Y of either a B needs to produce again either either a B so we
have different situations we can have a and a we can combine a a obviously
because it's a monoid b and b we can also combine b and b obviously
so these cases are easy another two cases when it's not clear what
to do now we have a left of XA and the right of YB so how do we combine
now we can't really combine a and B we have no idea what these types
are except that one is a semigroup and another is a monoid but they're
not combined a ball directly one with the other so we have to ignore
one of them which one know we can think about this but basically things
don't work unless here you ignore the X and here you ignore the Y
and let's check that this is correct first we don't have instances
then we define instances for int so let's see we need something that's
not a monoid that's a semigroup it's kind of hard to come up with
such such a thing most operations we know are monoidal like plus because
as neutral we needed an operation that does not have a neutral element
but is associative so it's not easy well fortunately there is a paper
which I'm referring to here it describes a function that is non commutative
but associative and this function does not have an inverse and it
does not have a neutral element so I'm using this function which is
buttons function which is described in this article you find by this
formula very simple formula kind of curious in this not commutative
and yet associative in any case this is fine for a semigroup and for
the double I use the same instances before the addition with zero
and it works so if you change this to X instead of Y or here to whine
sin of X it was not work it will break so this is a curious definition
but that's interesting we have only one choice to define annoying
instance on the disjunction so think about it option is a disjunction
so if you have a monoid an option a is a monoid an optional is also
a monoid disjunction of two monitors I'm annoyed because the second
group is less than a monoid so that's good conjunction of two Mundo
videos I'm annoyed so you can combine Malloy's quite easily and this
is a generic way of defining these combinations the next example is
to define a functor instance for this type now here were using try
which is a scholar standard library API just not a data type it's
a special thing you find in the Scala library sequence is a another
thing you find in the Scala library so we're just trying to say how
can we do things with Scala library you find type constructors can
we make them into functors in terms of cats library let's say yes
so we say here's this type I use the type definition to define the
type function f so that F after this becomes a type function defined
like this and then and in order to define a functor instance I need
to define an implicit value of type factor of F so that's why first
I define this name F like this is a type function and then I declare
the new factor instance by overriding the map now I just implement
the map so hard way implement the map well FA is this F of a which
is a sequence of trial of a I can map over a sequence the result will
be a try I can map over try and the result will be again a try so
this is a functor because it has a map this is also a function because
it hasn't helped composition of two funders is a factors because this
is a functor because I can just compose maps like this and all I'm
doing here is to make the cat's library aware of this being a factor
so that's all I need to do I'm not going to check laws here because
it's quite a lot of work to produce arbitrary values of these types
especially they try but it is easy to verify that composition of two
functions is a factor so we could have checked the laws if we wanted
to the next example is to define a cats by functor instance for this
type function and notice that this type function has two type parameters
x and y and by functor is a typeclass defined in cats as a standard
by standard typeclass that is like functor except with two type parameters
it's a factor with respect to both of them so to define that first
we define a type function or type constructor the name of that is
Q as we are given here and then I'm using either just to reduce typing
I don't want to declare my own sealed trait and two cases for the
disjunction but I could of course declare Q as a sealed trait with
two type parameters all be just more typing then all I need to do
is to define an implicit value of type by factor of Q so this is code
that does that to do that I need to override this function by map
so by map means that it's a factor when you transform both type parameters
at the same time can that's a factor so by factor means that it's
a factor with respect to both type parameters at once so you transform
both of them at the same time and the result transforms as you will
expect so for instance either of a a be transformed with two functions
two arbitrary functions one a to C and the other B to D replaces this
with either CCD so that's what is required now I'm not going to write
code for this function and we're just going to use my curry Howard
library to implement the code automatically using the type this is
possible and it works I'm just trying to reduce the time I'm spending
writing code now disk this is something we already know how to implement
functions given their types and this work can be done by the computer
so why not the non-trivial part here is how to define typeclass of
instances and that's what we do by hand having done this we have defined
a simple set value and we check the by funky laws now the checking
the by funky laws is again a function I implemented in my test code
the next example is to define a country funkier typeclass so this
typeclass is like functor except it has contra f map where the direction
of the arrow visa counter to the direction of the transformation of
the function values the type constructor values since a contra factor
and the example consists of first defining a new typeclass having
this partial type 2 value function and then to define an instance
of this type laws for the type constructor of this formula which is
type argument a going to integer so this is a contra funky because
the type argument is to the left of the function here and we expect
to be able to define an instance instance of a country funky die class
for this type of structure so let's see how that works so the given
country factor is this and let me define the type right away so after
this I have defined the type function C so first we need to define
a new typeclass which consists of a partial type two type function
carrying the data and we're required to carry this this type function
is contra functor with type variable which is the C now see here is
a parameter name it's a little confusing perhaps let's call this C
Bo just so that we don't confuse this with this C over here so this
is a contra factor this is a trait having a deaf method of the signature
we're required to help and in order to show that this type domain
includes this C we are required to produce a value of type control
factor of C so we do it like this so we we can either do a new country
factor or we can do an implicit object extends country factor is the
same thing could you do that it would be equalled we need to override
this function and this function is to have this type from FB to a
to a tend to be target again I let my curry Harvard library handle
this code and once I'm done it will work let's do the same thing with
cats lightly cats library has a contractor typeclass which is called
contravariant and it has a different name and a different type of
the function instead of F myopic as my episode not a flipped argument
order other than that is very similar so how do I know this well I
just let me show you how I know this I just do that it's not happy
because I need to define country map as it tells me so I press this
button here and I say implement methods tells me what methods I need
to implement now C of a is a - int as far as I remember and now I
can just say that's my very hard lighting at work all right the title
notation which is required and then and then I define equality comparison
function for this scene it says it's a function so C of int let's
say is a function and comparing two functions is not immediate need
to substitute different arguments and compare results so that's this
C equal code that says for all T C 1 of T must be equal to C 2 of
T and when the function C 1 is equal to C 2 so that I substitute into
the law checker and it will check my laws and that works tests lastly
the next example is to define a functor instance for the recursive
type we find by this formula so here let me first define this recursive
type so it's the seal trade with three cases and the third case is
recursive it contains the same type of the trade itself all the cases
on the trait are parameterize by the type parameter a and extend cubed
so this is a functor because any always occurs to the right of the
error or here but this is the recursive instance so let's find it
so let's implement the functor instance with a new factor so functor'
is the cat's functor just to check and needs to override the map let
me type override it's not necessary to type all right but it's more
checking this way if I mix up the type it was tell me that I'm not
overriding it in the right way so with these object oriented things
more checking has a better so type override make sure that it checks
that you are overriding it with the correct type okay so what does
this function do this function takes an FA of type QA it takes a function
f from A to B and it needs to produce K only so to do that we have
to match on this value let me rename this to Kiwi for clarity the
first case is the function from int to a well obviously I need to
compose this with a function f and then I will have a function from
int to be as required the second case doesn't have any age so it remains
unchanged just the type has type parameter is changed there was C
2 of a now it's C 2 of B ok so this is option int I said option int
I'm not sure something is fishy here well X is int that's true oh
I see it's a case it does not tell me the type and it tells me something
else tells me that possible trying with your types so the interesting
case is the third case so the Q here is a key of a and I need to produce
key of B now I know how to produce key of B so I used a recursive
call to the same map that I'm defining and I call this one q + and
put this into c3 so in this case I remain in the c3 case and I transform
this Q by using the recursive definition of the same map so that's
how I have to do it there's no no other way really so in this way
we defined a functor instance so let's see how I would check that
this function works correctly so for example I get this implicitly
founder Q dot map is the value that I just defined I could have said
Q functor instance dot map because I just defined it here but I don't
want to know about these names these names are completely unimportant
as long as you make them implicit so I'm going to use this as my data
and put identity function on it and see if that transforms the data
into the same form and it does so it's just sanity check I'm going
to check the laws are more systematically in a second to do that I
need an equality comparison so that's a bit involved because there
are many cases so I'm going to compare each case and then there's
a recursive case as well and I need to compare functions so all of
that requires extra code so here is the case for comparing functions
here's the recursive call so if I need to compare a c3 with c3 I recursively
compare their contents of c3 using the same C equal function so that's
how this works this is just to see that this works this would fail
is that on the same and finally I call the frontal or checker on this
type using C equal and the tests pass the final worked example was
a bit more advanced so I started but it's not so bad it's to show
that you can define a functor instance for a disjunction if you have
functor instances for the two type constructors so this is a bit more
advanced because you need to define a type function that is a disjunction
of two given type functions so I need to define an implicit def which
would have to type arguments F and G each of them being type functions
and two implicit arguments which are the evidence that F is a factor
and that the G is a function but what am I going to return I'm going
to return if evidence that F plus G is a function but what is F plus
G F plus G is a new type constructor in other words a new type function
but it doesn't have a name yet and I cannot have a name for it because
F and G are my parameters so the only way I mean to express this is
to say that this is an anonymous type function so I'm returning type
evidence that an anonymous type function is of typeclass functor and
this is this anonymous type function it's X going to either f of X
and G of X so that's a type X is a type parameter this is type expression
that defines my type function and so the lambda syntax is syntax for
anonymous type functions using the kind projector plugin so that is
the kind of difficult part and other than that the school is quite
straightforward the difficult part is reasoning about these higher-order
type functions so for convenience I can put a name on it in the body
of this function but I cannot put a name on it before I define the
function return type so that's why this kind projector is important
but other than that it's not so bad so type either FG is a type function
that is the same as this anonymous type function over there it just
has a name now and that's now easier to use I can just put this name
whenever I need it so now I return this functor instance by overriding
the map I have can either F G of a and I have a function from A to
B and I need to return either G only so how do I do that well I get
this FG a which is either F of T or F of a on G of a I'm matching
it if it's F of a then I use the evidence for F to get the map of
F away with less transformation T and if it's in the right I'll use
the evidence for G to get the map for the functor gene use that map
on the G value and the transformation T so that's it that's a very
straightforward code now I test this code I create filter instances
for some simple data types D 1 and D 2 very very simple data types
for these data types I can just implement functor instances like that
automatically and finally I I say well let's call this d1 or d2 so
funny again for convenience introduced a specific name for the disjunction
of two factors so this d1 or d2 is now a new type function which has
a disjunction of the two type functions d1 and d2 and I check the
function laws for d1 or d2 now the implicit function here was called
invisibly I didn't have to call it myself to produce the type evidence
that this is a functor but you see this function requires that evidence
it wouldn't break even compiling if I didn't have that evidence since
it compiles them it's correct now in this code I will show you how
I attempted to do this without the current projector it's kind of
difficult and the problem is we need to define this name if we cannot
have anonymous functions anonymous type functions we have to define
this name somewhere but we cannot define it before we have the type
constructors F and G and so we have to define it inside the body of
the function and then we can define the instance like this and check
the laws but nobody outside of this function can see that instance
because this is a function this is not an object this is not a well
is this a function has to be called and when you call us it will define
the instance and check the laws for it which works but normally outside
of this code will be able to use this in instance so if you don't
use the kind projector then you can't really express this situation
but you actually have defined a functor instance for a disjunction
of two arbitrary factors given as type parameters and this functor
instance you see it's usable outside of this code so I can put this
into a library I can put this into the companion objective factor
and nobody will have to do any imports or anything just we'll just
work automatically so that's the power of implicit function definitions
that can automatically build new implicit evidence value so this is
this becomes a new impressive evidence value that is automatically
built up from existing implicit evidence values these two so this
concludes the worked examples and here are some exercises for you
that are quite similar to the worked examples and once you go through
these exercises you will really have a solid understanding of type
functions and typeclasses by way of conclusion let me summarize what
are the problems we can solve now we can define arbitrary partial
type two type functions and partial type two valued functions typeclasses
are just a way to systematically manage partial type two value functions
a lot different so just PTDS under student is somewhat more systematic
way we can define these together or separately we can define some
of them then later more partial type two values functions we can combine
them we can use a Katz library to define instances for some standard
typeclasses such as mono word semigroup functor and so on we can derive
typeclass instances automatically from previously derived ones or
previously programmed ones and we can reason about higher type functions
types and kinds whatever that is necessary so what is it that we cannot
do at this point no there's still things we cannot do one of these
things is to derive typeclass instances automatically for polynomial
data types notice the case classes sealed traits with disjunction
in case classes with with data in them although that curry Howard
library can do this in some cases it cannot do this in all cases and
cannot do this automatically by implicit deaths either this requires
more advanced tools which are available in the shapeless library and
I have a link to a book online book a guide to shapeless where it
is explained in depth how automatically an automatic derivation of
type last instances works in the shapeless library another thing we
cannot do is to derive a recursive type generically from an arbitrary
type function so we have just seen an example of defining a functor
instance for recursive type but this type was defined by hand so suppose
we have a type function f given as a parameter and we want to define
a recursive type R via this equation this is a recursive type equation
or also called a fixed point equation R is a fixed point of the function
f so this fixed point is a function of F and that function is usually
denoted by Y the so called Y Combinator it takes an F which is itself
a function and it returns its fixed point so the y must be defined
by type equation of this sort so Y of F must be equal to Y of F is
that R which is equal to F of R so Y of F must be equal to F of Y
of F but this doesn't compile Scala doesn't allow us to make things
like this more work is necessary to even define such things or have
a generic type function that computes the recursive type and we haven't
seen how to do it also for such recursive types we cannot derive typeclass
instances automatically so for instance if F were a functor with some
other type parameter then R would also be a factor or if if F is a
mono then I would also be able know either something like this we
would like to derive automatically typeclass instances whenever they
exist and we cannot do this with the tools we have seen here now there
are some advanced libraries like matryoshka that implement type level
recursion or type level fix points that are able to handle these problems
giving you a link here also shapeless might help not sure in any case
these are more advanced topics which are perhaps further away from
practical applications than I would like and there are some interesting
discussions there is a blog post very interesting series of explanations
of this kind of stuff that I encourage you to look at if you're interested
this concludes chapter 5 
\end{comment}

