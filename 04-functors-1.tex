%% LyX 2.2.0 created this file.  For more info, see http://www.lyx.org/.
%% Do not edit unless you really know what you are doing.
\documentclass[english]{beamer}
\usepackage[T1]{fontenc}
\usepackage[latin9]{inputenc}
\setcounter{secnumdepth}{3}
\setcounter{tocdepth}{3}
\usepackage{babel}
\usepackage{amsmath}
\ifx\hypersetup\undefined
  \AtBeginDocument{%
    \hypersetup{unicode=true,pdfusetitle,
 bookmarks=true,bookmarksnumbered=false,bookmarksopen=false,
 breaklinks=false,pdfborder={0 0 1},backref=false,colorlinks=true}
  }
\else
  \hypersetup{unicode=true,pdfusetitle,
 bookmarks=true,bookmarksnumbered=false,bookmarksopen=false,
 breaklinks=false,pdfborder={0 0 1},backref=false,colorlinks=true}
\fi

\makeatletter

%%%%%%%%%%%%%%%%%%%%%%%%%%%%%% LyX specific LaTeX commands.
%% Because html converters don't know tabularnewline
\providecommand{\tabularnewline}{\\}

%%%%%%%%%%%%%%%%%%%%%%%%%%%%%% Textclass specific LaTeX commands.
 % this default might be overridden by plain title style
 \newcommand\makebeamertitle{\frame{\maketitle}}%
 % (ERT) argument for the TOC
 \AtBeginDocument{%
   \let\origtableofcontents=\tableofcontents
   \def\tableofcontents{\@ifnextchar[{\origtableofcontents}{\gobbletableofcontents}}
   \def\gobbletableofcontents#1{\origtableofcontents}
 }
 \newenvironment{lyxcode}
   {\par\begin{list}{}{
     \setlength{\rightmargin}{\leftmargin}
     \setlength{\listparindent}{0pt}% needed for AMS classes
     \raggedright
     \setlength{\itemsep}{0pt}
     \setlength{\parsep}{0pt}
     \normalfont\ttfamily}%
    \def\{{\char`\{}
    \def\}{\char`\}}
    \def\textasciitilde{\char`\~}
    \item[]}
   {\end{list}}

%%%%%%%%%%%%%%%%%%%%%%%%%%%%%% User specified LaTeX commands.
\usetheme[secheader]{Boadilla}
\usecolortheme{seahorse}
\title[Chapter 4, part 1: Functor laws]{Chapter 4: Functors. Part 1: Functor laws}
\subtitle{How to recognize a functor}
\author{Sergei Winitzki}
\date{December 28, 2017}
\institute[ABTB]{Academy by the Bay}
\setbeamertemplate{navigation symbols}{}

\makeatother

\begin{document}
\frame{\titlepage}
\begin{frame}{``Container-like'' type constructors}

\begin{itemize}
\item Visualize \texttt{\textcolor{blue}{\footnotesize{}Seq{[}T{]}}} as
a container with some items of type \texttt{\textcolor{blue}{\footnotesize{}T}}{\footnotesize \par}
\begin{itemize}
\item How to formalize this idea as a property of \texttt{\textcolor{blue}{\footnotesize{}Seq}}?
\end{itemize}
\item Another example of a container: \texttt{\textcolor{blue}{\footnotesize{}Future{[}T{]}}}{\footnotesize \par}
\begin{itemize}
\item a value of type \texttt{\textcolor{blue}{\footnotesize{}T}} will be
available later, or may fail to arrive
\end{itemize}
\end{itemize}
Let us separate the ``bare container'' functionality from other
functionality
\begin{itemize}
\item A ``bare container'' will allow us to:
\begin{itemize}
\item manipulate items held within the container
\begin{itemize}
\item In FP, to ``manipulate items'' means to \emph{apply functions to
values}
\end{itemize}
\end{itemize}
\item ``Container holds items'' = we can apply a function to the items
\begin{itemize}
\item but the new items \emph{remain} within the same container!
\item need \texttt{\textcolor{blue}{\footnotesize{}map:\ Container{[}A{]}
$\Rightarrow$ (A $\Rightarrow$ B) $\Rightarrow$ Container{[}B{]}}}{\footnotesize \par}
\end{itemize}
\item A ``bare container'' will \emph{not} allow us to:
\begin{itemize}
\item make a new container out of a given set of items
\item read values out of the container
\item add more items into container, delete items from container
\item wait until items are available in container, etc.
\end{itemize}
\end{itemize}
\end{frame}

\begin{frame}{\texttt{Option{[}T{]}} as a container I}

\begin{itemize}
\item In the short notation: $\text{Option}^{A}=1+A$
\item The \texttt{\textcolor{blue}{\footnotesize{}map}} function is required
to have the type
\[
\text{map}^{A,B}:1+A\Rightarrow\left(A\Rightarrow B\right)\Rightarrow1+B
\]
\end{itemize}
Main questions:
\begin{itemize}
\item How to avoid ``information loss'' in this function?
\item Does this \texttt{\textcolor{blue}{\footnotesize{}map}} allow us to
``manipulate values within the container''?
\end{itemize}
\end{frame}

\begin{frame}{\texttt{Option{[}T{]}} as a container II}

Avoiding ``information loss'' means:
\begin{itemize}
\item \texttt{\textcolor{blue}{\footnotesize{}map{[}A,A{]}(opt)(x$\Rightarrow$x)
== opt}} \textendash{} ``\textbf{identity law}'' for \texttt{\textcolor{blue}{\footnotesize{}map}}{\footnotesize \par}
\item We have two implementations of the type: 
\[
\text{map}^{[A,B]}=(1+a^{A})\Rightarrow(f^{A\Rightarrow B})=1+f(a)
\]
and
\[
\text{map}^{[A,B]}=(1+a^{A})\Rightarrow(f^{A\Rightarrow B})=1+0^{B}
\]
The second implementation has ``information loss''!
\item Short notation for code (type annotations are optional):
\end{itemize}
\begin{center}
\begin{tabular}{|c|c|}
\hline 
\textbf{Short notation} & \textbf{Scala code}\tabularnewline
\hline 
\hline 
$a^{A}$ & \texttt{\textcolor{blue}{\footnotesize{}val a: A}}\tabularnewline
\hline 
$f^{[A]\,B\Rightarrow C}$ & \texttt{\textcolor{blue}{\footnotesize{}def f{[}A{]}: B $\Rightarrow$
C ...}}\tabularnewline
\hline 
$a^{A}+b^{B}$ & \texttt{\textcolor{blue}{\footnotesize{}x: Either{[}A, B{]} match
\{...\}}}\tabularnewline
\hline 
$a^{A}+0^{B}$ & \texttt{\textcolor{blue}{\footnotesize{}Left(a):\ Either{[}A, B{]}}}\tabularnewline
\hline 
$1$ & \texttt{\textcolor{blue}{\footnotesize{}()}}\tabularnewline
\hline 
\end{tabular}
\par\end{center}

\end{frame}

\begin{frame}{\texttt{Option{[}T{]}} as a container III}


\framesubtitle{What it means to ``be able to manipulate values in a container''}
\begin{itemize}
\item Flip the two curried arguments in the type signature of \texttt{\textcolor{blue}{\footnotesize{}map}}:{\footnotesize{}
\[
\text{fmap}^{[A,B]}:\left(A\Rightarrow B\right)\Rightarrow\text{Option}^{A}\Rightarrow\text{Option}^{B}
\]
}{\footnotesize \par}
\item A function is ``\textbf{lifted}'' from $A\Rightarrow B$ to $\text{Option}^{A}\Rightarrow\text{Option}^{B}$
by \texttt{\textcolor{blue}{\footnotesize{}fmap}}:{\footnotesize{}
\[
\text{fmap}(f^{A\Rightarrow B}):\text{Option}^{A}\Rightarrow\text{Option}^{B}
\]
}{\footnotesize \par}
\item Being able to manipulate values means that functions \emph{behave
normally} when lifted, i.e.\ when applied within the container
\item The standard properties of function composition are{\footnotesize{}
\begin{align*}
f^{A\Rightarrow B}\circ id^{B\Rightarrow B} & =f^{A\Rightarrow B}\\
id^{A\Rightarrow A}\circ f^{A\Rightarrow B} & =f^{A\Rightarrow B}\\
f^{A\Rightarrow B}\circ(g^{B\Rightarrow C}\circ h^{C\Rightarrow D}) & =(f^{A\Rightarrow B}\circ g^{B\Rightarrow C})\circ h^{C\Rightarrow D}
\end{align*}
}and should hold for the ``lifted'' functions as well!
\item The ``identity law'' already requires that {\footnotesize{}$\text{fmap}(\text{id}^{A\Rightarrow A})=\text{id}^{\text{Option}^{A}\Rightarrow\text{Option}^{A}}$}{\footnotesize \par}
\item It remains to require that \texttt{\textcolor{blue}{\footnotesize{}fmap}}
should preserve function composition:{\footnotesize{}
\[
\text{fmap}(f^{A\Rightarrow B}\circ g^{B\Rightarrow C})=\text{fmap}(f^{A\Rightarrow B})\circ\text{fmap}(g^{B\Rightarrow C})
\]
}{\footnotesize \par}
\end{itemize}
\end{frame}

\begin{frame}{Functor: the definition}


\framesubtitle{An abstraction for ``bare container'' functionality}

A \textbf{functor} is:
\begin{itemize}
\item a type constructor with a type parameter, e.g.\ \texttt{\textcolor{blue}{\footnotesize{}MyType{[}T{]}}}{\footnotesize \par}
\item such that a function \texttt{\textcolor{blue}{\footnotesize{}map}}
or, equivalently, \texttt{\textcolor{blue}{\footnotesize{}fmap}} is
available:{\footnotesize{}
\begin{align*}
\text{map}^{[A,B]}: & \text{MyType}^{A}\Rightarrow\left(A\Rightarrow B\right)\Rightarrow\text{MyType}^{B}\\
\text{fmap}^{[A,B]}: & \left(A\Rightarrow B\right)\Rightarrow\text{MyType}^{A}\Rightarrow\text{MyType}^{B}
\end{align*}
}{\footnotesize \par}
\item such that the identity law and the composition law hold for any type
\texttt{\textcolor{blue}{\footnotesize{}T}}{\footnotesize \par}
\begin{itemize}
\item The laws are easier to formulate in terms of \texttt{\textcolor{blue}{\footnotesize{}fmap}}:\texttt{\textcolor{blue}{\footnotesize{}
\begin{align*}
\text{fmap}^{A,A}\,(\text{id}^{A}) & =\text{id}^{F^{A}}\\
\text{fmap}\left(f\circ g\right) & =\text{fmap}\left(f\right)\circ\text{fmap}\left(g\right)
\end{align*}
}}{\footnotesize \par}
\end{itemize}
\item Verify the laws for \texttt{\textcolor{blue}{\footnotesize{}Option{[}A{]}}}:
see test code
\end{itemize}
\begin{lyxcode}
\textcolor{blue}{\footnotesize{}def~fmap{[}A,B{]}:~(A~$\Rightarrow$~B)~$\Rightarrow$~Option{[}A{]}~$\Rightarrow$~Option{[}B{]}~=~f~$\Rightarrow$~\{}{\footnotesize \par}

\textcolor{blue}{\footnotesize{}~~case~Some(a)~$\Rightarrow$~Some(f(a))}{\footnotesize \par}

\textcolor{blue}{\footnotesize{}~~case~None~$\Rightarrow$~None}{\footnotesize \par}

\textcolor{blue}{\footnotesize{}\}}{\footnotesize \par}
\end{lyxcode}
\end{frame}

\begin{frame}{Examples of functors I}


\framesubtitle{(Almost) everything that has a ``\texttt{map}'' is a functor}
\begin{itemize}
\item Specific functors will have methods for creating them, reading values
out of them, adding / removing items, waiting for items to arrive,
etc.
\begin{itemize}
\item Common to all functors is the \texttt{\textcolor{blue}{\footnotesize{}map}}
function
\end{itemize}
\item Need to verify the laws!
\end{itemize}
Examples of functors in the Scala standard library:
\begin{itemize}
\item \texttt{\textcolor{blue}{\footnotesize{}Option{[}T{]}}}{\footnotesize \par}
\item \texttt{\textcolor{blue}{\footnotesize{}Either{[}L, R{]}}} with respect
to \texttt{\textcolor{blue}{\footnotesize{}R}}{\footnotesize \par}
\item \texttt{\textcolor{blue}{\footnotesize{}Seq{[}T{]}}} and \texttt{\textcolor{blue}{\footnotesize{}Iterator{[}T{]} }}{\footnotesize \par}
\item the many subtypes of \texttt{\textcolor{blue}{\footnotesize{}Seq}}
(\texttt{\textcolor{blue}{\footnotesize{}Range}}, \texttt{\textcolor{blue}{\footnotesize{}List}},
\texttt{\textcolor{blue}{\footnotesize{}Vector}}, \texttt{\textcolor{blue}{\footnotesize{}IndexedSeq}},
etc.)
\item \texttt{\textcolor{blue}{\footnotesize{}Future{[}T{]}}}{\footnotesize \par}
\item \texttt{\textcolor{blue}{\footnotesize{}Try{[}T{]}}}{\footnotesize \par}
\item \texttt{\textcolor{blue}{\footnotesize{}Map{[}K, V{]}}} with respect
to \texttt{\textcolor{blue}{\footnotesize{}V}} (using \texttt{\textcolor{blue}{\footnotesize{}mapValues}})
\end{itemize}
Examples of non-functors that have a \texttt{\textcolor{blue}{\footnotesize{}map}}:
\begin{itemize}
\item \texttt{\textcolor{blue}{\footnotesize{}Set{[}T{]}}} \textendash{}
it works only when \texttt{\textcolor{blue}{\footnotesize{}T}} has
a well-behaved ``\texttt{\textcolor{blue}{\footnotesize{}==}}''
operation
\item \texttt{\textcolor{blue}{\footnotesize{}Map{[}K, V{]}}} with respect
to both \texttt{\textcolor{blue}{\footnotesize{}K}} and \texttt{\textcolor{blue}{\footnotesize{}V}},
because it is a \texttt{\textcolor{blue}{\footnotesize{}Set}} w.r.t.
\texttt{\textcolor{blue}{\footnotesize{}K}}{\footnotesize \par}
\end{itemize}
See test code and https://gist.github.com/tpolecat/7401433
\end{frame}

\begin{frame}{Examples of functors II}


\framesubtitle{Polynomial type constructors as functors}
\begin{enumerate}
\item Short notation: $\text{QueryResult}^{A}=\text{String}\times\text{Int}\times A$
\begin{lyxcode}
\textcolor{blue}{\footnotesize{}case~class~QueryResult{[}A{]}(name:~String,~time:~Int,~data:~A)}{\footnotesize \par}
\end{lyxcode}
\item Short notation: $\text{Vec3}^{A}=A\times A\times A$
\begin{lyxcode}
\textcolor{blue}{\footnotesize{}case~class~Vec3{[}A{]}(x:~A,~y:~A,~z:~A)}{\footnotesize \par}
\end{lyxcode}
\item Short notation: $\text{QueryResult}^{A}=\text{String}+\text{String}\times\text{Int}\times A$
\begin{lyxcode}
\textcolor{blue}{\footnotesize{}sealed~trait~QueryResult{[}A{]}}{\footnotesize \par}

\textcolor{blue}{\footnotesize{}case~class~Error{[}A{]}(message:~String)~extends~QueryResult{[}A{]}}{\footnotesize \par}

\textcolor{blue}{\footnotesize{}case~class~Success{[}A{]}(name:~String,~time:~Int,~data:~A)~}{\footnotesize \par}

\textcolor{blue}{\footnotesize{}~~~~~~~~~~~~~~~~~~~~~extends~QueryResult{[}A{]}}{\footnotesize \par}
\end{lyxcode}
\end{enumerate}
See test code
\end{frame}

\begin{frame}{Examples of functors III: non-functors I}


\framesubtitle{Type constructors that cannot have ``\texttt{map}''}
\begin{enumerate}
\item Type constructors that \emph{consume} a value of the parameter type
\[
\text{NotContainer}^{A}=\left(A\Rightarrow\text{Int}\right)\times A
\]

\begin{lyxcode}
\textcolor{blue}{\footnotesize{}case~class~NotContainer{[}A{]}(x:~A~$\Rightarrow$~Int,~y:~A)}{\footnotesize \par}
\end{lyxcode}
\item Disjunction type constructors with non-parametric type values
\end{enumerate}
\begin{lyxcode}
\textcolor{blue}{\footnotesize{}sealed~trait~ServerAction{[}Res{]}}{\footnotesize \par}

\textcolor{blue}{\footnotesize{}case~class~GetResult{[}Res{]}(r:~Long~$\Rightarrow$~Res)~extends~ServerAction{[}Res{]}}{\footnotesize \par}

\textcolor{blue}{\footnotesize{}case~class~StoreId(x:~Long,~y:~String)~extends~ServerAction{[}Long{]}}{\footnotesize \par}

\textcolor{blue}{\footnotesize{}case~class~StoreName(name:~String)~extends~ServerAction{[}String{]}}{\footnotesize \par}
\end{lyxcode}
\begin{itemize}
\item The type constructor \texttt{\textcolor{blue}{\footnotesize{}ServerAction{[}Res{]}}}
is called a GADT (``generalized algebraic data type'')
\begin{itemize}
\item Not sure what the short notation should be for GADT types!
\end{itemize}
\end{itemize}
\end{frame}

\begin{frame}{Examples of functors III: non-functors II}


\framesubtitle{They could be functors, except for incorrect implementations of ``\texttt{fmap}''}

We need:
\[
\text{fmap}\,(f^{A\Rightarrow B}):\text{Container}^{A}\Rightarrow\text{Container}^{B}
\]

\begin{enumerate}
\item \texttt{\textcolor{blue}{\footnotesize{}fmap(f)}} ignores \texttt{\textcolor{blue}{\footnotesize{}f}}
\textendash{} e.g.\ always returns \texttt{\textcolor{blue}{\footnotesize{}None}}
for \texttt{\textcolor{blue}{\footnotesize{}Option{[}B{]}}}{\footnotesize \par}
\item \texttt{\textcolor{blue}{\footnotesize{}fmap(f)}} reorders data items
in a container: 
\[
\text{Container}^{A}\equiv A\times A;\qquad\text{fmap}^{A,B}\,(f^{A\Rightarrow B})(x^{A},y^{A})=\left(f(y),f(x)\right)
\]
or swap some elements in $A\times A\times A$:
\begin{lyxcode}
\textcolor{blue}{\footnotesize{}def~fmap(f:~A$\Rightarrow$B):~Vec3{[}A{]}~$\Rightarrow$~Vec3{[}B{]}~=}{\footnotesize \par}

\textcolor{blue}{\footnotesize{}~~\{~case~Vec3(x,~y,~z)~=~Vec3(f(y),~f(x),~f(z))~\}}{\footnotesize \par}
\end{lyxcode}
\item Does a special computation if types are equal: if \texttt{\textcolor{blue}{\footnotesize{}A}}
and \texttt{\textcolor{blue}{\footnotesize{}B}} are the same type,
do \texttt{\textcolor{blue}{\footnotesize{}fmap{[}A, A{]}(f) = identity}},
otherwise $f(x)$ is applied
\item Does a special computation if type is equal to a specific type, e.g.\ if
\texttt{\textcolor{blue}{\footnotesize{}A = B = Int}} then do $f(f(x))$
else $f(x)$
\item Does a special computation if $f$ is equal to some $f_{0}$, otherwise
use $f(x)$
\end{enumerate}
See test code
\end{frame}

\begin{frame}{Recursive polynomial types as functors}

\begin{itemize}
\item Example: List of pairs defined as a recursive type,
\[
LP^{A}\equiv1+A\times A\times LP^{A}
\]
\end{itemize}
\begin{lyxcode}
\textcolor{blue}{\footnotesize{}sealed~trait~LP{[}A{]}}{\footnotesize \par}

\textcolor{blue}{\footnotesize{}final~case~class~LPempty{[}A{]}()~extends~LP{[}A{]}}{\footnotesize \par}

\textcolor{blue}{\footnotesize{}final~case~class~LPpair{[}A{]}(x:~A,~y:~A,~tail:~LP{[}A{]})~extends~LP{[}A{]}}{\footnotesize \par}
\end{lyxcode}
\begin{itemize}
\item We can implement \texttt{\textcolor{blue}{\footnotesize{}map}} as
a recursive function:a
\end{itemize}
\begin{lyxcode}
\textcolor{blue}{\footnotesize{}def~map{[}A,~B{]}(lp:~LP{[}A{]},~f:~A~$\Rightarrow$~B):~LP{[}B{]}~=~lp~match~\{}{\footnotesize \par}

\textcolor{blue}{\footnotesize{}~~case~LPempty()~$\Rightarrow$~LPempty{[}B{]}()}{\footnotesize \par}

\textcolor{blue}{\footnotesize{}~~case~LPpair(x,~y,~tail)~$\Rightarrow$~LPpair{[}B{]}(f(x),~f(y),~map(tail,~f))}{\footnotesize \par}

\textcolor{blue}{\footnotesize{}\}}{\footnotesize \par}
\end{lyxcode}
\begin{itemize}
\item This is the only way to implement \texttt{\textcolor{blue}{\footnotesize{}map}}
that satisfies the functor laws!
\end{itemize}
See test code for checking the functor laws
\end{frame}

\begin{frame}{Contrafunctors}

\begin{itemize}
\item The type constructor $C^{A}\equiv A\Rightarrow\text{Int}$ is not
a functor (impossible to implement \texttt{\textcolor{blue}{\footnotesize{}map}}),
but we can implement \texttt{\textcolor{blue}{\footnotesize{}contrafmap}}:
\[
\text{contrafmap}^{A,B}:\left(B\Rightarrow A\right)\Rightarrow C^{A}\Rightarrow C^{B}
\]
\item The contrafunctor laws are analogous to functor laws:\texttt{\textcolor{blue}{\footnotesize{}
\begin{align*}
\text{contrafmap}^{A,A}(\text{id}^{A}) & =\text{id}^{C^{A}}\\
\text{contrafmap}\left(g\circ f\right) & =\text{contrafmap}\left(f\right)\circ\text{contrafmap}\left(g\right)
\end{align*}
}}The ``contra-'' reverses the arrow between $A$ and $B$
\item The type parameter $A$ is to the left of the function arrow (``consumed'')
\item ``Functors contain data; contrafunctors consume data''
\end{itemize}
Example of non-contrafunctor:
\begin{itemize}
\item The type $\text{NotContainer}^{A}=\left(A\Rightarrow\text{Int}\right)\times A$
is neither a functor nor a contrafunctor
\end{itemize}
\end{frame}

\begin{frame}{Covariance and contravariance}


\framesubtitle{Functors, contrafunctors, and subtyping }

Example of subtyping:
\begin{lyxcode}
\textcolor{blue}{\footnotesize{}sealed~trait~AtMostTwo}{\footnotesize \par}

\textcolor{blue}{\footnotesize{}final~case~class~Zero()~extends~AtMostTwo}{\footnotesize \par}

\textcolor{blue}{\footnotesize{}final~case~class~One(x:~Int)~extends~AtMostTwo}{\footnotesize \par}

\textcolor{blue}{\footnotesize{}final~case~class~Two(x:~Int,~y:~Int)~extends~AtMostTwo}{\footnotesize \par}
\end{lyxcode}
\begin{itemize}
\item Here \texttt{\textcolor{blue}{\footnotesize{}Zero}}, \texttt{\textcolor{blue}{\footnotesize{}One}},
and \texttt{\textcolor{blue}{\footnotesize{}Two}} are \textbf{subtypes}
of \texttt{\textcolor{blue}{\footnotesize{}AtMostTwo}}{\footnotesize \par}
\item We can pass \texttt{\textcolor{blue}{\footnotesize{}Two(10, 20)}}
to a function that takes an \texttt{\textcolor{blue}{\footnotesize{}AtMostTwo}}{\footnotesize \par}
\item This is equivalent to an automatic type conversion \texttt{\textcolor{blue}{\footnotesize{}Two
$\Rightarrow$ AtMostTwo}}{\footnotesize \par}
\item A container \texttt{\textcolor{blue}{\footnotesize{}C{[}A{]}}} is
\textbf{covariant} if \texttt{\textcolor{blue}{\footnotesize{}C{[}Two{]}}}
is a subtype of \texttt{\textcolor{blue}{\footnotesize{}C{[}AtMostTwo{]}}}{\footnotesize \par}
\begin{itemize}
\item And then a type conversion function\texttt{\textcolor{blue}{\footnotesize{}
C{[}Two{]} $\Rightarrow$ C{[}AtMostTwo{]}}} exists
\end{itemize}
\item More generally, when \texttt{\textcolor{blue}{\footnotesize{}X}} is
a subtype of \texttt{\textcolor{blue}{\footnotesize{}Y}} then we have
\texttt{\textcolor{blue}{\footnotesize{}X $\Rightarrow$ Y}} and we
need \texttt{\textcolor{blue}{\footnotesize{}C{[}X{]} $\Rightarrow$
C{[}Y{]}}}, which is guaranteed if we have a function of type
\[
\left(A\Rightarrow B\right)\Rightarrow(C^{A}\Rightarrow C^{B})
\]
\end{itemize}
Functors are covariant, contrafunctors are contravariant
\end{frame}

\begin{frame}{Worked examples}

\begin{itemize}
\item Decide if a type constructor is a functor, a contrafunctor, or neither
\item Implement a \texttt{\textcolor{blue}{\footnotesize{}map}} or a \texttt{\textcolor{blue}{\footnotesize{}contramap}}
function that satisfies the laws
\end{itemize}
\begin{enumerate}
\item Define case classes for these type constructors, and implement \texttt{\textcolor{blue}{\footnotesize{}map}}:
\begin{enumerate}
\item $\text{Data}^{A}\equiv\text{String}+A\times\text{Int}+A\times A\times A$
\item $\text{Data}^{A}\equiv1+A\times(\text{Int}\times\text{String}+A)$
\item $\text{Data}^{A}\equiv\left(\text{String}\Rightarrow\text{Int}\Rightarrow A\right)\times A+\left(\text{Boolean}\Rightarrow\text{Double}\Rightarrow A\right)\times A$
\end{enumerate}
\item Decide which of these type constructors are functors or contrafunctors,
and implement \texttt{\textcolor{blue}{\footnotesize{}fmap}} or \texttt{\textcolor{blue}{\footnotesize{}contrafmap}}
respectively:
\begin{enumerate}
\item $\text{Data}^{A}\equiv\left(A\Rightarrow\text{Int}\right)+\left(A\Rightarrow A\Rightarrow\text{String}\right)$
\item $\text{Data}^{A,B}\equiv\left(A+B\right)\times\left(\left(A\Rightarrow\text{Int}\right)\Rightarrow B\right)$
\end{enumerate}
\item Rewrite this code in the short notation; identify covariant and contravariant
type parameters; verify that with Scala:
\begin{lyxcode}
\textcolor{blue}{\footnotesize{}sealed~trait~Coi{[}A,~B{]}}{\footnotesize \par}

\textcolor{blue}{\footnotesize{}case~class~Pa{[}A,~B{]}(b:~(A,~B),~c:~B$\Rightarrow$Int)~~extends~Coi{[}A,~B{]}}{\footnotesize \par}

\textcolor{blue}{\footnotesize{}case~class~Re{[}A,~B{]}(d:~A,~e:~B,~c:~Int)~~~~extends~Coi{[}A,~B{]}}{\footnotesize \par}

\textcolor{blue}{\footnotesize{}case~class~Ci{[}A,~B{]}(f:~String$\Rightarrow$A,~g:~B$\Rightarrow$A)~extends~Coi{[}A,~B{]}}{\footnotesize \par}
\end{lyxcode}
\end{enumerate}
\end{frame}

\begin{frame}{Exercises}

Define case classes for these type constructors, decide if they are
covariant or contravariant, and implement \texttt{\textcolor{blue}{\footnotesize{}map}}
or \texttt{\textcolor{blue}{\footnotesize{}contramap}} as needed:
\begin{enumerate}
\item $\text{Data}^{A}\equiv\left(1+A\right)\times\left(1+A\right)\times\text{String}$
\item $\text{Data}^{A}\equiv\left(A\Rightarrow\text{String}\right)\Rightarrow\left(A\times\left(\text{Int}+A\right)\right)$
\item $\text{Data}^{A,B}\equiv\left(A\Rightarrow\text{String}\right)\times\left(\left(A+B\right)\Rightarrow\text{Int}\right)$
\item $\text{Data}^{A}\equiv\left(1+\left(A\Rightarrow\text{String}\right)\right)\Rightarrow\left(1+\left(A\Rightarrow\text{Int}\right)\right)\Rightarrow\text{Int}$
\item $\text{Data}^{B}\equiv\left(B+\left(\text{Int}\Rightarrow B\right)\right)\times\left(B+\left(\text{String}\Rightarrow B\right)\right)$
\item Rewrite this code in the short notation; identify covariant and contravariant
type parameters; verify that with Scala:
\end{enumerate}
\begin{lyxcode}
\textcolor{blue}{\footnotesize{}sealed~trait~Result{[}A{]}}{\footnotesize \par}

\textcolor{blue}{\footnotesize{}case~class~P{[}A{]}(a:~A,~b:~String,~c:~Int)~~~extends~Result{[}A{]}}{\footnotesize \par}

\textcolor{blue}{\footnotesize{}case~class~Q{[}A{]}(d:~Int$\Rightarrow$A,~e:~Int$\Rightarrow$String)~extends~Result{[}A{]}}{\footnotesize \par}

\textcolor{blue}{\footnotesize{}case~class~R{[}A{]}(f:~A$\Rightarrow$A,~g:~A$\Rightarrow$String)~~~~~extends~Result{[}A{]}}{\footnotesize \par}
\end{lyxcode}
\end{frame}

\end{document}
