\batchmode
\makeatletter
\def\input@path{{/Users/sergei.winitzki/Code/talks/ftt-fp/}}
\makeatother
\documentclass[english]{beamer}
\usepackage[T1]{fontenc}
\usepackage[latin9]{inputenc}
\setcounter{secnumdepth}{3}
\setcounter{tocdepth}{3}
\usepackage{babel}
\usepackage{amstext}
\ifx\hypersetup\undefined
  \AtBeginDocument{%
    \hypersetup{unicode=true,pdfusetitle,
 bookmarks=true,bookmarksnumbered=false,bookmarksopen=false,
 breaklinks=false,pdfborder={0 0 1},backref=false,colorlinks=true}
  }
\else
  \hypersetup{unicode=true,pdfusetitle,
 bookmarks=true,bookmarksnumbered=false,bookmarksopen=false,
 breaklinks=false,pdfborder={0 0 1},backref=false,colorlinks=true}
\fi

\makeatletter
%%%%%%%%%%%%%%%%%%%%%%%%%%%%%% Textclass specific LaTeX commands.
 % this default might be overridden by plain title style
 \newcommand\makebeamertitle{\frame{\maketitle}}%
 % (ERT) argument for the TOC
 \AtBeginDocument{%
   \let\origtableofcontents=\tableofcontents
   \def\tableofcontents{\@ifnextchar[{\origtableofcontents}{\gobbletableofcontents}}
   \def\gobbletableofcontents#1{\origtableofcontents}
 }
 \newenvironment{lyxcode}
   {\par\begin{list}{}{
     \setlength{\rightmargin}{\leftmargin}
     \setlength{\listparindent}{0pt}% needed for AMS classes
     \raggedright
     \setlength{\itemsep}{0pt}
     \setlength{\parsep}{0pt}
     \normalfont\ttfamily}%
    \def\{{\char`\{}
    \def\}{\char`\}}
    \def\textasciitilde{\char`\~}
    \item[]}
   {\end{list}}

%%%%%%%%%%%%%%%%%%%%%%%%%%%%%% User specified LaTeX commands.
\usetheme[secheader]{Boadilla}
\usecolortheme{seahorse}
\title[Chapter 7: Functor-lifted computations II]{Chapter 7: Computations lifted to a functor context II. Monads and semimonads}
\subtitle{Part 1: Practical work with monads and semimonads}
\author{Sergei Winitzki}
\date{2018-03-11}
\institute[ABTB]{Academy by the Bay}
\setbeamertemplate{headline}{} % disable headline at top
\setbeamertemplate{navigation symbols}{} % disable navigation bar at bottom
\usepackage[all]{xy}
\makeatletter
% Macros to assist LyX with XYpic when using scaling.
\newcommand{\xyScaleX}[1]{%
\makeatletter
\xydef@\xymatrixcolsep@{#1}
\makeatother
} % end of \xyScaleX
\makeatletter
\newcommand{\xyScaleY}[1]{%
\makeatletter
\xydef@\xymatrixrowsep@{#1}
\makeatother
} % end of \xyScaleY

\makeatother

\begin{document}
\frame{\titlepage}
\begin{frame}{Computations within a functor context: Semimonads}


\framesubtitle{Intuitions behind adding more ``generator arrows''}

Example: {\footnotesize{}
\[
\sum_{i=1}^{n}\sum_{j=1}^{n}\sum_{k=1}^{n}f(i,j,k)
\]
}{\footnotesize \par}

Using Scala's \texttt{\textcolor{blue}{\footnotesize{}for}}/\texttt{\textcolor{blue}{\footnotesize{}yield}}
syntax (``functor block'')

\texttt{\textcolor{blue}{\footnotesize{}}}%
\begin{minipage}[t]{0.49\columnwidth}%
\begin{lyxcode}
\textcolor{blue}{\footnotesize{}(for~\{~i~$\leftarrow$~1~to~n}{\footnotesize \par}

\textcolor{blue}{\footnotesize{}~~~~j~$\leftarrow$~1~to~n}{\footnotesize \par}

\textcolor{blue}{\footnotesize{}~~~~k~$\leftarrow$~1~to~n}{\footnotesize \par}

\textcolor{blue}{\footnotesize{}~~\}~yield~f(i,~j,~k)}{\footnotesize \par}

\textcolor{blue}{\footnotesize{}).sum}{\footnotesize \par}
\end{lyxcode}
%
\end{minipage}\texttt{\textcolor{blue}{\footnotesize{}}}%
\begin{minipage}[t]{0.49\columnwidth}%
\begin{lyxcode}
\textcolor{blue}{\footnotesize{}(1~to~m).flatMap~\{~i~$\Rightarrow$}{\footnotesize \par}

\textcolor{blue}{\footnotesize{}~~~(1~to~n).flatMap~\{~j~$\Rightarrow$}{\footnotesize \par}

\textcolor{blue}{\footnotesize{}~~~~~(1~to~n).map~\{~k~$\Rightarrow$}{\footnotesize \par}

\textcolor{blue}{\footnotesize{}~~~~~~~f(i,~j,~k)}{\footnotesize \par}

\textcolor{blue}{\footnotesize{}~~\}\}\}.sum}{\footnotesize \par}
\end{lyxcode}
%
\end{minipage}\texttt{\textcolor{blue}{\footnotesize{}\medskip{}
}}{\footnotesize \par}
\begin{itemize}
\item \texttt{\textcolor{blue}{\footnotesize{}map}} replaces the last left
arrow, \texttt{\textcolor{blue}{\footnotesize{}flatMap}} replaces
other left arrows
\begin{itemize}
\item When the functor is \emph{also} filterable, we can use ``\texttt{\textcolor{blue}{\footnotesize{}if}}''
as well
\end{itemize}
\item Standard library defines \texttt{\textcolor{blue}{\footnotesize{}flatMap()}}
as replacement of \texttt{\textcolor{blue}{\footnotesize{}map() $\circ$
flatten}} 
\begin{itemize}
\item \texttt{\textcolor{blue}{\footnotesize{}(1 to n).map(j $\Rightarrow$
...).flatten}} is \texttt{\textcolor{blue}{\footnotesize{}(1 to n).flatMap(j
$\Rightarrow$ ...)}} 
\end{itemize}
\item Functors having \texttt{\textcolor{blue}{\footnotesize{}flatMap}}/\texttt{\textcolor{blue}{\footnotesize{}flatten}}
are ``flattenable'' or \textbf{semimonads}
\begin{itemize}
\item Most of them also have method \texttt{\textcolor{blue}{\footnotesize{}pure:\ A
$\Rightarrow$ F{[}A{]}}} and so are \textbf{monads}
\begin{itemize}
\item The method \texttt{\textcolor{blue}{\footnotesize{}pure}} is not relevant
in the functor block
\item We will not need \texttt{\textcolor{blue}{\footnotesize{}pure}} in
this part of the tutorial; focus on semimonads
\end{itemize}
\end{itemize}
\end{itemize}
\end{frame}

\begin{frame}{What is \texttt{\textcolor{blue}{\footnotesize{}flatMap}} doing with
the data in a collection?}

Consider this schematic code using \texttt{\textcolor{blue}{\footnotesize{}Seq}}
as the container type:\texttt{\textcolor{blue}{\footnotesize{} }}%
\begin{minipage}[t]{0.49\columnwidth}%
\begin{lyxcode}
\textcolor{blue}{\footnotesize{}val~result~=~for~\{}{\footnotesize \par}

\textcolor{blue}{\footnotesize{}~~i~$\leftarrow$~1~to~m}{\footnotesize \par}

\textcolor{blue}{\footnotesize{}~~j~$\leftarrow$~1~to~n}{\footnotesize \par}

\textcolor{blue}{\footnotesize{}~~x~=~f(i,j)}{\footnotesize \par}

\textcolor{blue}{\footnotesize{}~~k~$\leftarrow$~1~to~p}{\footnotesize \par}

\textcolor{blue}{\footnotesize{}~~y~=~g(i,j,k)}{\footnotesize \par}

\textcolor{blue}{\footnotesize{}\}~yield~h(x,y)}{\footnotesize \par}
\end{lyxcode}
%
\end{minipage}\texttt{\textcolor{blue}{\footnotesize{}}}%
\begin{minipage}[t]{0.49\columnwidth}%
\begin{lyxcode}
\textcolor{blue}{\footnotesize{}val~result~=~\{}{\footnotesize \par}

\textcolor{blue}{\footnotesize{}~~(1~to~m).flatMap~\{~i~$\Rightarrow$}{\footnotesize \par}

\textcolor{blue}{\footnotesize{}~~~~(1~to~n).flatMap~\{~j~$\Rightarrow$}{\footnotesize \par}

\textcolor{blue}{\footnotesize{}~~~~~~val~x~=~f(i,j)}{\footnotesize \par}

\textcolor{blue}{\footnotesize{}~~~~~~(1~to~p).map~\{~k~$\Rightarrow$}{\footnotesize \par}

\textcolor{blue}{\footnotesize{}~~~~~~~~val~y~=~g(i,j,k)}{\footnotesize \par}

\textcolor{blue}{\footnotesize{}~~~~~~~~h(x,y)~~\}~\}~\}~\}}{\footnotesize \par}
\end{lyxcode}
%
\end{minipage}\texttt{\textcolor{blue}{\footnotesize{}\medskip{}
}}{\footnotesize \par}

Computations are repeated for all $i$, for all $j$, etc., from each
collection
\begin{itemize}
\item All ``generator lines'' must use the same container type
\begin{itemize}
\item Each generator line finally computes a container of \emph{the} \emph{same}
type
\item The total number of resulting data items is $\leq m*n*p$\texttt{\textcolor{blue}{\footnotesize{} }}{\footnotesize \par}
\item All the resulting data items must fit within \emph{the same} container
type!
\item {\footnotesize{}The set of }\emph{\footnotesize{}container capacity
counts}{\footnotesize{} must be closed under multiplication}{\footnotesize \par}
\end{itemize}
\item What container types have this property?
\begin{itemize}
\item \texttt{\textcolor{blue}{\footnotesize{}Seq,}} \texttt{\textcolor{blue}{\footnotesize{}NonEmptyList}}
\textendash{} can hold \emph{any} number of elements $\geq$ min.\ count
\item \texttt{\textcolor{blue}{\footnotesize{}Option}}, \texttt{\textcolor{blue}{\footnotesize{}Either}},
\texttt{\textcolor{blue}{\footnotesize{}Try}}, \texttt{\textcolor{blue}{\footnotesize{}Future}}
\textendash{} can hold $0$ or $1$ elements (``pass/fail'')
\item ``Tree-like'' containers, e.g.\ can hold only $3$, $6$, $9$,
$12$, ...\ elements
\item ``Non-standard'' containers: $F^{A}\equiv\text{String}\Rightarrow A$;
$F^{A}\equiv\left(A\Rightarrow\text{Int}\right)\Rightarrow\text{Int}$
\end{itemize}
\end{itemize}
\end{frame}

\begin{frame}{Worked examples: List-like monads}


\framesubtitle{\texttt{\footnotesize{}Seq}, \texttt{\footnotesize{}NonEmptyList},
\texttt{\footnotesize{}Iterator}, \texttt{\footnotesize{}Stream}}

Typical tasks for ``list-like'' monads:
\begin{itemize}
\item Create a list of all combinations or all permutations of a sequence
\item Traverse a ``solution tree'' with DFS and filter out incorrect solutions
\begin{itemize}
\item Can use eager (\texttt{\textcolor{blue}{\footnotesize{}Seq}}) or lazy
(\texttt{\textcolor{blue}{\footnotesize{}Iterator}}, \texttt{\textcolor{blue}{\footnotesize{}Stream}})
evaluation strategies
\item Usually, list-like containers have many additional methods
\begin{itemize}
\item append, prepend, concat, fill, fold, scan, etc.
\end{itemize}
\end{itemize}
\end{itemize}
Worked examples: see code
\begin{enumerate}
\item All permutations of \texttt{\textcolor{blue}{\footnotesize{}Seq(\textquotedbl{}a\textquotedbl{},
\textquotedbl{}b\textquotedbl{}, \textquotedbl{}c\textquotedbl{})}} 
\item All subsets of \texttt{\textcolor{blue}{\footnotesize{}Set(\textquotedbl{}a\textquotedbl{},
\textquotedbl{}b\textquotedbl{}, \textquotedbl{}c\textquotedbl{})}} 
\item All subsequences of length 3 out of a given sequence
\item Generalize examples 1-3 to support arbitrary length $n$ instead of
3
\item All solutions of the ``8 queens'' problem
\item Generalize example 5 to solve $n$-queens problem
\item Transform Boolean formulas between CNF and DNF 
\end{enumerate}
\end{frame}

\begin{frame}{Intuitions for pass/fail monads}


\framesubtitle{\texttt{\footnotesize{}Option}, \texttt{\footnotesize{}Either}, \texttt{\footnotesize{}Try},
\texttt{\footnotesize{}Future}}
\begin{itemize}
\item Container $F^{A}$ can hold $n=1$ or $n=0$ values of type $A$
\item Such containers will have methods to create ``pass'' and ``ethods to create ``pass'' and ``fail''
values
\end{itemize}
Schematic example of a functor block program using the \texttt{\textcolor{blue}{\footnotesize{}Try}}
functor:
\begin{lyxcode}
\textcolor{blue}{\footnotesize{}val~result:~Try{[}A{]}~=~for~\{~}\textrm{\textcolor{darkgray}{\footnotesize{}//~computations~in~the~}}\textcolor{darkgray}{\footnotesize{}Try}\textrm{\textcolor{darkgray}{\footnotesize{}~functor}}{\footnotesize \par}

\textcolor{blue}{\footnotesize{}~~x~$\leftarrow$~Try(...)~}\textrm{\textcolor{darkgray}{\footnotesize{}//~first~computation;~may~fail}}{\footnotesize \par}

\textcolor{blue}{\footnotesize{}~~y~=~f(x)~}\textrm{\textcolor{darkgray}{\footnotesize{}//~no~possibility~of~failure~in~this~line}}{\footnotesize \par}

\textcolor{blue}{\footnotesize{}~~if~p(y)~}\textrm{\textcolor{darkgray}{\footnotesize{}//~the~entire~expression~will~fail~if~this~is~false}}{\footnotesize \par}

\textcolor{blue}{\footnotesize{}~~z~$\leftarrow$~Try(g(x,~y))}\textrm{\textcolor{darkgray}{\footnotesize{}~//~may~fail~here}}{\footnotesize \par}

\textcolor{blue}{\footnotesize{}~~r~$\leftarrow$~Try(...)}\textrm{\textcolor{darkgray}{\footnotesize{}~//~may~fail~here~as~well}}{\footnotesize \par}

\textcolor{blue}{\footnotesize{}\}~yield~r~}\textrm{\textcolor{darkgray}{\footnotesize{}//~~}}\textcolor{darkgray}{\footnotesize{}r}\textrm{\textcolor{darkgray}{\footnotesize{}~is~of~type~}}\textcolor{darkgray}{\footnotesize{}A}\textrm{\textcolor{darkgray}{\footnotesize{},~so~}}\textcolor{darkgray}{\footnotesize{}result}\textrm{\textcolor{darkgray}{\footnotesize{}~is~of~type~}}\textcolor{darkgray}{\footnotesize{}Try{[}A{]}}{\footnotesize \par}
\end{lyxcode}
\begin{itemize}
\item Computations may yield a result ($n=1$), or may fail ($n=0$)
\item The functor block chains several such computations \emph{sequentially}
\begin{itemize}
\item Computations are sequential even if using the \texttt{\textcolor{blue}{\footnotesize{}Future}}
functor!
\item Once any computation fails, the entire functor block fails ($0*n=0$)
\item Only if \emph{all} computations succeed, the functor block returns
\emph{one} value
\item Filtering can also make the entire expression fail
\end{itemize}
\item ``Flat'' functor block replaces a chain of nested \texttt{\textcolor{blue}{\footnotesize{}if/else}}
or \texttt{\textcolor{blue}{\footnotesize{}match/case}}{\footnotesize \par}
\end{itemize}
\end{frame}

\begin{frame}{Worked examples: Pass/fail monads}

Type constructors:
\begin{itemize}
\item \texttt{\textcolor{blue}{\footnotesize{}Option{[}A{]}}} $\equiv1+A$
\item \texttt{\textcolor{blue}{\footnotesize{}Either{[}Z, A{]}}} $\equiv Z+A$
\item \texttt{\textcolor{blue}{\footnotesize{}Try{[}A{]}}} $\equiv$\texttt{\textcolor{blue}{\footnotesize{}
Either{[}Throwable, A{]}}} 
\end{itemize}
Typical tasks for pass/fail monads:
\begin{itemize}
\item Perform a linear sequence of computations that may fail
\item Avoid crashing on failure, instead return an \emph{error value}
\end{itemize}
Worked examples: see code
\begin{enumerate}
\item Read values of Java properties, checking that they all exist
\item Obtain values from \texttt{\textcolor{blue}{\footnotesize{}Future}}
computations in sequence
\item Make arithmetic safe by returning error messages in \texttt{\textcolor{blue}{\footnotesize{}Either}} 
\item Fail less: chain computations that may throw an exception
\end{enumerate}
\end{frame}

\begin{frame}{Worked examples: Tree-like monads}

Examples of tree-like type constructors:
\begin{itemize}
\item $F^{A}\equiv A+F^{A}\times F^{A}$ (binary tree, recursive)
\item $F^{A}\equiv A+S^{F^{A}}$ ($S$-shaped tree, where $S$ is a functor)
\item $F^{A}\equiv A\times A+F^{A}\times F^{A}$ (binary tree with binary
leaves)
\item $F^{A}\equiv S^{A}+S^{F^{A}}$ ($S$-shaped tree with $S$-shaped
leaves)
\end{itemize}
Typical tasks for tree-like monads:
\begin{itemize}
\item Traverse a tree, graft subtrees at leaves
\item Substitute subexpressions in a syntax tree
\end{itemize}
Worked examples: see code
\begin{enumerate}
\item Implement a tree of \texttt{\textcolor{blue}{\footnotesize{}String}}
properties with arbitrary branching
\item Implement variable substitution for a simple arithmetic language
\end{enumerate}
Example of a \emph{non-tree-like} type constructor:
\begin{itemize}
\item $F^{A}\equiv A+A\times A+A\times A\times A\times A+...$ (powers of
2, \emph{non-recursive})
\end{itemize}
\end{frame}

\begin{frame}{Worked examples: Single-value monads}


\framesubtitle{\texttt{\footnotesize{}Reader}, \texttt{\footnotesize{}Writer}, \texttt{\footnotesize{}Eval},
\texttt{\footnotesize{}Cont}, \texttt{\footnotesize{}State}}
\begin{itemize}
\item Container holds exactly $1$ value, together with a ``context''
\item Usually, methods exist to insert a value and to work with the ``context''
\end{itemize}
Typical tasks for single-value monads:
\begin{itemize}
\item Collecting extra information about computations along the way
\item Chaining computations with a nonstandard evaluation strategy
\end{itemize}
Examples: see code
\begin{enumerate}
\item Perform computations and log information about each step
\item Dependency injection with the \texttt{\textcolor{blue}{\footnotesize{}Reader}}
monad
\item Perform lazy or memoized computations in a sequence
\item A chain of asynchronous operations
\item A sequence of steps that update state while returning results
\end{enumerate}
\end{frame}

\begin{frame}{Exercises I}
\begin{enumerate}
\item Compute all subsequences of length 3 out of the sequence \texttt{\textcolor{blue}{\footnotesize{}(1
to m)}} 
\end{enumerate}
\end{frame}

\end{document}
